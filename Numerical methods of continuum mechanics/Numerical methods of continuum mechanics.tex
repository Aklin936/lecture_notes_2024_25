\documentclass[14pt]{extarticle}

\usepackage[T1,T2A]{fontenc}
\usepackage[utf8]{inputenc}
\usepackage[english,russian]{babel}
\usepackage{multirow}

\usepackage{caption}

\usepackage{amssymb}

\usepackage{amsmath}
\usepackage[hidelinks]{hyperref}


\usepackage{graphicx}%Вставка картинок
\graphicspath{{noiseimages/}}

\usepackage{float}%"Плавающие" картинки
\usepackage{wrapfig}%Обтекание фигур (таблиц, картинок и прочего)

\begin{document}

\section{Уравнение конвекции с диффузией}

\[ \frac{\partial Q}{\partial t} +\frac{\partial }{\partial x} (u Q) = \frac{\partial }{\partial x} (\mathcal{D} \frac{\partial Q}{\partial x}); t>0; x \in [-L, L] \]

Уравнение конвекции с диффузией.

\[ Pe = \frac{UL}{\mathcal{D}} \] - число Пекле

\[ \frac{L}{U} = t_{\textrm{конв}} \]
\[ \frac{L^2}{\mathcal{D}} = t_{\textrm{дифузионная}} \]
\[ \frac{t_{\textrm{диф}}}{t_{\textrm{конв}}} = \frac{L U}{\mathcal{D}} \]

\[ Re << 1 \] почти дифф процесс
\[ Re >> 1 \] почти конв процесс

Сначала будем рассматривать стационарное уравнение. 

\[ \frac{d}{dx}(uQ) - \frac{d}{dx} (\mathcal{D} \frac{d Q}{d x}) = 0 \]

На отрезке вводится разностная сетка \[ \Omega_n (x_{i-1}, x_{i}, x_{i+1})\].

Потоковые точки - точки с полуцелые точки

\[ x{i+\frac{1}{2}} = \frac{x_i + x_{i+1}}{2} \] - потоковые узлы

\[ h_{i+\frac{1}{2}} = x_{i+1} - x_{i} \]

\[ h_{i} = \frac{h_{i+\frac{1}{2}} - h_{i-\frac{1}{2}}}{2} \]

Вводим сеточные функции.

\[ Q_i = (x_{i+\frac{1}{2}}; x_{i-\frac{1}{2}}) \]
константа на отрезке
\[ [x_{i+\frac{1}{2}} \textrm{до} x_{i-\frac{1}{2}}] \]

D - константа в полуцелых точках (от $ x_i $ до  $ x_{i+1} $ константа)

Считаем скорость и тепло в потоковых точках, где диффузия постоянна

\[ Q_x = \frac{Q_{i+1}i = Q_i}{h_{i+\frac{1}{2}}} \]
\[ Q_{\overline{x}} = \frac{Q_{i} - Q_{i-1}}{h_{i-\frac{1}{2}}} \]
\[ Q_{\widehat{x}} = \frac{Q_{i+\frac{1}{2}} - Q_{i-\frac{1}{2}}}{h_{i}} \]

Между потоковыми точками интегрируем получаем формулу.

\[ \int_{x_{i}+\frac{1}{2}}^{x_{i}+\frac{1}{2}} \frac{d}{d x} [uQ - \mathcal{D}\frac{dQ}{dx}]dx = \]
\[ = [uQ - \mathcal{D}\frac{dQ}{dx}]|_{x_{i} - \frac{1}{2}}^{x_i + \frac{1}{2} } = \]
\[ W^{tot}_{i + \frac{1}{2}} \]

\[ W^{tot}_{i-\frac{1}{2}} = U_{i+\frac{1}{2}} Q_{i+\frac{1}{2}} = W^D_{i+\frac{1}{2}} - W^D_{i-\frac{1}{2}} \]

\[ W^D_{i+\frac{1}{2}} = - D_{i+\frac{1}{2}} \frac{dQ}{dx}|_{i+\frac{1}{2}} \]

\[ U_i{i+1/2} Q_{i+1/2} - D_{i+1/2}\frac{Q_{i+1} - Q_i}{h_{i+\frac{1}{2}}} -\]
\[ - [U_i{i-1/2} Q_{i-1/2} - D_{i-1/2} Q_{{\overline{x}}_{i-1/2}}]= 0 \]

\[ \frac{U_{i+1/2} Q_{i+1/2}  - U_{i-1/2} Q_{i-1/2}}{\overline{h_i}} -\]
\[-\frac{D_{i+1/2} Q_x - D_{i-1/2} Q_{\overline{x}}}{\overline{h_i}} = 0 \]

Интерполируем Q в полуцелые точки.

\[ Q_{i+1/2} = \Theta_{i+1/2} Q_i + (1-Q_{i+1/2}) \Theta_{i+1} \]

\begin{enumerate}
	\item \[ \Theta = 1/2 \]

	\[ W = .... \]

	Получится схема с центральными разностями

	\[ u Q_x - D Q_{x with lint x} = 0 \]

	\item \[ \Theta_{i+1/2} = 1/2(1+\frac{|u_{i+1/2}|}{U_{i+1/2}}) \]

	\[ W_{i+1/2} = u^+_{i+1/2} Q_i - \overline{U}_{i+1/2} Q_{i+1} \]
	\[ U'_{i+1/2} = \frac{1}{2} (U_{i+1/2} + |U_{i+1/2}|) \]
	\[ U with lin_{i+1/2} = \frac{1}{2} (U_{i+1/2} - |U_{i+1/2}|) \]

	\[ (u^+_{i+1/2} Q_i + \overline{u}^+_{i+1/2} Q_i+1) - D Q_{\overline{x} \widehat{x}} = 0 \]

	\section{Модельная задача}

	\[ \frac{d(UQ)}{dx} - D \frac{d^2Q}{dx^2} = 0 \]

\end{enumerate}

При увеличении числа Пeкле график прижимается к координатам снизу.

\section{Решение разностной задачи (центральные разности).}

\[ U \frac{Q_{i+1} - Q_{i-1}}{2h} - \frac{Q_{i+1} - 2Q_{i} - Q_{i-1}}{h^2} = 0\]

\[ Q_i = a + b q^i \]

\[ U \frac{Q_{i+1} - Q_i}{2h} - \frac{Q_{i+1} - 2Q_{i} - Q_{i-1}}{h^2} = 0\]

Подставляем

\[ \frac{U}{2h} b(q^{i+1} - q^{i-1}) -  D b \frac{q^{i+1}-2q^i+q^{i-1}}{h^2} = 0 \]

\[ \frac{u}{2} (q^2-1) - \frac{D}{h}(q^2-2q+1) = 0 \]

\section{Лекция 2}

Точное решение

\[ Q(x) = \frac{1-e^{\frac{Ux}{\mathcal{D}}}}{1-e^{\frac{U}{\mathcal{D}}}} = \]

\[ Q(x) = \frac{1-e^{Pe \cdot i}}{1-e^{Pe_n N}}=  \]
\[ Q(x) = \frac{1-\overline{q}^i}{1-\overline{q}^N} \]

\[ \overline{q} = e^{Pe_N} = 1+ \frac{Pe_n + Pe_n^2}{2}\]

Сеточная функция (Центральные разности)

\[ q = \frac{1+ \frac{Pe}{2}}{1- \frac{Pe}{2}} =  \]
\[ (1+ \frac{Pe_N}{2}(1+ \frac{1}{2} {Pe}_n + \frac{1}{4}{Pe}^2_n+...)) =  \]
\[ 1+ {Pe}_n + \frac{{Pe}^2_n}{2} \]

Но число пекре должно быть меньше 2

Направленные разности

\[  q + {Pe}_n \]

Для схемы с центральными разностями

\[ U \frac{Q_{i+1} - Q_i}{2h} - \frac{Q_{i+1} - 2Q_{i} - Q_{i-1}}{h^2} = 0\]
\[ Q_{i+1} (\frac{u}{2h} - \frac{\mathcal{d}}{h^2}) + Q_{i} (\frac{2\mathcal{d}}{h^2}) + Q_{i-1} (-\frac{u}{2h} - \frac{\mathcal{d}}{h^2}) = 0 \]
\[ - Q_{i+1} (1 - \frac{1}{2} \mathcal{D}) + 2Q_{i} - Q_{i-1} (1 - \frac{1}{2} {Pe}_n) = 0 \]

Для схеммы с направленными разностями

\[ U \frac{Q_{i} - Q_{i-1}}{h} - \mathcal{D}\frac{Q_{i+1} - 2Q_{i} + Q_{i-1}}{h^2} = 0\]
\[ Q_{i-1} (-\frac{u}{h}-\frac{\mathcal{D}}{h^2}) + Q_i(\frac{u}{h}-\frac{2\mathcal{D}}{h^2}) + Q_i(-\frac{\mathcal{D}}{h^2}) = 0\]
\[ -Q_{i-1} (1+{Pe}_n) ... \]

Попробуем предсказать поведеение вышеописанных схем

\[ Q_{i-1} = Q_i - h \frac{\partial q}{\partial x}|_i + \frac{h^2}{2!} \frac{\partial^2 q }{\partial x} - \frac{h^3}{3!} \frac{\partial^3 Q }{\partial x^3}+... \]
\[ Q_{i+1} = Q_i + h \frac{\partial q}{\partial x}|_i + \frac{h^2}{2!} \frac{\partial^2 q }{\partial x} + \frac{h^3}{3!} \frac{\partial^3 Q }{\partial x^3}+... \]

Первое слагаемое:
\[ \frac{U }{h }{(\frac{\partial Q }{\partial x})}_i - \frac{h^2}{2} \frac{\partial^2 Q }{\partial x^2} + \frac{h^3}{3!} \frac{\partial^3 Q }{\partial x^3} ... \]

\[ U \frac{\partial Q}{\partial x}|_i-\frac{1}{2}Uh \frac{\partial^2Q }{\partial x^2} - \mathcal{D} \frac{\partial^2}{\partial x^2}|_i/... = 0 \]
\[ U \frac{\partial Q}{\partial x}|_i - \mathcal{D}(1+\frac{1}{2} {Pe}_n)\frac{\partial^2 Q }{\partial x^2}|_i \]

ДЛя того чтобы коэф диффузии был похож на то что было в исходном уравнении $ \frac{{Pe}_n}{2} \textrm{ << } 1 $

\section{Схема А.А.Самарского}

\[ U^+Q_{\overline{x}} + U^- Q_x - \frac{\mathcal{D }}{1 + \frac{{Pe}_n }{2}} Q_{\overline{x}x} = 0 \]

\[ \overline{D} = \frac{\mathcal{D }}{1 + \frac{{Pe}_n}{2}} = \mathcal{D(1- \frac{|{Pe}_n| /2}{1+\frac{|{Pe}_n| }{2}})} \]

\[ u^+ Q_{\overline{x}} - u^-Q_x = u Q_{x0} - \frac{|U|_h}{2}Q_{\overline{x} x}\]

\[ U Q_{x0} - \mathcal{D} [\frac{1}{1+\frac{|Pe|}{2}}+\frac{{Pe}_n }{2}]Q_{\overline{x}x} = 0 \]
- Коэфф диффузии 
\[ \mathcal{D} [\frac{1}{1+\frac{|Pe|}{2}}+\frac{{Pe}_n }{2}] = \mathcal{D}^* \]

\[ {Pe}^*_n = \frac{Uh }{\mathcal{D }} \textrm{ - эффективное число Пикле } \]

\[ f = {Pe}^*({Pe}) \]
\[ \mathcal{D}^* = \mathcal{D} \frac{1+\frac{|{Pe}_n|}{2}+\frac{|{Pe}_n|^2}{4}}{1+\frac{|{Pe}_n|}{2}} \]
\[ {Pe}^* = \frac{Uh }{\mathcal{D}} \frac{1+\frac{|{Pe}_n }{2}}{1+\frac{|{Pe}_n|}{2}+\frac{|{Pe}_n|^2}{4}} \]

\[ {Pe}_n \rightarrow 0; {Pe}^*_n \rightarrow {Pe}_n  \]

\[ {Pe}_n \rightarrow \inf; {Pe}^*_n \rightarrow 2 \]

\[ |{Pe}*n| = \frac{{Pe}^2_n (\frac{1}{2} + \frac{1}{|{Pe}_n|})}{{Pe}^2_n (\frac{1}{4}+\frac{1}{2|{Pe}_n|}+\frac{1}{|{Pe}_n|^2})} \]

\section{Экспоненциальная схема}

\[ u \frac{dQ }{dx} - \mathcal{D} \frac{d^2Q }{dx^2} = 0 \]
\[ Q(x) = Q_i + {(Q_{i+1} - Q_i)}\frac{{exp({Pe}_{n, i+ \frac{1}{2}}\frac{x-x_i }{h_{i+\frac{1}{2}}})}}{exp({Pe}_{n, i+\frac{1}{2}})-1} \]

\[ W^{tot} = u_{i+\frac{1}{2}} Q_{i+\frac{1}{2}} - \mathcal{D} \frac{dQ}{dx}|_{i+\frac{1}{2}} = \]
\[ = u_{i+\frac{1}{2}} Q_i + u_{i+\frac{1}{2}}(Q_{i+\frac{1}{2}} Q_i) \frac{exp(\frac{{Pe}_{i+\frac{1}{2}}}{2}-1)}{exp({Pe_{n, i+\frac{1}{2}}}-1)} - \]
\[ - u_{i+\frac{1}{2}}(Q_{i+1}-Q_i) \frac{exp(\frac{{Pe}_{i+\frac{1}{2}}}{2})}{exp({Pe_{n, i+\frac{1}{2}}})-1} \]

\[ \frac{W_{i+\frac{1}{2}} - W_{i-\frac{1}{2}}}{h_i} = u Q_{\overline{x}} + u h \frac{e^P \frac{{Pe}_N}{2}-1}{e^{{Pe}_N}-1} Q_{\overline{x}x} - u h \frac{e^P \frac{{Pe}_n}{2}}{e^{{Pe}_n}-1} Q_{\overline{x}x} = 0 \]

\[ u Q_{\overline{x}} = u Q_{x0} - \mathcal{D} \frac{Pe}{2} Q_{\overline{x}x} \]

\[ u Q_{x0} - \mathcal{D}^2 Q_{\overline{x}x} = 0 \]
\[ \mathcal{D}^* = \mathcal{D} (\frac{{Pe}_N}{2} \frac{e^{Pe} + 1}{e^{Pe}-1}) = \mathcal{D} \frac{{Pe}_n }{2} coth\frac{{Pe}}{2} \]

\[ {Pe}_n = \frac{Uh }{\mathcal{D }} = 2 th \frac{{Pn}_n}{2} \]

\section{Схема Сполдинга}

\[ {Pe}^*_{h, i+\frac{1}{2}} = Pe_{h, i+\frac{1}{2}}, {Pe}_h \leq 2 \]
\[ = 2, {Pe}_h > 2 \]

\[ \Theta_{i+\frac{1}{2}} = \Theta_{i+\frac{1}{2}} Q_i + (1 - \Theta_{i+\frac{1}{2}}Q_{i+1})\]

\[ \Theta_{i+\frac{1}{2}} = \frac{1}{{Pe_n, i+\frac{1}{2}}}[{Pe_{n, \frac{1}{2}}}-1+max(- {Pe}_{n, i+\frac{1}{2}}, 1-\frac{{Pe}_{n, i+\frac{1}{2}}}{2}, 0)] \]


\begin{enumerate}
	\item 
	\[ |Pe_{n, i+\frac{1}{2}}| \leq 2 \]
	\[ \Theta_{i+\frac{1}{2}} = \frac{1}{2} \Rightarrow \] буквально схема с центральными разностями.
	\item
	\[ |Pe_{n, i+\frac{1}{2}}| > 2 \]

	\begin{enumerate}
	\item 
	\[ {Pe}_n > 2 \]
	\[ \Theta_{i+\frac{1}{2}} = \frac{1}{{Pe}_{n, i+\frac{1}{2}}} [{Pe}_{n+\frac{1}{2}} -1 + 0] = 1 - \frac{1}{Pe} \]

	\item
	\[ Pe_{n} < -2 \]
	\[ \Theta_{i+\frac{1}{2}} = \frac{1}{{Pe}_{n, i+\frac{1}{2}}} [{Pe}_{n+\frac{1}{2}} -1 - {Pe}_n] = - \frac{1}{{Pe}_{n}} \]

	\end{enumerate}
	Или оба варианта в одной формуле

	\[ \frac{1}{2}(1+ \frac{|{Pe}_n|}{{Pe}_n}) - \frac{1}{{Pe}_n} \]

\end{enumerate}

Пподставляем $ \Theta $
\[ Q_{i+\frac{1}{2}} = [ \frac{1}{2} (1 + \frac{|{Pe}_n|}{{Pe}_n}) = \frac{1}{{Pe}_n}] Q_i + [\frac{1}{2} (1 - \frac{|{Pe}_n|}{{Pe}_n} + \frac{1}{{Pe}_n })]Q_{i+\frac{1}{2}} \]

\[ W_{i+\frac{1}{2}} = u Q_{i+\frac{1}{2}} = U^+_{i+\frac{1}{2}} Q_i + U^-_{i+\frac{1}{2}} Q_i+1 + D Q_x \]

\[ \frac{d }{dx }u Q - D \frac{d^2Q }{dx^2} = 0 \]
\[ W_{i+\frac{1}{2}} - W_{i-\frac{1}{2}} = U^+ Q_{\overline{x}} + \overline{u} Q_x + D Q_{\overline{x}x} \]

Таких схем множество, но суть такова, что до определенного момента рассматривается направленные разности, а потом переходят на константу

\section{Схема Патанкара}

Надо разобрать к экзамену! Есть в презентации. 

\[ \Theta_{i+\frac{1}{2}} = \frac{1}{{Pe}_{n, i \frac{1}{2}}}[{Pe}_{n, i+\frac{1}{2}} - 1 + max(0, {Pe}_n) + max(0, (1 - 0.1 |{Pe}_n|)^5)] \]

Теперь нужно определить в полуцелых точках.

\section{Разностная схема на расширенных шаблонах}

\[ Q_{i+\frac{1}{2}} = \frac{1}{2} (Q_{i+1} + Q_{i}) - \eta ( Q_{i+1} - 2 Q_i + Q_{i-1}) \]

\begin{enumerate}
	\item \[ u \frac{d }{dx} = u Q_{x^0} -\frac{U h }{2}( Q_x \overline{x} - Q_{x x})    \frac{1}{2} h^2 Q_{\overline{x} \overline{x} } \]

	\item
	\[ \eta = \frac{1}{2} \textrm{ - схема с направленными разностями } \]

	\item
	\[ \eta = \frac{1}{4} -\textrm{  схема Фрома } \]
	\item
	\[ \eta = \frac{1}{6} -\textrm{  схема с искуственной дисперсией }\]
	\item
	\[ \eta = \frac{1}{8} -\textrm{  схема QUICK  }\]

\end{enumerate}

\section{Нестационарные задачи}

\[ \frac{\partial Q}{\partial t} + \frac{\partial }{\partial x }(u Q) = \mathcal{D} \frac{\partial^2 Q }{\partial x^2 } \]

\[ u = const, {Pe} = \frac{U L }{\mathcal{D}} >> 1 \]

Сетка вводится аналогично стационарной задаче.

\[ t_k = \tau_k \]

\[ \int_{t_k }^{t_{k+1}}\int_{x_{i-\frac{1}{2}}}^{x_{i+\frac{1}{2}}} \frac{\partial Q }{\partial t} dx dt + \int_{t_k }^{t_{k+1}}\int_{x_{i-\frac{1}{2}}}^{x_{i+\frac{1}{2}}} \frac{\partial}{\partial x} (u Q) dx dt = \int_{t_k }^{t_{k+1}}\int_{x_{i-\frac{1}{2}}}^{x_{i+\frac{1}{2}}} \frac{\partial}{\partial x} (\mathcal{D} \frac{\partial Q }{\partial x }) dx dt \]

По слагаемым
\begin{enumerate}
	\item Выносим, так как не зависит от t, по x константа на отрезке интегрирования
	\[ h_i \cdot (Q^{k=1}_i - Q^k_i) \]
	\item \[ \int_{t_k }^{t_{k+1}}(W_{i+\frac{1}{2}} - W_{i-\frac{1}{2}} dt ) \]
	Воспользуемся квадратурной формулой
	\[ = \tau (W^{\sigma}_{i+\frac{1}{2}} + W^{\sigma}_{i-\frac{1}{2}}) \]
	\[ f^{(\sigma)} = \sigma f^{t_k + 1} + (1-\sigma) f^{t_k} = \sigma \hat{f} + (1-\sigma) f \]
	Возникает ровно та же проблема определения W в полуцелых точках
	\item
	\[ \int_{t_k }^{t_{k+1}} [ \mathcal{D} \frac{\partial Q }{\partial x} |_{i+\frac{1}{2}} - \mathcal{D} \frac{\partial Q }{\partial x} |_{i-\frac{1}{2}}] dt = \]
	\[ \tau [(\mathcal{D} \frac{\partial Q }{\partial x})_{i+\frac{1}{2}}^{(\sigma)} - (\mathcal{D} \frac{\partial Q }{\partial x})_{i-\frac{1}{2}}^{(\sigma)}] \]
\end{enumerate}

\section{Решение модельной задачи}

Отрезок от -L до L. Начальные данные: 1 при отрицательных значений, 0 при положительных.

\[ \frac{\partial Q }{\partial t } + u \frac{\partial Q }{\partial x} = \mathcal{D} \frac{\partial^2 Q }{\partial x^2} \]

\[ S = t, y = x - ut \]
\[ Q(x,t) = Q(y(x,t), S(t)) \]
\[ \frac{\partial Q }{\partial t } = \frac{\partial Q }{\partial S} \frac{ Q }{\partial t } + \frac{\partial Q }{\partial y}\frac{\partial y}{\partial t} = \frac{\partial Q }{d S}- \frac{\partial Q }{\partial y}\]
\[ \frac{\partial Q}{\partial y} - u \frac{\partial Q}{i\partial y}+ u \frac{\partial Q }{\partial y } \mathcal{D } \frac{\partial^Q }{\partial y^2} \]
\[ \frac{\partial Q }{\partial S } = \mathcal{D } \frac{\partial^2 Q}{\partial y^2} \Rightarrow\] интеграл Пуассона
\[ Q = \frac{1}{\sqrt{\pi}} [\frac{\sqrt{\pi}}{2} - \int_{0}^{\frac{y}{2 sqrt{\mathcal{D S}}}} e^{-z^2} dz] =  \]
\[ \frac{1}{\sqrt{\pi}} [\frac{\sqrt{\pi}}{2} - \int_{0}^{\frac{x - ut }{2 \sqrt{\mathcal{D }t }}} e^{-z^2} dz] \]
Подставить это дома и убедиться, что оно является решением уравнения.

\subsection{Направленные разности.}

**рисунок фронта**

Самое большое размытие при использовании чисто неявной схемы ($ \sigma = 1 $)

Меньше всего размывается явная схема ($ \sigma = 0 $).

То есть неявная схема ухудшает проблемы направленных разностей

\subsection{Центральные разности}

**рисунок фронта

Чисто неявная схема снова размывает сильнее.

($ \sigma = 0.5 $)

\section{Диффузия и дисперсия}

Дисперсия - скорость убывания волны зависит от амплитуды.

\[ \frac{\partial Q }{\partial t} + u \frac{\partial Q }{\partial x } = \mathcal{D } \frac{\partial^2 Q }{\partial x^2 } + \beta \frac{\partial^3 Q }{\partial x^3 } + \gamma \frac{\partial^4 Q }{\partial x^4} \]

\[Q(t, x) = e^{-P(m)t} e^{i m (x - q(m) t)} \]
m - волновое число,
P(m) - скорость убывания амплитуды,
q(m) - скорость распространения волны,
$ i^2 = -1 $.

\[ Q = e^{-P t } e^{im(x-qt)} \]

\[ \frac{\partial Q }{\partial t} + u \frac{\partial Q }{\partial x } = \mathcal{D } \frac{\partial^2 Q }{\partial x^2 } + \beta \frac{\partial^3 Q }{\partial x^3 } + \gamma \frac{\partial^4 Q }{\partial x^4} \]

Подставим Q и посмотрим каими должны быть P и m, для того чтобы оно удовлетворяло уравнениям.

\[ -P - i m q + U i m = - \mathcal{D} m^2 - i m^3 p + \gamma m^4 \]

\[ P = \mathcal{D }m^2 - \gamma m^4 \] - скорость убывания
\[ q = u + M^2 \beta \] - скорость волны

На амплитуду влияют четные производные ( но они чередуются уменьшает-увеличивает-уменьшает-...)

скорость движения от третьей производной (нечетныепроизводные) проверить с каким знаком.

\section{Дифференциальное представление разностных схем}

\[ L(Q) = 0 \textrm{ Дифференциальное уравнение} \]
\[ L^{\tau}_h (Q, \tau, h) = 0 \textrm{ - Разностная схема } \]
\[ L_H^{\tau} (Q) = L(Q) + \epsilon(Q, \tau, h) = 0 \textrm{ - Погрешность аппроксимации } \]

Не будем убирать погрешность и будем его рассматривать как бесконечный ряд.

\[ L_h^{\tau} (Q) = L(Q) + \sum_{i=0}^{\infty} \alpha_i \frac{\partial Q }{\partial x^i } = 0 \]

Процедура по созданию подобного представления (в общих чертах):

\subsection{Дифференциально приближение для схемы с направленными разностями.}

\[ Q_t + u Q^{(\sigma)}_{\overline{x}} - \mathcal{D} Q^{(\sigma)}_{\overline{x}} x = 0 \]

\[ Q^{(\sigma)} = Q^{0.5} + (\sigma - 0.5) \tau Q_t = \frac{\hat{Q} + Q}{2} + (\sigma - 0.5) (\hat{Q} Q)\]

\[ Q_t + u Q^{(0.5)}_{\overline{x}} - \mathcal{D} Q^{(0.5)}_{\overline{x} x} + u(\sigma - 0.5) \tau Q_{t \overline{x }} - \mathcal{D }(\sigma-0.5) Q_{t \overline{x} x} \]

Разложим в точке $ (t_k + \frac{\tau }{2}, x_i Q (t_k + \frac{\tau }{2}) Q^(0.5)) $

\[ Q_t = \frac{\partial \overline{Q }}{\partial t } + \frac{t^2}{2} \frac{\partial^3 \overline{Q  }}{\partial t^3 } + \dots \]

\[ Q^{(0.5)} = \overline{Q} +\frac{\tau^2}{8} \frac{\partial^2 Q}{\partial x ^2} + \dots \]

и тд.

Как избавиться от высоких производных и смешанных производных по t и x. Метод дифференциального представления.

\section{Устойчивость разностных схем. Метод Неймана (Метод гармоник.)}

\[ Q(t, x) = e^{-p(m)t} e^{Im(x-qt)} \textrm{ - Бегущая волна } = a(t) e^{I m x}\]

\[ a(t) = e^{-t [ p(m) + I q m]} \textrm{ Комплексная амплитуда }\]

\[ p(m) = \mathcal{D} m^2, q=u \textrm{ Для нашего конкретного уравнения } \]

\[ Q(t + \tau, x) = e^{-t + \tau}[\mathcal{D}m^2 + (I m q)] e^{Imx} = G Q (t,x) \]

G - Множитель перехода

\[ G = e^{- \tau (\mathcal{D} m^2 = I m u)}  \]

\subsection{Явная схема с направленными разностями}

\[ Q^{k+1}_i = Q^k_i - c(Q^k_i - Q^k_{i-1}) + S (Q^k_{i+1} - 2 Q_i^k + Q^k_{i-1}) \]

\[ c = \frac{u \tau}{h} \textrm{ - число Куррента }, S = \frac{\mathcal{D } \tau }{h^2} \textrm{ - Тепловое число Куррента }\]

\[ Q^k_i = Q (t_k, x_i) = e^{-pt_k} e^{Im [x_i - q t_k]} = \]
\[ = e^{\tau k [ p - I m q]} e^{I m h i} =  \]
\[ = a_k e^{I \Theta i}\]

\[ \tau k = t_k, ih = x_i, \Theta = m h \]

Надо выяснить при каких P и Q при подстановке бегущей волны данное уравнение является решением нашего уравнения.

\[ a_{k+1} e^{I \Theta i} = a_k e^{I \Theta i} - c a_k (e^{I \Theta (i-1)}) + S a_k (e^{I \Theta (i-1)} - 2e^{I \Theta i }) + e^{I \Theta(i+1)} \]

\[ e^{e m i}; a_{k+1} = a_k G \]

\[ G = 1 - c(1-e^{-I \Theta}) + S (e^{I \Theta} - 2 + e^{-I \Theta}) \label{eq:f1} \]

\[ c(1-e^{-I \Theta}) = 1 - cos \Theta + i sin \Theta \]

\[ (e^{I \Theta} - 2 + e^{-I \Theta}) = 2 cos \Theta - 2 \]

Напоминание:
\[ e^{i \phi} = cos \phi + i sin \phi \]
\[ e^{-i \phi} = cos \phi - i sin \phi \]
-----------------------------------------

\[ \ref{eq:f1} = 1 - (c+2S)(1-cos \Theta) + I c sin \Theta \textrm{ Множитель перехода со слоя на слой } \]

Схема устойчива тогда, когда $ |G|>1 $
\[ \Theta = m h \]
\[ \lambda_{min} = 2h = \frac{2 \pi}{m} \textrm{ - минимальная длинна волны } \]

\[ \Rightarrow mh = \pi \textrm{ Короткие волны }\]

\[ mh \rightarrow 0 \textrm{ - Длинные волны } \]

\[ 1 - (c + 2S ) (1- cos \Theta ) - I c sin \Theta \]

\[ {|G|}^2 = 1 - 4 (c+2S) sim^2 \frac{\Theta}{2} + {(c+2S)}^2 \cdot 4 {sin}^4 \frac{\Theta   }{2} + 4 c^2 {sin}^2 \frac{\Theta   }{2} {cos}^2 \frac{\Theta   }{2} < 1\]

\[ -(c+2S) + {(c+2S)}^2 {sin}^2 \frac{\Theta   }{2} + c^2 {cos^2} \frac{\Theta}{2} < 0 \]

\[ -(c+2S) ({sin}^2 \frac{\Theta}{2} + {cos}^2 \frac{\Theta}{2}) + {(c+2S)}^2 {sin}^2 \frac{\Theta   }{2} + c^2 {cos^2} \frac{\Theta}{2} < 0 \]

\[ (c+ 2S) [ (c+ 2S) - 1] {sin}^2 \frac{\Theta}{2} + [c^2- (c+2S)] {cos}^2 \frac{\Theta    }{2} < 0; \forall \Theta \in [0, \pi] \]

\[ c+2S - 1 < 0; c^2 - (c + 2S) < 0 \textrm{ - Условия устойчивости }\]

Метод Неймана дает необходимое условие устойчивости, но вообще говоря не достаточное.

\section{Изменение амплитуды волны.}

\[ {|G|}^2 = 1+ 4( c + 2S) {sin}^2 \frac{\Theta    }{2} = 4 (c+ 2S )^2 {sin}^2 \frac{\Theta    }{2} {cos}^2 \frac{\Theta   }{2} \]

Для волны $ \Theta = mh \rightarrow 0 $

\[ {|G|}^2 = 1 - 4 (c + 2S) \frac{{(mh)}^2}{4} + 4 (c +2S)^2 \frac{{(mh)}^4}{2}^4 + 4c^2 \frac{{(mh)}^2}{4} = 1 + {(mh)}^2 [ c^2 - (c+ 2S)] =  \]
\[ = 1 - (mh)^2 [ c+2S - c^2] \]
\[ |G| = 1 - {(mh)}^2 (S + \frac{1}{2} c(1-c)) \label{eq:1*} \]

\[ |G| = 1 - (m^2 h^2) (\frac{\mathcal{D} \tau}{h^2} + \frac{1}{2} c (1-c)) =  \]
\[ = 1 - m^2 \tau \mathcal{D} - \frac{1}{2} m^2 h^2 c (1-c) \]

\[ 1 - m^2 \tau \mathcal{D}\textrm{ То как убывает волна в исх уравнении } \]
\[ \frac{1}{2} m^2 h^2 c (1-c) \textrm{ Добавка разностной схемы } \]

\[ |G| = e^{-\tau \mathcal{D} m^2} = 1 - \tau \mathcal{D} m^2 \]

\subsection{Коротко про устойчивость схемы с центральными разностями}

\[ Q_t + u Q_{x^0  \mathcal{D }Q_{\overline{x }x}} \]
\[ Q^{k+1}_i = Q^k_i - \frac{1}{2} c (Q^k_{i_k} - Q^k_{i-1}) + S(Q_{i-1}^k - 2 Q^k_i + Q_{i+1}^k) \]

\[ G = 1 - \frac{c }{2} (e^{I \Theta} - e^{-I \Theta}) + S(e^{-I\Theta} - 2 + e^{I\Theta}) = \]
\[ = 1- 4 S {sin}^2 \frac{\Theta}{2} -2 c I sin \frac{\Theta}{2} cos\frac{\Theta}{2} \]

\[ {|G|}^2 = 1 - 8 S {sin}^2 \frac{\Theta}{2} + 16 S^2 {sin}^4 \frac{\Theta}{2} + 4c^2 {sin}^2 \frac{\Theta}{2} {cos}^2 \frac{\Theta}{2} < 1\]

\[ -2S+4S^2 {sin}^2 \frac{\Theta}{2} + c^2 {cos}^2 \frac{\Theta}{2} < 0 \]

\[ {sin}^2 \frac{\Theta}{2} [4S^2 - 2S] + {cos}^2 \frac{\Theta}{2}[c^2 - 2S] < 0 \]
\[ 2S^2 - S < 0; c^2-2S < 0 \textrm{ - Условие устойчивости } \]


Рассмотрим для длинных волн
\[ {|G|}^2 = 1 - 8 S {sin}^2 \frac{\Theta}{2} + 16 S^2 {sin}^4 \frac{\Theta}{2} + 4c^2 {sin}^2 \frac{\Theta}{2} {cos}^2 \frac{\Theta}{2} < 1\]

\[ \approx = 1 - 8 S \frac{{(mh)^2}}{4}+ 4 c^2\frac{{(mh)}^2}{4} =  \]
\[ = 1 - {(mh)}^2(2S-c^2) \]

\[ |G| \approx 1 - {(mh)}^2 S + {(mh)}^2 \frac{c^2}{2} =  \]
\[ 1 - \frac{m^2 h^2 \mathcal{D} \tau}{h^2} + m^2 h^2 \frac{c^2}{2} \]

Длинные волны убывают медленнее чем в дифференциальном случае. (В прошом примере было наоборот)

дз решить задачи из таблицы (хотя бы некоторые), позже возможно следует ей прислать.

\section{Уравнение движения вязкой несжимаемой жидкости}

\[ \rho [\frac{\partial V }{\partial t } + (v \cdot \triangledown)v] = - \triangledown p + \chi \triangle V = g \rho \]
\[ div V = 0;  (\triangledown \cdot v) = 0\]
\[ \rho = \rho(p, t) = \rho_0 (1 + \beta T) \]

V - вектор скорости
p - Давление
$ \chi $ - коэффициент вязкости

\[  \frac{\mathcal{D} \rho}{\mathcal{D} x} =  \textrm{ Полная производная }\]
\[ = \frac{\partial \rho}{\partial t} + \mathcal{V}_x \frac{\partial \rho}{\partial x} + \mathcal{V}_y \frac{\partial \rho}{\partial y} + \mathcal{V}_z \frac{\partial \rho}{\partial z} \]

\[  \frac{\mathcal{D} \rho}{\mathcal{D} x} + \rho div V = 0 \]
\[ = \frac{\partial \rho}{\partial t} + \mathcal{V}_x \frac{\partial \rho}{\partial x} + \mathcal{V}_y \frac{\partial \rho}{\partial y} + \mathcal{V}_z \frac{\partial \rho}{\partial z} + \rho div V = 0\]

Жидкость несжимаема тоже самое, что полная производная равна 0, это одновременно означает что дивергенция равна 0. 

\[ \frac{\partial \mathcal{V}_x}{\partial t} + \mathcal{V}_x \frac{\partial \mathcal{V}_x}{\partial x}+ \mathcal{V}_y \frac{\partial \mathcal{V}_y}{\partial y}+ \mathcal{V}_z \frac{\partial \mathcal{V}_z}{\partial z} = \]
\[ = \frac{\mathcal{V}_x}{\partial t} + \frac{\partial}{\partial x} (\mathcal{V}^2_x) - \mathcal{V}\frac{\partial\mathcal{V}_x}{} ... \]

\end{document}

\documentclass[14pt]{extarticle}

\usepackage[T1,T2A]{fontenc}
\usepackage[utf8]{inputenc}
\usepackage[english,russian]{babel}
\usepackage{multirow}

\usepackage{caption}

\usepackage{amsmath}
\usepackage[hidelinks]{hyperref}


\usepackage{graphicx}%Вставка картинок
\graphicspath{{noiseimages/}}

\usepackage{float}%"Плавающие" картинки
\usepackage{wrapfig}%Обтекание фигур (таблиц, картинок и прочего)

\begin{document}

\section{Первая лекция (вторая)}

Преподлватель: Андрей Горобец.

Сферы применения: можно рассчитывать свойства предметов до того как тот был создан.

На лекции показывали примеры. 

g++ - fopenmp функция что это?

рассмотрели unroll.

Классификация параллелизма Флинна. 

Флопсы количество произведенных инструкций деленные на время. TPP: пиковая производительность устройства. BW: пропускная способность памяти. AI: флоп на байт вычислительная интенсивность (как сильно нагружен процессор к памяти) TPB: теоретически достижимая производительность. Rmax: фактически достигаемая производительность.

CPI - количество тактов на операцию.

IPC - количество инструкций выполняемых за такт.

FLOP per cucle - количество операций с плавающей тоской.

ILP - конвейерный параллелизм и суперскалярность.

SIMD расширение - параллелизм ядра процессора.

FMA операции (AVX-512), векторные инструкции. + в может производить за один таки последовательно сложение люс умножение. 16 операций за такт.

Ключевой момент - доступ к памяти.

Устройство конвеера. Разные типы конфликтов, которые прерывают конвеер.

Как определяется область видимости переменной в си.

OoOE (out-of-order) процессор растаскивает инструкции.

vLIW - very long instruction word. Компилятор сам распределяет инструкции.

SMT - simultaneous multithreading одновременная многопоточность. Команды из разных потоков выполняются одновременно.

Иерархия памяти.

Поколения памяти - пропускная возможность увеличивается, но задержка оо- швах.

Кэш.
Выравнивание из-за кэша.

\section{Лекция 2}

FUA - fused multiplier addition.

LLC - распределенный кеш. 

\subsection{ Основы стандарта OpenMP (Старый). }

макрос для секций

PU - processing unit. С общей памятью.

Характеристики памяти: Пропускная способность и время доступа (латентность).

Неоднородная общая память - NUMA (в разные места разное время доступа и пропускная способность).

Презентацию добавить.

Неоднородность работы.

Определение NUMA раздела с помощью масок.

\section{Лекция 3}

Имя лектора: Михаил Владимирович Якобовский.

Цель курса - дать понимание перед тем как писать программу имеет ли смысл ее распаралелливать.

Большая часть суперкомпьютеров $ >100 000 $

Четыре области применения многопроцессорных систем

\begin{enumerate}
	\item Сокращение времени решения высокопроизводительных задач.
	\item Сокращение времени обработки больших данных.
	\item Решение задач в реальном веремени.
	\item Создание систем высокой надежности.
\end{enumerate}

Оптически матричный умножитель - хорошая производительность, но низкая точность (специфический класс систем).

На нижнем уровне данные всегда передаются синхронным методом.

В итоге все равно будет передача точка-точка, даже в групповых передачах данных.

За микросекунду процессор выполняет примерно 1000 операций (порядок $ 10^3 $). 

Ускорение это $ S_p = \frac{T_1}{T_p}  $

Эффективность $ E_p = \frac{S_p }{p} $

Предел масштабируемости (не универсальное определение) - минимальное число процессоров, при котором достигается максимаольное ускорение.

Каскадная схема сложения 8-ми чисел в 1000 раз медленнее на многопроцессорной системе (задача очень маленькая, из-за этого идет замедление, слишком много взаимодействий между процессорами).

$ T_1^* $ - Самый быстрый последовательный алгоритм. Относительно него не должно быть сверхлинейного ускорения.

\end{document}
\documentclass[12pt, a4paper]{article}

\usepackage[T1,T2A]{fontenc}
\usepackage[utf8]{inputenc}
\usepackage[english,russian]{babel}

\usepackage{pdfpages}
\usepackage{multirow}

\usepackage{caption}

\usepackage[fleqn]{amsmath}
\usepackage{amssymb}
\usepackage{cancel}

\usepackage[hidelinks]{hyperref}

\usepackage{graphicx}%Вставка картинок

\usepackage{float}%"Плавающие" картинки
\usepackage{wrapfig}%Обтекание фигур (таблиц, картинок и прочего)


\setlength{\emergencystretch}{10pt}

\usepackage{indentfirst} %Отступ после секции

\usepackage[left=2cm,right=1cm,
top=2cm,bottom=2cm,bindingoffset=0cm]{geometry}

\usepackage{setspace}

\usepackage{tocloft}
\setlength\cftsecnumwidth{0em}

\usepackage{titlesec}% http://ctan.org/pkg/titlesec
\titleformat{\section}%
[hang]% <shape>
{\normalfont\bfseries\Large}% <format>
{}% <label>
{0pt}% <sep>
{}% <before code>
\renewcommand{\thesection}{}% Remove section references...
\renewcommand{\thesubsection}{\arabic{section}.\arabic{subsection}}%... from subsections

\begin{document}

\thispagestyle{empty}

\begin{center}
	\ \vspace{-1cm}
	
	{Московский государственный университет имени М. В. Ломоносова}\\
	Факультет вычислительной математики и кибернетики\\
	Кафедра вычислительных методов
	
	\vspace{8cm}
	\begin{spacing}{2.5}
		{\huge \bfseries ВАРИАЦИОННО-ПРОЕКЦИОННЫЕ МЕТОДЫ В ЗАДАЧАХ МАТЕМАТИЧЕСКОЙ ФИЗИКИ}
	\end{spacing}
	

\end{center}

\vfill

\begin{center}
	Москва, 2024
\end{center}

\enlargethispage{2\baselineskip}

\newpage

\tableofcontents

\newpage

\section{Лекция 1}

\subsection{Исторический экскурс}

Лекции с 9:30 два часа.

Вариационная постановка для задача мат физики.
задача - условие на границе + начальное. (Экстремум функционала энергии, поэтому энергетические пространства).

Соболев - прямые методы решения задач. позволяют найти решение с помощью СЛАУ. Наиболее известен метод Рица.

\subsection{Метод Дирихле}

Дана область $ \omega \in \mathbb{R}^2$.

\[ M = {u:u_{0}(x,y), (x,y) \in \partial \Omega} \]
\[ \iint_{\Omega}[{(\frac{\partial u}{\partial x})}^2+{(\frac{\partial u}{\partial y})}^2] dx dy \rightarrow min \]
Интеграл Дирихле $ \Rightarrow \overline{u} $ - гармонический в $ \Omega $

\subsection{Контрпример Вейерштрасса. }

\[ M={y;y(x) \in c'[-1;1]}, y(-1)=-1, y(1)=1 \]
\[ J(y) = int^1_{-1} x^2{(y')}^2 dx, J(y) \geq 0 \]
\[ y_\varepsilon(x) = \frac{arctg(\frac{x}{\varepsilon})}{arctg(\frac{1}{\varepsilon})} \]
\[ {y'}_{\varepsilon}(x) = \frac{1}{arctg(\frac{1}{\varepsilon})} \cdot \frac{1}{1+ \frac{x^2}{\varepsilon^2}} \cdot \frac{1}{\varepsilon} = \]
\[ = \frac{\varepsilon} {arctg (\frac{1}{\varepsilon})} \cdot \frac{1}{\varepsilon^2+x^2} \]
\[ J(y_{\varepsilon}) = \int_{-1}^1 \frac{x^2 \varepsilon^2}{arctg^2(\frac{1}{\varepsilon})} \cdot \frac{1}{{(x^2+y^2)}^2} dx = frac{2 \varepsilon}{arctg(\frac{1}{\varepsilon})} = \frac{0}{\frac{\pi}{2}}\]
\[ J(\overline{y}) = \int_{-1}^{1} x^2 y^2 dx = 0 \Rightarrow y'=0 \]
Противоречие: $ y(-1)=-1, y(1)=1 $

\subsection{Контрпример Адамара}

\[ u(x,y) = \sum_{n=1}^{\inf}\frac{\rho^2}{2^n} cos(2^n \Theta), x=\rho cos \Theta, y=\rho sin \Theta \]
\[ \rho \leq 1 \]
Непрерывны и гармоничны.
Интеграл Дирихле в круге $ \rho \leq r \leq 1 $
\[ \pi sum_{n=1}^{\inf} {r^2}^{2n+1} {\rightarrow}_{r \rightarrow 1} \inf \]

\subsection{Метод Ритца}

\[ J(\omega) = int_{a}^{b} f(x,\omega, \omega', ... , \omega^{(k)}) dx \rightarrow inf \]
\[ \omega \in M \textrm{ класс допустимых функций } \]
\[ \psi_0, \psi_1, ... \psi_n, ... (\textrm{ координатные функции }) \]
Св-ва:
\[ 1) \forall a_1 ... a_n \in \mathbb{R}, \forall_n \]
\[ \omega_n = \omega_0+ \sum_{i=1}^{n} a_i \omega_i \in M \]
\[ 2) \forall \omega \in M\textrm{и} \forall varepsilon > 0 \]
*Уравнение полноты*
\[ H(\omega n) = F(a_1, ... , a_n) \rightarrow inf \]
\[ || \omega - \psi_0 - \sum_{i=1}{n} a_i \psi_i || < \varepsilon \]
Рассмотрим:
\[ J(\omega_n) = F(\alpha_1, ..., \alpha_n) \rightarrow inf \]
\[ \frac{\partial J}{\partial a_1}(\omega_n)=0, ... \frac{\partial J}{\partial a_n}(\omega_n) = 0 -\textrm{ альтернативная система уранений } \]
\[ \Rightarrow a_1, ... , a_n - \textrm{ решение } \]

Насколько хорошо приближает метод искомое решение?
На примере задачи об упругой поластине.

\[ \Omega_{\subset \mathbb{R}^2} - \textrm{ обл }, S = \partial \Omega \]
\[ \textrm{ изгиб } \omega(x,y) \textrm{ удовлетворяет ур-ю Сори Жульен }\]
\[ \Delta^2 \omega = \frac{\partial^4}{\partial x^2} + 2 \frac{\partial^4 \omega}{{\partial x}^2 {\partial y}^2} + \frac{\partial^4 \omega}{{\partial y}^4} = \frac{q(x y)}{\mathcal{D}}; (x,y) \in \Omega \]
\[ \mathcal{D} - \textrm{ жесткость пластины при упругом изгибе } \]
\[ q(x,y),\textrm{- Интенсивность давления } \]
\[ \omega (x,y) = 0 \]
\[ \frac{\partial \omega (x,y)}{\partial\mathcal{D} } = 0 \leftarrow \textrm{ Производная по нормали к S } \]
\[ J(\omega) = \iint_{Omega}( \frac{1}{2} (\Delta \omega)^2 - f(\omega) d \Omega \rightarrow inf) \]
\[ f = \frac{q(x,y)}{\mathcal{D}} \in C'(\overline{\Omega}) \]
\[ \omega = \omega_1 + \omega_2 \]
Рассмотрим без доказательства ограниченности снизу.
\[ \omega_1 = \frac{1}{8\pi} \iint_{\Omega} r^2 ln r f(\xi, \eta) d\xi \eta \]
\[ (x,y) (\xi, \eta) - \textrm{ точки из } \Omega r-\textrm{ расстояние между } (x,y)\textrm{ и } (\xi, \eta) \]
\[ J(\omega) = J_0 + \frac{1}{2} \iint_{\Omega} (\Delta \omega_2)^2 dx dy \]
\[ j(\omega) \geq J_0 \Rightarrow \exists inf J(\omega)\]
Введем $ \psi_1(x,y), ... , \psi_n(x,y) $ - координатные ф-ции
\[ 1) \psi_n(x,y), \frac{\partial^{k+l} \psi_n}{\partial x^k \partial x^l} \in C( \overline{ \Omega}), k \leq \varepsilon , l \leq \varepsilon \]
\[ 2) \psi_n(x,y) \textrm{ удовлетворяет краевым условиям} \]
\[ 3) \forall \textrm{ ф-ии } \zeta(x,y): \]
\[ \textrm{ a) удовлетворяет пункту 1 } \]
\[ \textrm{ б) } \zeta(x,y) \equiv 0 (x,y) \in \Omega \rho \]
\[ \Rightarrow \exists m \in \mathbb{N} , \alpha_1, ... \alpha_m \in \mathbb{R}: \]
\[ |\zeta(x,y) - \sum_{i=1}^{m} \alpha_i \psi_i (x_i, y_i)|< \varepsilon \]
\[ | \frac{\partial^{k+1}\zeta}{\partial x^k \partial y^l} - \sum_{i=1}^{n} \alpha_i \frac{\partial^{k+l} \psi_i (x,y)}{\partial x^k \partial y^l}| < \varepsilon \]
Условие полноты $ k \leq \varepsilon, l \leq \varepsilon \Rightarrow $

приближенное решение :
\[ \omega_n = \alpha_1 \psi_1+ ... + \alpha_n \psi_n \rightarrow J(\omega) \]
\[ J_n = \iint_\Omega (\frac{1}{2} {(\Delta \omega_n)}^2 - f (\omega_n))dx dy \]
\[ \alpha_i \textrm{ выбираем : } J (\omega_n) \rightarrow J(\omega) \]
\[ \sum_{k=1}^n A_{ik} a_k = B_i, i= \overline{1,n} \label{*} \]
\[ A_{ik} = \iint_\Omega \Delta \psi_i \Delta \psi_k dx dy \Rightarrow \]
\[ \exists! \textrm{ решение } a_1, ... , a_n \textrm{ в } \omega_n = ... \textrm{ приближение решение } \]
\[ B_i = \iint_\Omega f\psi_i dx dy \]
\[ \rightarrow \textrm{ Сущ ед решения } a_1, ..., a_n \textrm{в} \omega_n = ... (\textrm{ приближенное решение }) \]
\[ \textrm{ Рассмотрим } \forall b_1, ... b_n \]
\[ \zeta_n = b_1 \psi_n + ... + b_n \xi_n \]
\[ \ref{*} b_i \textrm{ и } \sum_{i=1}^{n} \]
\[ \sum_{i=1}^n \sum_{k=1}{n} A_{ik} a_k b_i = \sum_{i=1}^{n} b_i B_i \]
\[ \sum_{i=x}^n \sum_{k=1}{n} \iint_\Omega b_i \Delta \psi_i \Delta \psi_k a_k dx dy - \sum_{i=1}^{n} \iint_\Omega f b_i \psi_i dx dy = 0 \]
\[ \sum_{i=1}^{n} [ \iint_\omega b_i \Delta \psi_i \sum_{k=1}^{n} a_k \Delta \xi_k dx dy - \iint_{\Omega f b_i \psi_i dx dy}] = 0 \]
...
\[ \int_Omega (\Delta \omega_n \sum_{i=1}{n} b_i \psi_i) - f(\sum_{i=1}{n} b_i \psi_i) dx dy = 0 \]
\[ \iint_Omega(\Delta \Omega_n \zeta_n - f \zeta) dx dy = 0 \]
\[ J_n^{(0)} = -\frac{1}{2} \iint_\Omega {(\Delta \omega_u)}^2 dx dy \textrm{ не возрастает y } \geq inf \]
\[ \forall \varepsilon > 0 \textrm{ по критерию Коши } \Rightarrow N(\varepsilon) \forall_n > N(\varepsilon) \]

\newpage

\section{Лекция 2}

\[ \varphi_1 (x,y), ..., \varphi_n(x,y) \text{ --- координатные функции}, \quad \quad  w_n = \alpha \varphi_1 + ... + \alpha_n \varphi_n \]
\[ J_n^{(0)} = -\frac{1}{2} \iint\limits_{\Omega} {(\Delta w_n)}^2 dxdy \]
\[ \forall \varepsilon >0 \ \exists N(\varepsilon) \quad \forall n \geq N(\varepsilon) \quad \forall m: 0 \leq J_n^{(0)} - J_{n+m}^{(0)} \leq \frac{1}{2} \varepsilon \]
\[ \frac{\omega_{m+n} - \omega_n}{\sqrt{\varepsilon}} = \varphi (x,y) \]
\[ \iint\limits_{\Omega}{(\Delta \varphi)}^2 dx dy < 1 \]

Обозначим $S = \partial \Omega$ --- границу области $\Omega$

\[ \varphi (x,y) = \frac{1}{2\pi}\int\limits_S \left( \varphi \frac{\partial (\ln r)}{\partial n} - \ln r \frac{\partial  \varphi}{\partial N} \right) dS + \frac{1}{2 \pi} \int\limits_{\Omega} \Delta \varphi \ln r \ d\xi d \eta \]
\[ {\left| \int\limits_{x} f(x) \overline{g}(x)dx \right|}^2 \leq \left( \int\limits_{x} {|f(x)|}^2 dx \right) \left( \int\limits_{x}{|g(x)|}^2 dx \right) \]
\[ |\varphi (x,y)| \leq \frac{1}{2\pi}{\left( \iint\limits_{\Omega} {(\Delta \varphi)}^2 d \xi d \eta \right) }^{1/2} {\left( \iint\limits_{\Omega}{\ln}^2 r \ d\xi d\eta \right)}^{1/2} \]
\[ |\varphi (x,y)| \leq C_1 \]
\[ |\omega_{n+m} - \omega_{n} | \leq C_1 \sqrt{\varepsilon} \]
\[ \omega_n \underset{\Omega}{\rightrightarrows}  w_n(x,y) \in C(\Omega) \]

\subsection{Метод Бубнова -- Галеркина}

$ w_n =  \alpha_1 \varphi_1 + ... + \alpha_n \varphi_n\ $

$ L w - \lambda M w = 0 $

$L, M$ --- дифференциальные операторы

$ \sum\limits_{i=1}^{n} (A_{ik} - \lambda B_{i_k}) a_k = 0, \quad k = \overline{1,n} $

\[
\begin{vmatrix} 
	A_{11} - \lambda B_{11} & \dots & A_{1n} - \lambda B_{1n} \\
	\vdots & \ddots & \vdots \\
	A_{n1} - \lambda B_{n1} & \dots & A_{nn} - \lambda B_{nn} 
\end{vmatrix}
= 0
\] \\

$ N (x,y) = L w_n - \lambda M w_n \enspace \text{--- \enspace невязка} $

$ N(x,y) \perp \varphi_i, \quad i = \overline{1,n} $

\subsection{Повторение}

\begin{enumerate}
	\item $ f(x) \overset{\text{п.в.}}{=} 0 \Rightarrow \int\limits_{\Omega} f(x) dx = 0 $
	\item $ \int\limits_{\Omega} f(x) dx = 0, \enspace f(x) >= 0 \enspace \Rightarrow \enspace f(x)\overset{\text{п.в.}}{=} 0 $ 
	\item $ |f(x)| < \varphi(x), \varphi \text{ --- суммируема по Лебегу} \Rightarrow f(x) \text{ --- суммируема по Лебегу} $
	\item $ \{\varphi_n(x)\} $ --- суммируемы с квадратами по Лебегу
	\\ \\
	$\lim\limits_{n,k \rightarrow \infty} \int\limits_{\Omega}{|\varphi_k (x) - \varphi_{n} (x)|}^2 dx = 0
	$
	\\ \\ \\
\end{enumerate}

Обозначим $V$ -- линейное пространство \\

$ (\varphi , \psi) - \text{скалярное произведение:} \quad (\boldsymbol{\cdot},\boldsymbol{\cdot}): V \times V \rightarrow \mathbb{C} $

\begin{enumerate}
	\item $  (\varphi, \psi) = \overline{(\psi, \varphi)} $
	\item $ (a_1 \varphi_1 + a_2 \varphi_2, \psi) = a_1 (\varphi_1, \psi) + a_2 (\varphi_2, \psi) $
	\item $ (\varphi , \varphi) \geq 0 $
	\item $ (\varphi, \varphi) = 0 \quad \Rightarrow \quad \varphi = \mathbf{0} $
\end{enumerate} 

$ \| \varphi \| = \sqrt{(\varphi, \varphi)} $ \\

\begin{itemize}
	\item Неравенство Коши-Буняковского
	
	$ | (\varphi, \psi) | \leq \| \varphi \| \| \psi \|$ 
	
	\item Неравенство треугольника
	
	$ \| \varphi + \psi \| \leq \| \varphi \| + \| \psi \| $
\end{itemize}
\[ L_2(\Omega): \quad (\varphi, \psi) = \int\limits_{\Omega}^{} \varphi(x) \overline{\psi(x)}dx \]
\[ L_2(\Omega , \sigma): \quad (\varphi, \psi) = \int\limits_{\Omega}^{} \varphi(x) \overline{\psi(x)}\sigma (x) dx \]
\[ L_2(\Omega^m): \quad (\varphi, \varphi) = \int\limits_{\Omega}^{} \sum_{k=1}^{m} \varphi_k(x) \overline{\varphi_k(x)}dx \] 

\newpage

\textbf{Критерий линейной зависимости системы функций}

\begin{gather*}
	\varphi_1, ..., \varphi_n \text{ линейно зависима (ЛЗ) в } H
\\
	\hspace{20mm} \Updownarrow
\\
	\begin{vmatrix} 
		(\varphi_1, \varphi_1) & \dots & (\varphi_1, \varphi_n) \\
		\vdots & \ddots & \vdots \\
		(\varphi_n, \varphi_1) & \dots & (\varphi_n, \varphi_n) 
	\end{vmatrix}
	= 0
\end{gather*} \\

\textbf{Опр.} $M$ --- плотно в $H$, если $ \forall p \in H$ и $\forall \varepsilon >0 \enspace \exists \varphi_n \in M: \| \varphi_n - \varphi \| < \varepsilon $. \\

$ C_0^{(\infty)} (\Omega) $ плотно в $ L_2(\Omega) $

$\quad \quad \quad \quad \quad \uparrow$

$ \forall \varepsilon > 0 : \quad \forall \varphi \in H \quad $ 
\begin{tabular}[t]{l}
	$ \exists \varphi_n^1 \in C_0^{(\infty)}(\Omega) : \quad \|\varphi_n^1 - \varphi\| < \varepsilon/2 $ \\ 
	$ \exists \varphi_n^2 \in C_0^{\infty} (\Omega) : \quad \|\varphi_n^2 - \varphi_n^1\| < \varepsilon/2 $ \\  
	$...$    
\end{tabular}

$ C_0^{(k)} (\Omega) $ плотно в $ L_2(\Omega) $ \\ \\

$ \{\varphi_n \} $ --- ортонормированная система (ОНС)

$ (\varphi_n, \varphi_m) = \delta _{nm} $

$ {\|\varphi\|}^2  = {\|\varphi_1\|}^2 +{\|\varphi_2\|}^2+ ... +{\|\varphi_n\|}^2 + ...$ \\


$ \{ \varphi_n \} $ полная в  $H$, если из $ (\varphi, \varphi_k) = 0 \enspace \forall k \in \mathbb{N} \quad \Rightarrow \quad \varphi = \mathbf{0} $


$ \forall \varphi \in H: \quad a_k = (\varphi, \varphi_k) - \textrm{ коэффициенты Фурье} $ \\


\textbf{Теор.} $H$ --- гильбертово, $\{\varphi_k\}$ --- полная ортонормированная система (ПОНС) \\ \\
$ \Rightarrow {\| \varphi \|}^2 = \sum\limits^{\infty}_{k=1} {|a_k|}^2  = \sum\limits^{\infty}_{k=1} {|(\varphi, \varphi_k)|}^2 $ --- равенство Парсеваля \\ \\ 


\textbf{Теор.} $ \exists a_k: \quad \sum\limits_{k=1}^{\infty} {|a_k|}^2 $ сходится, $\{\varphi_n\}$ --- ПОНС в $H$, тогда: \\ \\ $\sum\limits_{k=1}^{\infty} a_k \varphi_k$ сходится по $\|\boldsymbol{\cdot}\|$ к $\varphi \in H$, при этом $\|\varphi\| = \sum\limits_{k=1}^{\infty} {|a_k|}^2$. \\ \\


\textbf{Опр.} $H$ cепарабельно если $ \exists M - $ счетное мн-во плотное в H. \\

\textbf{Теор.} $H$ сепарабельно $ \Leftrightarrow \exists $ ПОНС (счетная или конечная) в $H$. \\ 

$ \{ u: \int\limits_{\Omega}^{} u dx = 0 \} $ --- пример подпространства в $ L_2(\Omega) $. \\ \\ 

Пусть $ H_1 $ --- подпространство в $H$

$ \forall \varphi \in H \quad \exists ! \varphi_1 \in H_1: \| \varphi - \varphi_1 \| = \underset{\psi \in H_1}{\min}  \| \varphi - \psi \| $ --- проекция $\varphi$ на $H_1$

$ \varphi = \varphi_1 + \varphi_2 $, \quad \quad $ H_2 = \varphi \perp H_1 $ --- ортогональное дополнение \\

$l$ --- линейный функционал $: \quad M \subset H \rightarrow \mathbb{R}/\mathbb{C} $

$ |l_{\varphi}| \leq \|l\| \cdot {\|\varphi\|}_H $

$ \underset{\psi \rightarrow \varphi}{\lim} l_\psi = l_\varphi \quad \quad \quad \forall \varepsilon > 0 \quad \exists \delta: \| \psi - \varphi \| < \delta: \quad |l_\psi - l_\varphi| < \varepsilon$ \\ \\

\textbf{Теор. (Рисса)} $ \forall l $ --- непрерывного линейного функционала в $H$ $\exists! \psi \in H:  l_\varphi = (\varphi, \psi)$ \\ \\


Пусть $M$ --- плотно в $H$, \quad $\Phi: M \times M \rightarrow \mathbb{C} (\mathbb{R})  $

$ \Phi(\varphi, \psi): \Phi(\varphi, \psi) = \overline{\Phi(\psi, \varphi)} $

$ \Phi(\varphi, \varphi) $ --- квадратичная форма \\ \\


$ H: D_A \subset H $ --- область определения некоторого оператора А

Линейный оператор $A$ ограничен $\Leftrightarrow A$ непрерывен

$\varphi \in D_A, \quad A\varphi \in R_A$ --- область значений оператора $A$

$\varphi \in D_A \rightarrow ! \enspace A\varphi \in R_A$

\newpage

\section{Лекция 3}

\begin{equation*}
	\left.\begin{aligned}
		&Au=f \\
		&u, f \in H
	\end{aligned}\quad \right | \quad \Omega \subset \mathbb{R}^m, \quad H = L_2(\Omega)
\end{equation*}

\[
\begin{cases}
	-\Delta u = f, \qquad f \in C(\overline{\Omega}) \\
	u |_{s} = 0
\end{cases}
\]
\[ D_A = \{u \in C^2 (\overline{\Omega}); \enspace u|_s = 0\} \]
\[ A = -\Delta u \] \\

\textbf{Формула Остроградского}

\[ \int\limits_{\Omega}\left(\frac{\partial \varphi}{\partial x } + \frac{\partial \psi }{\partial y} + \frac{\partial \omega}{\partial y}\right) d\Omega = \int\limits_{S } \biggl( \varphi \cos(\overline{n} \cdot x) + \psi \cos(\overline{n}\cdot y) + \omega \cos (\overline{n} \cdot z) \biggr) dS\]
\[ W = 
\begin{pmatrix}
	\varphi \\
	\psi \\
	\omega
\end{pmatrix}
\qquad \qquad \int\limits_{\Omega}^{} \text{div} W d\Omega = \int\limits_{S}^{} W_n dS
\] \\


Пусть $ \varphi = uv, \ \psi = \omega = 0 $

\[ \int\limits_{\Omega} u \frac{\partial v }{\partial x } d \Omega = - \int\limits_{\Omega}^{} v \frac{\partial u }{\partial x } d \Omega + \int\limits_{S}^{} uv \cos(\overline{n} \cdot x) dS \]

\begin{equation}
	\label{3.0}
	\tag{0}
	\int\limits_{\Omega } u \frac{\partial v }{\partial x_i} d \Omega = - \int\limits_{\Omega }^{} v \frac{\partial u}{\partial x_i} d \Omega + \int\limits_{S }^{} uv cos(\overline{n} \cdot x_i ) dS \qquad \text{в} \ \mathbb{R}^m
\end{equation} \\

\subsection{Формулы Грина}

\[ Lu = - \sum_{i,k =1}^{m } \frac{\partial }{\partial x_i} \left(A_{ik} (P) \frac{ \partial u(P)}{\partial x_k}\right) + C(P) u(P) \]

\[ D_L = \{ u \in C^2(\overline{\Omega}) \}, \quad P \in \Omega \subset \mathbb{R}^m, \quad C(P) \in C(\overline{\Omega}) \]
\[ A_{ik}(P) \in C(\overline{\Omega}), \quad A_{ik}(P) = A_{ki}(P) \enspace \forall P, \quad i,k=\overline{1,n} \]
\[ \int\limits_{\Omega} v Lu d \Omega = - \sum_{i,k =1}^{m }  \int\limits_{\Omega} v \frac{\partial }{\partial x_i} \left(A_{ik} \frac{ \partial u}{\partial x_k}\right) d \Omega + \int\limits_{\Omega}^{}C u v d \Omega  \]

в \eqref{3.0} подставим $ u \rightarrow v, v \rightarrow A_{ik} \frac{\partial u }{\partial x_k }$

\begin{equation}
	\label{3.1}
	\int\limits_{\Omega}^{} v L u d\Omega = \int\limits_{\Omega}^{} \sum_{i,k = 1}^{m}A_{ik} \frac{\partial u }{\partial x_k} \frac{\partial v }{\partial x_i} d\Omega + \int\limits_{\Omega} C u vd\Omega - \int\limits_{S }^{} v \sum_{i,k=1}^{m} A_{ik} \frac{\partial u}{\partial x_k } cos(\overline{n} \cdot x_i ) dS
\end{equation}

\begin{equation}
	\label{3.2}
	\int\limits_{\Omega}^{} u L u d\Omega = \int\limits_{\Omega} \left[ \sum_{i,k = 1}^{m}A_{ik} \frac{\partial u }{\partial x_i} \frac{\partial u }{\partial x_k} + Cu^2 \right] d\Omega - \int\limits_{S } u \sum_{i,k =1}^{m} A_{ik} \frac{\partial u}{\partial x_k } cos(\overline{n} \cdot x_i ) dS
\end{equation}

из \eqref{3.1} вычитаем ее же, но поменяв местами $u$ и $v$: \qquad $ \eqref{3.1} - \eqref{3.1}_{u \rightleftarrows v} $

\begin{equation*}
	\begin{split}
		\int\limits_{\Omega }^{} (vLu - uLv) d\Omega = & \int\limits_{\Omega} \cancelto{0}{\left[ \sum_{i,k = 1}^{m}A_{ik} \frac{\partial u }{\partial x_k} \frac{\partial v }{\partial x_i} - \sum_{i,k = 1}^{m}A_{ik} \frac{\partial u }{\partial x_i} \frac{\partial v }{\partial x_k} \right]} d\Omega \ - \\
		& - \int\limits_{S} \left[ v \sum_{i,k =1}^{m} A_{ik} \frac{\partial u}{\partial x_k } cos(\overline{n} \cdot x_i ) - u \sum_{i,k =1}^{m} A_{ik} \frac{\partial v}{\partial x_k } cos(\overline{n} \cdot x_k )\right] dS
	\end{split}	
\end{equation*}

\[ N \boldsymbol{\cdot} := \sum_{i,k=1}^{m } A_{ik } \frac{\partial \boldsymbol{\cdot} }{\partial x_i } cos(\overline{n} \cdot x_i) \]

\begin{equation}
	\label{3.3}
	\int\limits_{\Omega} \left( vLu - uLv \right) d\Omega = \int\limits_{S} \left( uNv - vNu \right) dS
\end{equation}

Частный случай формул Грина, это оператор Лапласа:

\[ Lu = - \Delta u; \ A_{ii} = 1; \ A_{ik} = 0, \ i \neq k; \ C = 0 \]

\begin{equation}
	\label{3.4}
	- \int\limits_{\Omega} v \Delta u d\Omega  = \int\limits_{\Omega}^{} \sum_{i=1}^{m } \frac{\partial u }{\partial x_i }  \frac{\partial v }{\partial x_i} d \Omega - \int\limits_{S }^{} v \frac{\partial u }{\partial n } dS
\end{equation}

\begin{equation}
	\label{3.5}
	- \int\limits_{\Omega} u \Delta u d \Omega = \int\limits_{\Omega}^{} {\left(\frac{\partial u }{\partial x_i}\right)}^2 d \Omega - \int\limits_{S }^{} u \frac{\partial u }{\partial n } dS
\end{equation}

\begin{equation}
	\label{3.6}
	- \int\limits_{\Omega} (v \Delta u - u \Delta v ) d\Omega = \int\limits_{S }^{} \left(v \frac{\partial u }{\partial n } - u \frac{\partial v}{\partial n}\right) dS
\end{equation}

\subsection{Положительные операторы}

Пусть оператор $A$ симметричен в $H$

\textbf{Опр.} Оператор называется положительным, если $ \forall u \in D_A \subset H, \qquad (Au, u) \geq 0 \Leftrightarrow u = 0 $ \\

\textbf{Пр. 1} 
\[Bu = -\frac{ d^2 }{d x^2 }u \qquad \textrm{в } L_2 (0,1); \qquad D_B = \{u \in C^2_0 (0,1): u(0) = u(1) = 0\} \]
\[ (B u, v) = - \int\limits_{0}^{1} v  \frac{d^2 u }{d x^2} dx = \int\limits_{0}^{1} \frac{du }{dx} \frac{d v }{d x } dx - v \left.\frac{d u }{dx }\right|^1_0 = - \int\limits_{0}^{1} u \frac{d^2v }{dx^2 } = (u, Bv) \quad \forall u,v \in D_B \]
\[ (Bu, u) = \int\limits_{0}^{1} {\left(\frac{du }{dx }\right)}^2 dx = 0 \]
\[ (Bu, u) = 0 \Rightarrow \frac{du }{dx } = 0 \Rightarrow u = const, u(0) = 0 \Rightarrow u = 0 \]

\textbf{Пр. 2}
\[ Cu = - \frac{ d^2 }{dx^2 }u, \qquad D_C = \left\{ u \in C^2(0,1),
	\begin{cases}
		u'(0)+\alpha u(0)=0 \\
		u'(1)+\beta u(1)=0
	\end{cases}
	\alpha, \beta = const
\right\}
\]
\[(Cu, v) = \int\limits_{0}^{1} \frac{du}{dx} \frac{dv}{dx} dx + \alpha u(0)v(0) + \beta u(1)v(1) = (u, Cv)\]
\[\alpha > 0, \beta \geq 0\]
\[(Cu, u) = \int\limits_{0}^{1} {\left(\frac{du}{dx}\right)}^2 dx + \alpha u^2(0) + \beta u^2(1) \geq 0\]
\[\alpha = \beta = 0, \quad u \equiv 1 \Rightarrow (Cu, u) = 0 \Rightarrow C \text{ не является положительным}\]

\textbf{Пр. 3}
\[ Au = - \Delta u , \qquad D_A = \{ u \in C^2(\Omega): \quad u|_s = 0, \quad \Omega \subset \mathbb{R}^m, S = \partial \Omega, H = L_2(\Omega) \} \]
\[ (Au, u) = (-\Delta u, u ) = - \int\limits_{\Omega }^{} u \Delta u d \Omega = \int\limits_{\Omega }^{} \sum_{i = 1}^{m } {\left(\frac{\partial u }{\partial x_i }\right)}^2 d\Omega - \cancelto{0}{\int\limits_{S }^{} u \frac{\partial u }{\partial n } dS}  \geq 0 \]
\[ \frac{\partial u }{\partial x_i } = const, \ u|_s = 0 \Rightarrow u = 0 \] 

\newpage

Рассмотрим мембрану

$\Omega$ в плоскости $(x,y)$, $\ u(x,y)$ --- изгиб мембраны
\[ - \Delta u = \frac{q}{T} \]

$q$ --- поперечная нагрузка на единицу площади

$T$ --- натяжение мембраны

$ {u|}_S = 0 $ --- мембрана закреплена на краях
\[ (A u , u) = (- \Delta u , u) = \iint\limits_{\Omega} \left[ \left(\frac{\partial u }{\partial x }\right)^2 + \left(\frac{ \partial u }{\partial y }\right)^2 \right] dx dy  \]

\subsection{Положительно определенные операторы}

\textbf{Опр.} Симметричный оператор $A$ называется положительно определенным, если 
\begin{equation}
	\exists \gamma > 0 : (Au, u) \geq \gamma^2 {\|u\|}^2
\end{equation}

\textbf{Пр. 1 (продолжение)}
\[ B: u(0) = 0, u \in D_B \]
\[ u(x) = \int\limits_{0}^{x } u'(t) dt, \quad x \in [0, 1] \]
\[ u^2(x) \leq \int\limits_{0}^{x} 1^2 dt \cdot \int\limits_{0}^{x} {(u'(t))}^2 dt = x \int\limits_{0}^{x} {(u'(t))}^2 dt \leq x \int\limits_{0}^{1} {(u'(t))}^2 dt \]
\[ \int\limits_{0}^{1} u^2 (x) dx \leq \frac{1}{2} \int\limits_{0}^{1} {(u'(t))}^2 dt \]
\[ \gamma^2 \|u\|^2 \leq (Bu, u) , \quad \gamma = \sqrt{2} \quad \Rightarrow B \text{ является положительно определенным} \]

\textbf{Пр. 4}
\[ Lu = - \frac{d }{ d x } \left(x^3 \frac{du}{dx}\right) \quad \text{в } L_2 (0,1) \]
\[ D_L = \{u \in C^2[0,1], \ u(1) = 0 \} \]
\[(Lu, v) - (u, Lv) = \int\limits_{0}^{1} \frac{d}{dx} \left[ x^3 \left(u \frac{dv }{dx } - v \frac{du }{dx } \right)  \right] dx = \left. \left[ x^3 \left(u \frac{dv }{dx } - v \frac{du }{dx } \right) \right] \right|_0^1 = 0\]
\[ (Lu, u) = \int\limits_{0}^{1} x^3 {\left(\frac{du }{dx }\right)}^2 dx \geq 0 \quad \Rightarrow L \text{ является положительно определенным} \]
\[ \frac{(Lu, u)}{\|u\|^2} \geq \gamma^2, \qquad u_\delta (x) = \begin{cases}
	(\delta - x)^3, & 0 \leq x \leq \delta \\
	0, & \delta \leq x \leq 1
\end{cases}, \qquad u_\delta \in \mathcal{D}_L  \]
\[ \frac{(Lu_\delta , u_\delta)}{{\|u_\delta\|}^2}  = \frac{\int_{0}^{1} x^3 {(\frac{du_\delta}{dx})}^2 dx}{\int_{0}^{\delta} {(\delta -x)}^3 dx} = \frac{9 \int_{0}^{1} x^3 {(\delta -x)}^4 dx}{\int_{0}^{\delta} {(\delta -x)}^6 dx} = \frac{9}{40} \delta \quad \Rightarrow L \text{ не явл. положительно опр.} \]

\newpage

\subsection{Энергетическая норма}

Пусть $A$ --- положительно определен в $H$ (гильберт.)

На $D_A: \quad {[u, v]}_A = {(A u, v)}_H$
 
Можно показать что выполняются все аксиомы скалярного произведения

\begin{enumerate}
	\item $ {[u, v]}_A = \overline{[v, u]}_A $ \\
		  $ (Au, v) = (u, Av) = \overline{(Av, u)} = \overline{[v, u]} $
	\item $ [a_1 u + a_2 u, v] = a_1[u, v] + a_2[u, v]$
	\item $ (Au, u) = [u, u] \geq \gamma \|i\|^2 \geq 0 $
	\item $ [u, u] = 0 \Leftrightarrow u = 0 $
\end{enumerate}

$ |u| = [u, u] $ --- энергетическая норма

$ D_A $ предгильбертово, дополним его по $ {|\boldsymbol{\cdot}|}_A \ \Rightarrow \ $ гильбертово пр-во $ H_A $ 

\[ u \in H_A \Leftrightarrow \left[ \begin{array}{l}
	u \in D_A \\
	\exists u : \ \{ u_n \} \in D_A: \ | u_n - u | \underset{n \rightarrow \infty}{\rightarrow} 0
\end{array} \right. \]

\newpage

\section{Лекция 4}

\subsection{10.02 Энергетические пр-ва (2)}

\[ H_A -\textrm{ энергетическое пр-во } \]

(0) \[ {||u||}_H \leq \frac{1}{\gamma   } {||u||}_A \]

\[ u \in H_A -> u \in \mathcal{D}_A \]
\[ -> \exists \{ u_n \} \in \mathcal{D}_A  {lim}_{n-> \infty} || u_n-u||_A \]

Теорема 
\[ \forall \exists u \in H_A -> \textrm{  злем из H различным } u_1 u_2 \in H_A \textrm{ отв разн. злем из H }\]

Доказательство 

...

2) \[  u_{1, n} \rightarrow_{||.||_A} u_n; u_{2,n} -\rightarrow_{||.||_A} u_2\]

\[ u_1 \textrm{ и  } u_2 \rightarrow u \textrm{ из H }; u = u_1 - u_2\]
...

\[  r \in H A \in \{ \in u_n \mathcal{D}_A ||u_n -n ||_A \rightarrow_{n \rightarrow \infty} 0 \} \]

\[ || u_n ||_A \rightarrow_{n \rightarrow \infty} ||u||_A \]



\subsection{Пример}

\[ Bu = - \frac{d^2}{d x^2} u ; D_{B} = {u \in C^2 (0,1]}, u(0) = u(1) = 0 \]
\[ H = :_2 (0, 1); \]

\[ u \in H_B; \exists \{ u_n \} \in D_B || u_h - u || \rightarrow_{B \rightarrow \infty} \rightarrow 0\]

...

A - положительно, но не положительно определено.

Теорема

\[ u \in H_A : u \in H \rightleftarrows \exists \{ u_n \} \in D_A \]
\[ || u = u_n || \rightarrow_{n \rightarrow \infty} 0 \]
\[ || u_k - u_n ||_H \rightarrow_{u,k \rightarrow + \infty} 0 \]

\subsection{Пример 3}

\subsection{Энергетический метод}
(для положительно определенных операторов)

\[ Au = f \label{eq:1} \]

\[ A: \mathcal{D} (A) \in H \rightarrow H; \]

Теорема 

A положителен в H уравнении \ref{eq:1} $ \exists $ не более одного решения.

Доказательство

\[ u_1, u_2 - \textrm{ Решения \ref{eq:1}} ... \]

Теорема о функциональной энергии

A - положительный в H; u - решение \ref{eq:1} $ \rightleftarrows $ доставляет минимум функционала

\[ F(u) = (Au, u)_H - (f, u)_H - (u, f)_H \label{eq:2}\]

Доказательство

...

\[  \]

Пример 4
\[ \Delta^2 \omega = \frac{\partial^4}{\partial x^4}\omega + 2 \frac{\partial^4 \omega }{\partial^2 \omega \partial^2 y} + \frac{\partial^4 \omega }{\partial y ^4} \]
\[ \mathcal{D}_A = \{ \omega \in c^4( \overline{\Omega}); \omega|_S = 0; \frac{\partial \omega}{\partial n} |_S = 0 \} \]

\[ A-\omega = \frac{a(x,y)}{\mathcal{D}} \]

\subsection{Обобщение решения задачи о min для ф.э.}

\[ A - \textrm{ Поллжительно определено в H } Au=f \ref{eq:1} f \in H \]
фикс $ f \in H \forall u \in H_A (u, f)_H :\textrm{ ф-ла } : H_A \rightarrow \mathcal{R} $

\[ | (u, f)_H | \leq ||f||_H {||u||}_H \leq {||f||}_H \frac{1}{\gamma} ||u||_A; \gamma ||f||_H - const \]

\[ \textrm{ Опр } (f, u) \Rightarrow \textrm{ по Т Рисса } \exists u_0\in H_A (f, u)_H = [u, u_0]_A \]

\[ F(u) = [u, u]_A - [u, u_0] - [u_0, u]_A \]

\[ +-[u_0, u_0]_A \label{eq:4}\]

\[ F(u) = || u - u_0 ||^2_A - ||u_0||^2_A  \]

\[ argmin_{u \in H_A} F(u) = u_0 \textrm{ Обощенное решение } Au = f \]

Если H сепарабельно, энергетическое про-во тоже сепарабельно, $ \exists \{ \omega_n \} $ ПОНС

\[ r_0 = \sum_{n=1}^{\infty} [u_0, \omega_n] \omega_n  \ref{eq:5}\]

\[ u = \omega_n [u_0 \omega_n]_A = (f, \omega_n)_H \]

\[ u_0 = \sum_{n=1}^{\infty} (f, \omega_n)_H \omega_n \]

\section{Лекция }

\subsection{ Применение энергетического метода для краевых задач}

\begin{enumerate}
	\item Немного опазадал, пример часть примера пропустил
	...
	\[ {(Lu, u)}_H = \sum_{k=0}^{m} \int_{x_1}^{x_2} p_k (x) (\frac{d^k u }{d x^k })^2 dx >= \int_{x_1}^{x^2} p_{n_1} (x) (\frac{d^m u }{d x^n } dx^3) >= p_0 \int_{x_1}^{x^2} (\frac{d^m u }{dx^m } dx = p_0 {||u_(m)||}^2_H) \]
	...
	\[ (L u_M ) >= \partial^2 {||u||}^2_H, \gamma = \sqrt{p_0} {(\frac{\sqrt{2}}{x_2-x_1})}^m \]
	...
	\[ {||u||}_A \leq \sqrt{p_0 } {||u||^{(m)}}_H \exists \{ u_N (x) \}\]

	\[ {lim}_{n \rightarrow \infty} = 0; u_0 - \textrm{ точное решение  } \]

	\[ {||u_n - u_k ||}_A <= {||u_n - u_0||}_A + {|| u_k - u_0||}_A \rightarrow 0 \]
	\[ u^{(l)}_n (x_1) = u_k^(l) (x_1) = 0, l = \overline{0, m-1} \]
	...

	\item Изгиб балки
	\[ L_{\omega} = \frac{d^2}{dx^2} [E I(x) \frac{d^2 \omega }{dx^2}] + K\omega = q(x)\]
	\[ \omega - \textrm{ Прогиб балки } \]
	\[ E - \textrm{ модуль Юнга } \]
	\[ I(x) - \textrm{ момент инерции } \]
	\[ q(x) - \textrm{ интенсивность нагрузки на балку } \]
	\[ K - \textrm{ коэф податливости основания } \]

	\[ \omega (0) = \omega (l) = 0 \]
	\[ \omega'(0) = \omega'(l) = 0 ; A - \textrm{ Положительно определен } \]
	Аналогично задачи минимизации функционала

	\[ F(\omega) = \int_{0}^{l} (E I(x) {\omega''}^2 + K\omega^2 - 2q(x \omega)) dx = (L\omega , \omega) -2 (\omega, q)\]

	Воспользуемся методом Рица

	\[ {u_n (x)}^{\infty}_{n=1}, \phi_n (x) = {(x-l)}^2 x^{n+1}, \textrm{ Полная система в  } H_A \]

	\[ \omega_n = \sum_{k=1}^{n} a_k \phi_k (x) = {(x-l)}^2 \sum_{k=1}^{n} a_k x^{k+1} \]
	\[ \sum_{n}^{k=1} a_k A_{1k} = b_{ij}; i=\overline{1, n} \]
	\[ b_j = (q, \phi_j)_H = \int_{0}^{l } a(x) (x-l) x dx \]
	\[ A_{ik} = (L \phi_i, \phi_k)_H = \int_{0}^{l} (E I(x) \frac{d^2 \phi_i}{d x^2} \frac{d^2 \phi_k}{d x^2} + k\phi_i \phi_k) dx \]
	...
	\[ \omega(0) = 0; \omega''(l) = 0 \]
	\[ \omega' (0) = 0 \]
	\[ \frac{d }{d x } (E I(x) \frac{d^2 \omega }{d x^2})^{x=l}_{x=0} = 0 \]
	Тут тоже можно доказать полажительную определенность
	
	\item Краевая задача для систем ОДУ
	\[ - \sum_{k=1}^{s} [ \frac{d }{dx }(p_{jk}(x) \frac{du_k (x)}{dx}) - q_{jk}(x) u_k (x)] = f_j(x) \]
	краевые ...

	\[ - \frac{d }{dx } [P(x) \frac{du }{dx }] + Q(x) u(x) = f(x) \]
	\[ u(x_1) = u (x_2) = 0 \]
	\[ (u, v)_{H= L_2(x_1, x_2)} = \int_{x_1}^{x_2} u (x) \cdot v(x) dx = \int_{x_1}^{x_2} \sum_{k=1}^{s} u_k(x) v_k (x) dx \]

	\subsubsection{Теорема}
	\[ P(x), Q(x) \textrm{ симметр. } x \in [x_1, x_2] \Rightarrow A \textrm{ Симметричный } \]
	Доказательство
	\[ {(Au, v)}_H = - \int_{x_1}^{x^2} v(x) \cdot \frac{d}{dx} [P(x) \frac{du }{dx }] dx + \int_{x_1}^{x_2} v(x) \cdot Q(x u(x dx)) = \]
	\[ = \int_{x_1}^{x_2} P \frac{du }{dx } \cdot \frac{dv }{dx } + v(x \cdot Q(x) u(x))dx \]
	\[ Q u \cdot v = \sum_{j,k=1}^{s } q_{jk u_k \cdot v_j} =  \]
	\[ = \sum_{i,j =1}^{s} q_{k,j} v_j \cdot v_k \]
	Следовательно оператор симметричен

	\subsubsection{Теорема}
	\[ P(x), Q(x) \textrm{ симметрич на } [x_1, x_2] \]
	\[ P(x)  \textrm{ положит. опр. } Q(x)\textrm{  неотр на } (x_1, x_2] \Rightarrow A \textrm{ положительно определен } \]
	доказательство
	\[ P(x) \textrm{ пол. опр } \forall x \Rightarrow \textrm{ пусть } \lambda_1(x) >0  \]
	\[ \exists \lambda > 0 = const; \lambda_1(x) > \hat{\lambda}> 0  x \in [x_1, x_2] \]

	\[ \forall t = (t, ... , s ) \]
	\[ P(x) t \cdot t = \sum_{j, k=1}^{s } P_{jk}(x)t_j t_k \geq  \lambda_1(x) \sum_{k=1}^{s } t_k^2 \geq \]
	\[ \geq \hat{\lambda } \sum_{k=1}^{s } t^2_k \]
	\[ Q(x) t\cdot t = \sum_{k, k=1}^{s }q_{jk} t_j t_k \geq 0 \]

	\[ (u, u)_H = \int_{x_1}^{x_2} (P \frac{du }{dx } \cdot \frac{du }{dx}) dx \geq \hat{\lambda } \int_{k=1}^{s} (\frac{du_k}{idx}^2) dx \]
	\[ (Au, u)_H \geq \frac{2 \hat{\lambda}}{{(x_2 - x_1)}^2} \int_{x_1}^{x_2} (\sum_{k=1}^{s } u_k^2) dx = \frac{2 \hat{\lambda}}{{(x_2 - x_1)}^2} dx = \frac{\hat{\lambda}}{{(x_2-x_1)}^2} {||u||}^2_H \]

	\[ {(Au, u)}_H \geq \gamma^2 {||u||}^2_H \]
	...


\end{enumerate}

\subsection{Основные кр задачи для ур-я Пуассона}

\[ -\Delta u = f(p) \textrm{ в } \Omega \in \mathcal{R}^m \]\label{eq:5_1}

з. Дирихле

\[ u|_{\partial \Omega} = 0 \]\label{eq:5_2}
\[ A u = - \Delta u = \sum_{k=1}^{m } \frac{\partial^2 u }{\partial x_k^2} \]
\[ P_A = \{ u \in c^2 (\overline{\Omega}_1) u |_{2 \Omega} = 0 \} \]
\[ H = L_2 (\Omega) \]
\[ (- \Delta, u )_h = \int_{\Omega}^{} \sum_{i=1}^{n} (\frac{\partial n }{\partial x_i})^2 d \Omega - \int_{ \partial \Omega}^{} u \frac{\partial u }{\partial n } dS = \int_{\Omega}^{} {({grub} u)}^2 d\Omega \geq 0 \]\label{eq:5_3}

\[  \rightleftarrows \]
\[ F(u) = (-\Delta u, u )_H -2(u, f)_H  \]\label{eq:5_4}
\[ F(u) = \int_{\Omega }^{} ((grad u )^2 - 2uf) d\Omega  \]\label{eq:5_5}

\[ [\frac{\partial u }{\partial n} + \gamma (P) u ] |_{\partial \Omega} = 0 \]\label{eq:5_6}

\[ (- \Delta u, u)_H = \int_{\Gamma}^{} (grad u)^2 d\Omega + \int_{\partial u^2}^{} dS \geq 0 \]

\[ (-\Delta u, u)_H =0 \Rightarrow u = condt \int_{ \partial \xi}^{} \gamma c^2 dS = 0 \Rightarrow c = 0 \Rightarrow u= 0\]

\[ F(u) = \int_{\Omega}^{} ((grad u)^2 - 2uf) d\Omega + \int_{\gamma \Omega} \gamma n^2 dS \]\label{eq:5_7}

\[ \frac{\partial u }{\partial n}|_{\partial \Omega } = 0 \]\label{eq:5_8}

з Неймана

\ref{eq:5_1}, \ref{eq:5_8} 

\[ (- \Delta, u )_H = - \int_{\partial \Omega}^{} u  \frac{\partial u}{\partial n} dS + \int_{\Omega}^{} (grad u )^2 d\Omega \geq \]

\[ u == 1 (- \Delta u, u)_H = 0 \]

\[ \textrm{ при  } V == 1 \]
\[ \int_{\Omega }^{} \Delta u d\Omega = \int_{\partial \overline{\Omega}}^{} \frac{\partial u }{\partial n } dS = 0 \]

\[ S_{\Omega} f d \Omega = 0\]
Условие разрешимости \ref{eq:5_1} \ref{eq:5_8}


\end{document}
\documentclass[12pt, a4paper]{article}

\usepackage[T1,T2A]{fontenc}
\usepackage[utf8]{inputenc}
\usepackage[english,russian]{babel}

\usepackage{pdfpages}
\usepackage{multirow}

\usepackage{caption}

\usepackage[fleqn]{amsmath}
\usepackage{amssymb}
\usepackage{cancel}

\usepackage[hidelinks]{hyperref}

\usepackage{graphicx}%Вставка картинок

\usepackage{float}%"Плавающие" картинки
\usepackage{wrapfig}%Обтекание фигур (таблиц, картинок и прочего)


\setlength{\emergencystretch}{10pt}

\usepackage{indentfirst} %Отступ после секции

\usepackage[left=2cm,right=1cm,
top=2cm,bottom=2cm,bindingoffset=0cm]{geometry}

\usepackage{setspace}

\usepackage{tocloft}
\setlength\cftsecnumwidth{0em}

\usepackage{titlesec}% http://ctan.org/pkg/titlesec
\titleformat{\section}%
[hang]% <shape>
{\normalfont\bfseries\Large}% <format>
{}% <label>
{0pt}% <sep>
{}% <before code>
\renewcommand{\thesection}{}% Remove section references...
\renewcommand{\thesubsection}{\arabic{section}.\arabic{subsection}}%... from subsections

\begin{document}

\thispagestyle{empty}

\begin{center}
	\ \vspace{-1cm}

	{Московский государственный университет имени М. В. Ломоносова}\\
	Факультет вычислительной математики и кибернетики\\
	Кафедра вычислительных методов

	\vspace{8cm}
	\begin{spacing}{2.5}
		{\huge \bfseries ВАРИАЦИОННО-ПРОЕКЦИОННЫЕ МЕТОДЫ В ЗАДАЧАХ МАТЕМАТИЧЕСКОЙ ФИЗИКИ}
	\end{spacing}


\end{center}

\vfill

\begin{center}
	Москва, 2024
\end{center}

\enlargethispage{2\baselineskip}

\newpage

\tableofcontents

\newpage

\section{Лекция 1}

\subsection{Исторический экскурс}

Лекции с 9:30 два часа.

Вариационная постановка для задача мат физики.
задача - условие на границе + начальное. (Экстремум функционала энергии, поэтому энергетические пространства).

Соболев - прямые методы решения задач. позволяют найти решение с помощью СЛАУ. Наиболее известен метод Рица.

\subsection{Метод Дирихле}

Дана область $ \omega \in \mathbb{R}^2$.

\[ M = {u:u_{0}(x,y), (x,y) \in \partial \Omega} \]
\[ \iint_{\Omega}[{(\frac{\partial u}{\partial x})}^2+{(\frac{\partial u}{\partial y})}^2] dx dy \rightarrow min \]
Интеграл Дирихле $ \Rightarrow \overline{u} $ - гармонический в $ \Omega $

\subsection{Контрпример Вейерштрасса. }

\[ M={y;y(x) \in c'[-1;1]}, y(-1)=-1, y(1)=1 \]
\[ J(y) = int^1_{-1} x^2{(y')}^2 dx, J(y) \geq 0 \]
\[ y_\varepsilon(x) = \frac{arctg(\frac{x}{\varepsilon})}{arctg(\frac{1}{\varepsilon})} \]
\[ {y'}_{\varepsilon}(x) = \frac{1}{arctg(\frac{1}{\varepsilon})} \cdot \frac{1}{1+ \frac{x^2}{\varepsilon^2}} \cdot \frac{1}{\varepsilon} = \]
\[ = \frac{\varepsilon} {arctg (\frac{1}{\varepsilon})} \cdot \frac{1}{\varepsilon^2+x^2} \]
\[ J(y_{\varepsilon}) = \int_{-1}^1 \frac{x^2 \varepsilon^2}{arctg^2(\frac{1}{\varepsilon})} \cdot \frac{1}{{(x^2+y^2)}^2} dx = frac{2 \varepsilon}{arctg(\frac{1}{\varepsilon})} = \frac{0}{\frac{\pi}{2}}\]
\[ J(\overline{y}) = \int_{-1}^{1} x^2 y^2 dx = 0 \Rightarrow y'=0 \]
Противоречие: $ y(-1)=-1, y(1)=1 $

\subsection{Контрпример Адамара}

\[ u(x,y) = \sum_{n=1}^{\inf}\frac{\rho^2}{2^n} cos(2^n \Theta), x=\rho cos \Theta, y=\rho sin \Theta \]
\[ \rho \leq 1 \]
Непрерывны и гармоничны.
Интеграл Дирихле в круге $ \rho \leq r \leq 1 $
\[ \pi sum_{n=1}^{\inf} {r^2}^{2n+1} {\rightarrow}_{r \rightarrow 1} \inf \]

\subsection{Метод Ритца}

\[ J(\omega) = int_{a}^{b} f(x,\omega, \omega', ... , \omega^{(k)}) dx \rightarrow inf \]
\[ \omega \in M \textrm{ класс допустимых функций } \]
\[ \psi_0, \psi_1, ... \psi_n, ... (\textrm{ координатные функции }) \]
Св-ва:
\[ 1) \forall a_1 ... a_n \in \mathbb{R}, \forall_n \]
\[ \omega_n = \omega_0+ \sum_{i=1}^{n} a_i \omega_i \in M \]
\[ 2) \forall \omega \in M\textrm{и} \forall varepsilon > 0 \]
*Уравнение полноты*
\[ H(\omega n) = F(a_1, ... , a_n) \rightarrow inf \]
\[ || \omega - \psi_0 - \sum_{i=1}{n} a_i \psi_i || < \varepsilon \]
Рассмотрим:
\[ J(\omega_n) = F(\alpha_1, ..., \alpha_n) \rightarrow inf \]
\[ \frac{\partial J}{\partial a_1}(\omega_n)=0, ... \frac{\partial J}{\partial a_n}(\omega_n) = 0 -\textrm{ альтернативная система уранений } \]
\[ \Rightarrow a_1, ... , a_n - \textrm{ решение } \]

Насколько хорошо приближает метод искомое решение?
На примере задачи об упругой поластине.

\[ \Omega_{\subset \mathbb{R}^2} - \textrm{ обл }, S = \partial \Omega \]
\[ \textrm{ изгиб } \omega(x,y) \textrm{ удовлетворяет ур-ю Сори Жульен }\]
\[ \Delta^2 \omega = \frac{\partial^4}{\partial x^2} + 2 \frac{\partial^4 \omega}{{\partial x}^2 {\partial y}^2} + \frac{\partial^4 \omega}{{\partial y}^4} = \frac{q(x y)}{\mathcal{D}}; (x,y) \in \Omega \]
\[ \mathcal{D} - \textrm{ жесткость пластины при упругом изгибе } \]
\[ q(x,y),\textrm{- Интенсивность давления } \]
\[ \omega (x,y) = 0 \]
\[ \frac{\partial \omega (x,y)}{\partial\mathcal{D} } = 0 \leftarrow \textrm{ Производная по нормали к S } \]
\[ J(\omega) = \iint_{Omega}( \frac{1}{2} (\Delta \omega)^2 - f(\omega) d \Omega \rightarrow inf) \]
\[ f = \frac{q(x,y)}{\mathcal{D}} \in C'(\overline{\Omega}) \]
\[ \omega = \omega_1 + \omega_2 \]
Рассмотрим без доказательства ограниченности снизу.
\[ \omega_1 = \frac{1}{8\pi} \iint_{\Omega} r^2 ln r f(\xi, \eta) d\xi \eta \]
\[ (x,y) (\xi, \eta) - \textrm{ точки из } \Omega r-\textrm{ расстояние между } (x,y)\textrm{ и } (\xi, \eta) \]
\[ J(\omega) = J_0 + \frac{1}{2} \iint_{\Omega} (\Delta \omega_2)^2 dx dy \]
\[ j(\omega) \geq J_0 \Rightarrow \exists inf J(\omega)\]
Введем $ \psi_1(x,y), ... , \psi_n(x,y) $ - координатные ф-ции
\[ 1) \psi_n(x,y), \frac{\partial^{k+l} \psi_n}{\partial x^k \partial x^l} \in C( \overline{ \Omega}), k \leq \varepsilon , l \leq \varepsilon \]
\[ 2) \psi_n(x,y) \textrm{ удовлетворяет краевым условиям} \]
\[ 3) \forall \textrm{ ф-ии } \zeta(x,y): \]
\[ \textrm{ a) удовлетворяет пункту 1 } \]
\[ \textrm{ б) } \zeta(x,y) \equiv 0 (x,y) \in \Omega \rho \]
\[ \Rightarrow \exists m \in \mathbb{N} , \alpha_1, ... \alpha_m \in \mathbb{R}: \]
\[ |\zeta(x,y) - \sum_{i=1}^{m} \alpha_i \psi_i (x_i, y_i)|< \varepsilon \]
\[ | \frac{\partial^{k+1}\zeta}{\partial x^k \partial y^l} - \sum_{i=1}^{n} \alpha_i \frac{\partial^{k+l} \psi_i (x,y)}{\partial x^k \partial y^l}| < \varepsilon \]
Условие полноты $ k \leq \varepsilon, l \leq \varepsilon \Rightarrow $

приближенное решение :
\[ \omega_n = \alpha_1 \psi_1+ ... + \alpha_n \psi_n \rightarrow J(\omega) \]
\[ J_n = \iint_\Omega (\frac{1}{2} {(\Delta \omega_n)}^2 - f (\omega_n))dx dy \]
\[ \alpha_i \textrm{ выбираем : } J (\omega_n) \rightarrow J(\omega) \]
\[ \sum_{k=1}^n A_{ik} a_k = B_i, i= \overline{1,n} \label{*} \]
\[ A_{ik} = \iint_\Omega \Delta \psi_i \Delta \psi_k dx dy \Rightarrow \]
\[ \exists! \textrm{ решение } a_1, ... , a_n \textrm{ в } \omega_n = ... \textrm{ приближение решение } \]
\[ B_i = \iint_\Omega f\psi_i dx dy \]
\[ \rightarrow \textrm{ Сущ ед решения } a_1, ..., a_n \textrm{в} \omega_n = ... (\textrm{ приближенное решение }) \]
\[ \textrm{ Рассмотрим } \forall b_1, ... b_n \]
\[ \zeta_n = b_1 \psi_n + ... + b_n \xi_n \]
\[ \ref{*} b_i \textrm{ и } \sum_{i=1}^{n} \]
\[ \sum_{i=1}^n \sum_{k=1}{n} A_{ik} a_k b_i = \sum_{i=1}^{n} b_i B_i \]
\[ \sum_{i=x}^n \sum_{k=1}{n} \iint_\Omega b_i \Delta \psi_i \Delta \psi_k a_k dx dy - \sum_{i=1}^{n} \iint_\Omega f b_i \psi_i dx dy = 0 \]
\[ \sum_{i=1}^{n} [ \iint_\omega b_i \Delta \psi_i \sum_{k=1}^{n} a_k \Delta \xi_k dx dy - \iint_{\Omega f b_i \psi_i dx dy}] = 0 \]
...
\[ \int_Omega (\Delta \omega_n \sum_{i=1}{n} b_i \psi_i) - f(\sum_{i=1}{n} b_i \psi_i) dx dy = 0 \]
\[ \iint_Omega(\Delta \Omega_n \zeta_n - f \zeta) dx dy = 0 \]
\[ J_n^{(0)} = -\frac{1}{2} \iint_\Omega {(\Delta \omega_u)}^2 dx dy \textrm{ не возрастает y } \geq inf \]
\[ \forall \varepsilon > 0 \textrm{ по критерию Коши } \Rightarrow N(\varepsilon) \forall_n > N(\varepsilon) \]

\newpage

\section{Лекция 2}

\[ \varphi_1 (x,y), ..., \varphi_n(x,y) \text{ --- координатные функции}, \quad \quad  w_n = \alpha \varphi_1 + ... + \alpha_n \varphi_n \]
\[ J_n^{(0)} = -\frac{1}{2} \iint\limits_{\Omega} {(\Delta w_n)}^2 dxdy \]
\[ \forall \varepsilon >0 \ \exists N(\varepsilon) \quad \forall n \geq N(\varepsilon) \quad \forall m: 0 \leq J_n^{(0)} - J_{n+m}^{(0)} \leq \frac{1}{2} \varepsilon \]
\[ \frac{\omega_{m+n} - \omega_n}{\sqrt{\varepsilon}} = \varphi (x,y) \]
\[ \iint\limits_{\Omega}{(\Delta \varphi)}^2 dx dy < 1 \]

Обозначим $S = \partial \Omega$ --- границу области $\Omega$

\[ \varphi (x,y) = \frac{1}{2\pi}\int\limits_S \left( \varphi \frac{\partial (\ln r)}{\partial n} - \ln r \frac{\partial  \varphi}{\partial N} \right) dS + \frac{1}{2 \pi} \int\limits_{\Omega} \Delta \varphi \ln r \ d\xi d \eta \]
\[ {\left| \int\limits_{x} f(x) \overline{g}(x)dx \right|}^2 \leq \left( \int\limits_{x} {|f(x)|}^2 dx \right) \left( \int\limits_{x}{|g(x)|}^2 dx \right) \]
\[ |\varphi (x,y)| \leq \frac{1}{2\pi}{\left( \iint\limits_{\Omega} {(\Delta \varphi)}^2 d \xi d \eta \right) }^{1/2} {\left( \iint\limits_{\Omega}{\ln}^2 r \ d\xi d\eta \right)}^{1/2} \]
\[ |\varphi (x,y)| \leq C_1 \]
\[ |\omega_{n+m} - \omega_{n} | \leq C_1 \sqrt{\varepsilon} \]
\[ \omega_n \underset{\Omega}{\rightrightarrows}  w_n(x,y) \in C(\Omega) \]

\subsection{Метод Бубнова -- Галеркина}

$ w_n =  \alpha_1 \varphi_1 + ... + \alpha_n \varphi_n\ $

$ L w - \lambda M w = 0 $

$L, M$ --- дифференциальные операторы

$ \sum\limits_{i=1}^{n} (A_{ik} - \lambda B_{i_k}) a_k = 0, \quad k = \overline{1,n} $

\[
\begin{vmatrix}
	A_{11} - \lambda B_{11} & \dots & A_{1n} - \lambda B_{1n} \\
	\vdots & \ddots & \vdots \\
	A_{n1} - \lambda B_{n1} & \dots & A_{nn} - \lambda B_{nn}
\end{vmatrix}
= 0
\] \\

$ N (x,y) = L w_n - \lambda M w_n \enspace \text{--- \enspace невязка} $

$ N(x,y) \perp \varphi_i, \quad i = \overline{1,n} $

\subsection{Повторение}

\begin{enumerate}
	\item $ f(x) \overset{\text{п.в.}}{=} 0 \Rightarrow \int\limits_{\Omega} f(x) dx = 0 $
	\item $ \int\limits_{\Omega} f(x) dx = 0, \enspace f(x) >= 0 \enspace \Rightarrow \enspace f(x)\overset{\text{п.в.}}{=} 0 $
	\item $ |f(x)| < \varphi(x), \varphi \text{ --- суммируема по Лебегу} \Rightarrow f(x) \text{ --- суммируема по Лебегу} $
	\item $ \{\varphi_n(x)\} $ --- суммируемы с квадратами по Лебегу
	\\ \\
	$\lim\limits_{n,k \rightarrow \infty} \int\limits_{\Omega}{|\varphi_k (x) - \varphi_{n} (x)|}^2 dx = 0
	$
	\\ \\ \\
\end{enumerate}

Обозначим $V$ -- линейное пространство \\

$ (\varphi , \psi) - \text{скалярное произведение:} \quad (\boldsymbol{\cdot},\boldsymbol{\cdot}): V \times V \rightarrow \mathbb{C} $

\begin{enumerate}
	\item $  (\varphi, \psi) = \overline{(\psi, \varphi)} $
	\item $ (a_1 \varphi_1 + a_2 \varphi_2, \psi) = a_1 (\varphi_1, \psi) + a_2 (\varphi_2, \psi) $
	\item $ (\varphi , \varphi) \geq 0 $
	\item $ (\varphi, \varphi) = 0 \quad \Rightarrow \quad \varphi = \mathbf{0} $
\end{enumerate}

$ \| \varphi \| = \sqrt{(\varphi, \varphi)} $ \\

\begin{itemize}
	\item Неравенство Коши-Буняковского

	$ | (\varphi, \psi) | \leq \| \varphi \| \| \psi \|$

	\item Неравенство треугольника

	$ \| \varphi + \psi \| \leq \| \varphi \| + \| \psi \| $
\end{itemize}
\[ L_2(\Omega): \quad (\varphi, \psi) = \int\limits_{\Omega}^{} \varphi(x) \overline{\psi(x)}dx \]
\[ L_2(\Omega , \sigma): \quad (\varphi, \psi) = \int\limits_{\Omega}^{} \varphi(x) \overline{\psi(x)}\sigma (x) dx \]
\[ L_2(\Omega^m): \quad (\varphi, \varphi) = \int\limits_{\Omega}^{} \sum_{k=1}^{m} \varphi_k(x) \overline{\varphi_k(x)}dx \]

\newpage

\textbf{Критерий линейной зависимости системы функций}

\begin{gather*}
	\varphi_1, ..., \varphi_n \text{ линейно зависима (ЛЗ) в } H
\\
	\hspace{20mm} \Updownarrow
\\
	\begin{vmatrix}
		(\varphi_1, \varphi_1) & \dots & (\varphi_1, \varphi_n) \\
		\vdots & \ddots & \vdots \\
		(\varphi_n, \varphi_1) & \dots & (\varphi_n, \varphi_n)
	\end{vmatrix}
	= 0
\end{gather*} \\

\textbf{Опр.} $M$ --- плотно в $H$, если $ \forall p \in H$ и $\forall \varepsilon >0 \enspace \exists \varphi_n \in M: \| \varphi_n - \varphi \| < \varepsilon $. \\

$ C_0^{(\infty)} (\Omega) $ плотно в $ L_2(\Omega) $

$\quad \quad \quad \quad \quad \uparrow$

$ \forall \varepsilon > 0 : \quad \forall \varphi \in H \quad $
\begin{tabular}[t]{l}
	$ \exists \varphi_n^1 \in C_0^{(\infty)}(\Omega) : \quad \|\varphi_n^1 - \varphi\| < \varepsilon/2 $ \\
	$ \exists \varphi_n^2 \in C_0^{\infty} (\Omega) : \quad \|\varphi_n^2 - \varphi_n^1\| < \varepsilon/2 $ \\
	$...$
\end{tabular}

$ C_0^{(k)} (\Omega) $ плотно в $ L_2(\Omega) $ \\ \\

$ \{\varphi_n \} $ --- ортонормированная система (ОНС)

$ (\varphi_n, \varphi_m) = \delta _{nm} $

$ {\|\varphi\|}^2  = {\|\varphi_1\|}^2 +{\|\varphi_2\|}^2+ ... +{\|\varphi_n\|}^2 + ...$ \\


$ \{ \varphi_n \} $ полная в  $H$, если из $ (\varphi, \varphi_k) = 0 \enspace \forall k \in \mathbb{N} \quad \Rightarrow \quad \varphi = \mathbf{0} $


$ \forall \varphi \in H: \quad a_k = (\varphi, \varphi_k) - \textrm{ коэффициенты Фурье} $ \\


\textbf{Теор.} $H$ --- гильбертово, $\{\varphi_k\}$ --- полная ортонормированная система (ПОНС) \\ \\
$ \Rightarrow {\| \varphi \|}^2 = \sum\limits^{\infty}_{k=1} {|a_k|}^2  = \sum\limits^{\infty}_{k=1} {|(\varphi, \varphi_k)|}^2 $ --- равенство Парсеваля \\ \\


\textbf{Теор.} $ \exists a_k: \quad \sum\limits_{k=1}^{\infty} {|a_k|}^2 $ сходится, $\{\varphi_n\}$ --- ПОНС в $H$, тогда: \\ \\ $\sum\limits_{k=1}^{\infty} a_k \varphi_k$ сходится по $\|\boldsymbol{\cdot}\|$ к $\varphi \in H$, при этом $\|\varphi\| = \sum\limits_{k=1}^{\infty} {|a_k|}^2$. \\ \\


\textbf{Опр.} $H$ cепарабельно если $ \exists M - $ счетное мн-во плотное в H. \\

\textbf{Теор.} $H$ сепарабельно $ \Leftrightarrow \exists $ ПОНС (счетная или конечная) в $H$. \\

$ \{ u: \int\limits_{\Omega}^{} u dx = 0 \} $ --- пример подпространства в $ L_2(\Omega) $. \\ \\

Пусть $ H_1 $ --- подпространство в $H$

$ \forall \varphi \in H \quad \exists ! \varphi_1 \in H_1: \| \varphi - \varphi_1 \| = \underset{\psi \in H_1}{\min}  \| \varphi - \psi \| $ --- проекция $\varphi$ на $H_1$

$ \varphi = \varphi_1 + \varphi_2 $, \quad \quad $ H_2 = \varphi \perp H_1 $ --- ортогональное дополнение \\

$l$ --- линейный функционал $: \quad M \subset H \rightarrow \mathbb{R}/\mathbb{C} $

$ |l_{\varphi}| \leq \|l\| \cdot {\|\varphi\|}_H $

$ \underset{\psi \rightarrow \varphi}{\lim} l_\psi = l_\varphi \quad \quad \quad \forall \varepsilon > 0 \quad \exists \delta: \| \psi - \varphi \| < \delta: \quad |l_\psi - l_\varphi| < \varepsilon$ \\ \\

\textbf{Теор. (Рисса)} $ \forall l $ --- непрерывного линейного функционала в $H$ $\exists! \psi \in H:  l_\varphi = (\varphi, \psi)$ \\ \\


Пусть $M$ --- плотно в $H$, \quad $\Phi: M \times M \rightarrow \mathbb{C} (\mathbb{R})  $

$ \Phi(\varphi, \psi): \Phi(\varphi, \psi) = \overline{\Phi(\psi, \varphi)} $

$ \Phi(\varphi, \varphi) $ --- квадратичная форма \\ \\


$ H: D_A \subset H $ --- область определения некоторого оператора А

Линейный оператор $A$ ограничен $\Leftrightarrow A$ непрерывен

$\varphi \in D_A, \quad A\varphi \in R_A$ --- область значений оператора $A$

$\varphi \in D_A \rightarrow ! \enspace A\varphi \in R_A$

\newpage

\section{Лекция 3}

\begin{equation*}
	\left.\begin{aligned}
		&Au=f \\
		&u, f \in H
	\end{aligned}\quad \right | \quad \Omega \subset \mathbb{R}^m, \quad H = L_2(\Omega)
\end{equation*}

\[
\begin{cases}
	-\Delta u = f, \qquad f \in C(\overline{\Omega}) \\
	u |_{s} = 0
\end{cases}
\]
\[ D_A = \{u \in C^2 (\overline{\Omega}); \enspace u|_s = 0\} \]
\[ A = -\Delta u \] \\

\textbf{Формула Остроградского}

\[ \int\limits_{\Omega}\left(\frac{\partial \varphi}{\partial x } + \frac{\partial \psi }{\partial y} + \frac{\partial \omega}{\partial y}\right) d\Omega = \int\limits_{S } \biggl( \varphi \cos(\overline{n} \cdot x) + \psi \cos(\overline{n}\cdot y) + \omega \cos (\overline{n} \cdot z) \biggr) dS\]
\[ W =
\begin{pmatrix}
	\varphi \\
	\psi \\
	\omega
\end{pmatrix}
\qquad \qquad \int\limits_{\Omega}^{} \text{div} W d\Omega = \int\limits_{S}^{} W_n dS
\] \\


Пусть $ \varphi = uv, \ \psi = \omega = 0 $

\[ \int\limits_{\Omega} u \frac{\partial v }{\partial x } d \Omega = - \int\limits_{\Omega}^{} v \frac{\partial u }{\partial x } d \Omega + \int\limits_{S}^{} uv \cos(\overline{n} \cdot x) dS \]

\begin{equation}
	\label{3.0}
	\tag{0}
	\int\limits_{\Omega } u \frac{\partial v }{\partial x_i} d \Omega = - \int\limits_{\Omega }^{} v \frac{\partial u}{\partial x_i} d \Omega + \int\limits_{S }^{} uv cos(\overline{n} \cdot x_i ) dS \qquad \text{в} \ \mathbb{R}^m
\end{equation} \\

\subsection{Формулы Грина}

\[ Lu = - \sum_{i,k =1}^{m } \frac{\partial }{\partial x_i} \left(A_{ik} (P) \frac{ \partial u(P)}{\partial x_k}\right) + C(P) u(P) \]

\[ D_L = \{ u \in C^2(\overline{\Omega}) \}, \quad P \in \Omega \subset \mathbb{R}^m, \quad C(P) \in C(\overline{\Omega}) \]
\[ A_{ik}(P) \in C(\overline{\Omega}), \quad A_{ik}(P) = A_{ki}(P) \enspace \forall P, \quad i,k=\overline{1,n} \]
\[ \int\limits_{\Omega} v Lu d \Omega = - \sum_{i,k =1}^{m }  \int\limits_{\Omega} v \frac{\partial }{\partial x_i} \left(A_{ik} \frac{ \partial u}{\partial x_k}\right) d \Omega + \int\limits_{\Omega}^{}C u v d \Omega  \]

в \eqref{3.0} подставим $ u \rightarrow v, v \rightarrow A_{ik} \frac{\partial u }{\partial x_k }$

\begin{equation}
	\label{3.1}
	\int\limits_{\Omega}^{} v L u d\Omega = \int\limits_{\Omega}^{} \sum_{i,k = 1}^{m}A_{ik} \frac{\partial u }{\partial x_k} \frac{\partial v }{\partial x_i} d\Omega + \int\limits_{\Omega} C u vd\Omega - \int\limits_{S }^{} v \sum_{i,k=1}^{m} A_{ik} \frac{\partial u}{\partial x_k } cos(\overline{n} \cdot x_i ) dS
\end{equation}

\begin{equation}
	\label{3.2}
	\int\limits_{\Omega}^{} u L u d\Omega = \int\limits_{\Omega} \left[ \sum_{i,k = 1}^{m}A_{ik} \frac{\partial u }{\partial x_i} \frac{\partial u }{\partial x_k} + Cu^2 \right] d\Omega - \int\limits_{S } u \sum_{i,k =1}^{m} A_{ik} \frac{\partial u}{\partial x_k } cos(\overline{n} \cdot x_i ) dS
\end{equation}

из \eqref{3.1} вычитаем ее же, но поменяв местами $u$ и $v$: \qquad $ \eqref{3.1} - \eqref{3.1}_{u \rightleftarrows v} $

\begin{equation*}
	\begin{split}
		\int\limits_{\Omega }^{} (vLu - uLv) d\Omega = & \int\limits_{\Omega} \cancelto{0}{\left[ \sum_{i,k = 1}^{m}A_{ik} \frac{\partial u }{\partial x_k} \frac{\partial v }{\partial x_i} - \sum_{i,k = 1}^{m}A_{ik} \frac{\partial u }{\partial x_i} \frac{\partial v }{\partial x_k} \right]} d\Omega \ - \\
		& - \int\limits_{S} \left[ v \sum_{i,k =1}^{m} A_{ik} \frac{\partial u}{\partial x_k } cos(\overline{n} \cdot x_i ) - u \sum_{i,k =1}^{m} A_{ik} \frac{\partial v}{\partial x_k } cos(\overline{n} \cdot x_k )\right] dS
	\end{split}
\end{equation*}

\[ N \boldsymbol{\cdot} := \sum_{i,k=1}^{m } A_{ik } \frac{\partial \boldsymbol{\cdot} }{\partial x_i } cos(\overline{n} \cdot x_i) \]

\begin{equation}
	\label{3.3}
	\int\limits_{\Omega} \left( vLu - uLv \right) d\Omega = \int\limits_{S} \left( uNv - vNu \right) dS
\end{equation}

Частный случай формул Грина, это оператор Лапласа:

\[ Lu = - \Delta u; \ A_{ii} = 1; \ A_{ik} = 0, \ i \neq k; \ C = 0 \]

\begin{equation}
	\label{3.4}
	- \int\limits_{\Omega} v \Delta u d\Omega  = \int\limits_{\Omega}^{} \sum_{i=1}^{m } \frac{\partial u }{\partial x_i }  \frac{\partial v }{\partial x_i} d \Omega - \int\limits_{S }^{} v \frac{\partial u }{\partial n } dS
\end{equation}

\begin{equation}
	\label{3.5}
	- \int\limits_{\Omega} u \Delta u d \Omega = \int\limits_{\Omega}^{} {\left(\frac{\partial u }{\partial x_i}\right)}^2 d \Omega - \int\limits_{S }^{} u \frac{\partial u }{\partial n } dS
\end{equation}

\begin{equation}
	\label{3.6}
	- \int\limits_{\Omega} (v \Delta u - u \Delta v ) d\Omega = \int\limits_{S }^{} \left(v \frac{\partial u }{\partial n } - u \frac{\partial v}{\partial n}\right) dS
\end{equation}

\subsection{Положительные операторы}

Пусть оператор $A$ симметричен в $H$

\textbf{Опр.} Оператор называется положительным, если $ \forall u \in D_A \subset H, \qquad (Au, u) \geq 0 \Leftrightarrow u = 0 $ \\

\textbf{Пр. 1}
\[Bu = -\frac{ d^2 }{d x^2 }u \qquad \textrm{в } L_2 (0,1); \qquad D_B = \{u \in C^2_0 (0,1): u(0) = u(1) = 0\} \]
\[ (B u, v) = - \int\limits_{0}^{1} v  \frac{d^2 u }{d x^2} dx = \int\limits_{0}^{1} \frac{du }{dx} \frac{d v }{d x } dx - v \left.\frac{d u }{dx }\right|^1_0 = - \int\limits_{0}^{1} u \frac{d^2v }{dx^2 } = (u, Bv) \quad \forall u,v \in D_B \]
\[ (Bu, u) = \int\limits_{0}^{1} {\left(\frac{du }{dx }\right)}^2 dx = 0 \]
\[ (Bu, u) = 0 \Rightarrow \frac{du }{dx } = 0 \Rightarrow u = const, u(0) = 0 \Rightarrow u = 0 \]

\textbf{Пр. 2}
\[ Cu = - \frac{ d^2 }{dx^2 }u, \qquad D_C = \left\{ u \in C^2(0,1),
	\begin{cases}
		u'(0)+\alpha u(0)=0 \\
		u'(1)+\beta u(1)=0
	\end{cases}
	\alpha, \beta = const
\right\}
\]
\[(Cu, v) = \int\limits_{0}^{1} \frac{du}{dx} \frac{dv}{dx} dx + \alpha u(0)v(0) + \beta u(1)v(1) = (u, Cv)\]
\[\alpha > 0, \beta \geq 0\]
\[(Cu, u) = \int\limits_{0}^{1} {\left(\frac{du}{dx}\right)}^2 dx + \alpha u^2(0) + \beta u^2(1) \geq 0\]
\[\alpha = \beta = 0, \quad u \equiv 1 \Rightarrow (Cu, u) = 0 \Rightarrow C \text{ не является положительным}\]

\textbf{Пр. 3}
\[ Au = - \Delta u , \qquad D_A = \{ u \in C^2(\Omega): \quad u|_s = 0, \quad \Omega \subset \mathbb{R}^m, S = \partial \Omega, H = L_2(\Omega) \} \]
\[ (Au, u) = (-\Delta u, u ) = - \int\limits_{\Omega }^{} u \Delta u d \Omega = \int\limits_{\Omega }^{} \sum_{i = 1}^{m } {\left(\frac{\partial u }{\partial x_i }\right)}^2 d\Omega - \cancelto{0}{\int\limits_{S }^{} u \frac{\partial u }{\partial n } dS}  \geq 0 \]
\[ \frac{\partial u }{\partial x_i } = const, \ u|_s = 0 \Rightarrow u = 0 \]

\newpage

Рассмотрим мембрану

$\Omega$ в плоскости $(x,y)$, $\ u(x,y)$ --- изгиб мембраны
\[ - \Delta u = \frac{q}{T} \]

$q$ --- поперечная нагрузка на единицу площади

$T$ --- натяжение мембраны

$ {u|}_S = 0 $ --- мембрана закреплена на краях
\[ (A u , u) = (- \Delta u , u) = \iint\limits_{\Omega} \left[ \left(\frac{\partial u }{\partial x }\right)^2 + \left(\frac{ \partial u }{\partial y }\right)^2 \right] dx dy  \]

\subsection{Положительно определенные операторы}

\textbf{Опр.} Симметричный оператор $A$ называется положительно определенным, если
\begin{equation}
	\exists \gamma > 0 : (Au, u) \geq \gamma^2 {\|u\|}^2
\end{equation}

\textbf{Пр. 1 (продолжение)}
\[ B: u(0) = 0, u \in D_B \]
\[ u(x) = \int\limits_{0}^{x } u'(t) dt, \quad x \in [0, 1] \]
\[ u^2(x) \leq \int\limits_{0}^{x} 1^2 dt \cdot \int\limits_{0}^{x} {(u'(t))}^2 dt = x \int\limits_{0}^{x} {(u'(t))}^2 dt \leq x \int\limits_{0}^{1} {(u'(t))}^2 dt \]
\[ \int\limits_{0}^{1} u^2 (x) dx \leq \frac{1}{2} \int\limits_{0}^{1} {(u'(t))}^2 dt \]
\[ \gamma^2 \|u\|^2 \leq (Bu, u) , \quad \gamma = \sqrt{2} \quad \Rightarrow B \text{ является положительно определенным} \]

\textbf{Пр. 4}
\[ Lu = - \frac{d }{ d x } \left(x^3 \frac{du}{dx}\right) \quad \text{в } L_2 (0,1) \]
\[ D_L = \{u \in C^2[0,1], \ u(1) = 0 \} \]
\[(Lu, v) - (u, Lv) = \int\limits_{0}^{1} \frac{d}{dx} \left[ x^3 \left(u \frac{dv }{dx } - v \frac{du }{dx } \right)  \right] dx = \left. \left[ x^3 \left(u \frac{dv }{dx } - v \frac{du }{dx } \right) \right] \right|_0^1 = 0\]
\[ (Lu, u) = \int\limits_{0}^{1} x^3 {\left(\frac{du }{dx }\right)}^2 dx \geq 0 \quad \Rightarrow L \text{ является положительно определенным} \]
\[ \frac{(Lu, u)}{\|u\|^2} \geq \gamma^2, \qquad u_\delta (x) = \begin{cases}
	(\delta - x)^3, & 0 \leq x \leq \delta \\
	0, & \delta \leq x \leq 1
\end{cases}, \qquad u_\delta \in \mathcal{D}_L  \]
\[ \frac{(Lu_\delta , u_\delta)}{{\|u_\delta\|}^2}  = \frac{\int_{0}^{1} x^3 {(\frac{du_\delta}{dx})}^2 dx}{\int_{0}^{\delta} {(\delta -x)}^3 dx} = \frac{9 \int_{0}^{1} x^3 {(\delta -x)}^4 dx}{\int_{0}^{\delta} {(\delta -x)}^6 dx} = \frac{9}{40} \delta \quad \Rightarrow L \text{ не явл. положительно опр.} \]

\newpage

\subsection{Энергетическая норма}

Пусть $A$ --- положительно определен в $H$ (гильберт.)

На $D_A: \quad {[u, v]}_A = {(A u, v)}_H$

Можно показать что выполняются все аксиомы скалярного произведения

\begin{enumerate}
	\item $ {[u, v]}_A = \overline{[v, u]}_A $ \\
		  $ (Au, v) = (u, Av) = \overline{(Av, u)} = \overline{[v, u]} $
	\item $ [a_1 u + a_2 u, v] = a_1[u, v] + a_2[u, v]$
	\item $ (Au, u) = [u, u] \geq \gamma \|u\|^2 \geq 0 $
	\item $ [u, u] = 0 \Leftrightarrow u = 0 $
\end{enumerate}

$ |u| = [u, u] $ --- энергетическая норма

$ D_A $ предгильбертово, дополним его по $ {|\boldsymbol{\cdot}|}_A \ \Rightarrow \ $ гильбертово пр-во $ H_A $

\[ u \in H_A \Leftrightarrow \left[ \begin{array}{l}
	u \in D_A \\
	\exists u : \ \{ u_n \} \in D_A: \ | u_n - u | \underset{n \rightarrow \infty}{\rightarrow} 0
\end{array} \right. \]

\newpage

\section{Лекция 4}

\subsection{Энергетическое пространство}

Пусть $A$ --- положительно определен в $H$ (гильберт.)

На $D_A: \enspace \begin{array}{l}
	{[u, v]}_A = {(A u, v)}_H \\
	{\|u\|}_A = {[u,u]}_A
\end{array} $

$ H_A $ --- энергетическое пространство
\begin{equation}
	{\|u\|}_H \leq \frac{1}{\gamma   } {\|u\|}_A
	\label{4.0}
	\tag{4.0}
\end{equation}
\[ u \in H_A \begin{array}{l@{\ }l}
	\raisebox{-1ex}{$\nearrow$} & u \in D_A \\
	\raisebox{1ex}{$\searrow$} & \exists \{ u_n \} \in D_A: \ \lim\limits_{n \rightarrow \infty} {\|u_n - u\|}_A = 0
\end{array} \]


\textbf{Теор.} $ \qquad \forall u \in H_A \rightarrow $ только один элемент из $H$, причем различные $u_1, u_2 \in H_A$ отвечают различным элементам из $H$

\underline{Док-во.}

\begin{enumerate}
	\item
	$ {u_n}: \lim\limits_{n \rightarrow \infty} {\|u_n - u\|}_A = 0 $ \\
	$ {\|u_n - u_m\|}_A \leq {\|u_n - u\|}_A + {\|u_m - u\|}_A \underset{n,m \rightarrow \infty}{\rightarrow} 0 $ \\
	$ \Rightarrow {\|u_n - u_m\|}_H \rightarrow 0 $ при $ n,m \rightarrow \infty $ \\
	$ \Rightarrow \exists \lim\limits_{n \rightarrow \infty} {\|u_n - u_m\|}_H = 0 $

	\item
	$ u_{1, n} \underset{{\|\cdot\|}_A}{\rightarrow}  u_1, \quad u_{2, n} \underset{{\|\cdot\|}_A}{\rightarrow}  u_2 $ \\
	$ u_1 \ \text{и} \ u_2 \rightarrow u \in H, \qquad u = u_1 - u_2$ \\
	$ \exists \{u_n\} \in H_A \qquad {\|u_n - u\|}_A \rightarrow 0 $ \\
	$ \forall f \in H \qquad |(f, u_n)| \overset{\text{КБ}}{\leq} \|f\|\cdot\|u_n\| \leq {\|f\|}_A\cdot\frac{1}{\gamma}\cdot{\|u_n\|}_A \rightarrow 0 $ \\
	$ \forall \varphi \in D_A \qquad A\varphi = f \in H $ \\
	Тогда $ (A\varphi, u_n) \rightarrow 0 $ \\
	$ {[\varphi, u_n]}_A = (A\varphi, u_n) \rightarrow 0 $ \\
	Переходя к пределу: $ {[\varphi, u]}_A = 0 \ \forall \varphi \qquad \Rightarrow u = 0 \Rightarrow u_1 = u_2 \hfill \square $
\end{enumerate}

\textbf{Пример 1}
\[ Bu = - \frac{d^2}{d x^2}u, \qquad D_{B} = \{u \in C^2 [0,1], \ u(0) = u(1) = 0 \} \]
\[ H = L_2 (0, 1), \ u \in H_B \]
\[ u \in H_B, \quad \exists \{u_n\} \in D_B \quad {\|u_n - u\|}_B \rightarrow 0 \]
\[ {\|u_n - u_k\|}_B \leq {\|u_n - u\|}_B + {\|u_k - u\|}_B \underset{n,k \rightarrow \infty}{\rightarrow} 0 \]
\[ {\|u_n - u_k\|}_B^2 = \int\limits_{0}^{1} {\left(\frac{du_n}{dx} - \frac{du_k}{dx} \right)}^2 dx \underset{n,k \rightarrow \infty}{\rightarrow} 0 \]
\[ \Rightarrow \{\frac{du_n}{dx}\} \ \text{фундаментальна в} \ L_2(0,1) \Rightarrow \exists v(x) \in H\]
\[ u_n(x) = u_n(0) + \int\limits_{0}^{x} u_n^{'}(t) dt, \quad u_n \in D_B, \qquad \ \text{при} \ x=0:\ u_n(0) = 0 \]
\[ \text{Переходя к пределу:} \quad u(x) = \int\limits_{0}^{x} v(t) dt \quad \text{и} \quad u(0)=0\]
\[ u(1) = \int\limits_{0}^{1} v(t) dt = \lim\limits_{n \rightarrow \infty} \int\limits_{0}^{1} u_n^{'}(t) dt = \lim\limits_{n \rightarrow \infty} \left( u_n(1) - u_n(0) \right) = 0 \]
\[ \text{Следовательно,} \ u \ \text{абсолютно непрерывная на} [0,1], \text{удовлетворяет граничным условиям}\]
\[ u' \in L_2(0,1) \]

\textbf{Пример 2}
\[ Cu = - \frac{d^2}{dx^2} u(x); \qquad u'(0)+\alpha u(0)=0, \ u'(1) + \beta u(1)=0 \]
\[ \exists \{u_n\} \in D_C, \quad \alpha > 0, \ \beta \geq 0 \]
\[ \int\limits_{0}^{1} {\left( \frac{du_n}{dx} - \frac{du_k}{dx} \right)}^2 dx \underset{n, k \rightarrow \infty}{\rightarrow} 0 \]
\[ |u_n(0) - u_k(0)| \underset{n, k \rightarrow \infty}{\rightarrow} 0 \]
\[ u_n(x) = u_n(0) + \int\limits_{0}^{x} u_n'(t) dt \]
\[ u(x) = c_0 + \int\limits_{0}^{x} v(t) dt \]

\textbf{Теор.} Пусть оператор $A$ положительный, но не положительно определенный. Тогда
\[ u \in H_A: \quad u \in H \Leftrightarrow \exists \{ u_n \} \in D_A \]
\[ {\| u = u_n \|}_A \underset{n \rightarrow \infty}{\rightarrow} 0 \quad \text{и} \quad {\| u_k - u_n \|}_H \underset{n,k \rightarrow \infty}{\rightarrow} 0 \]

\textbf{Пример 3}

\subsection{Энергетический метод}
(для положительно определенных операторов)

\[ Au = f \label{eq:1} \]

\[ A: \mathcal{D} (A) \in H \rightarrow H; \]

Теорема

A положителен в H уравнении \ref{eq:1} $ \exists $ не более одного решения.

Доказательство

\[ u_1, u_2 - \textrm{ Решения \ref{eq:1}} ... \]

Теорема о функциональной энергии

A - положительный в H; u - решение \ref{eq:1} $ \rightleftarrows $ доставляет минимум функционала

\[ F(u) = (Au, u)_H - (f, u)_H - (u, f)_H \label{eq:2}\]

Доказательство

...

\[  \]

Пример 4
\[ \Delta^2 \omega = \frac{\partial^4}{\partial x^4}\omega + 2 \frac{\partial^4 \omega }{\partial^2 \omega \partial^2 y} + \frac{\partial^4 \omega }{\partial y ^4} \]
\[ \mathcal{D}_A = \{ \omega \in c^4( \overline{\Omega}); \omega|_S = 0; \frac{\partial \omega}{\partial n} |_S = 0 \} \]

\[ A-\omega = \frac{a(x,y)}{\mathcal{D}} \]

\subsection{Обобщение решения задачи о min для ф.э.}

\[ A - \textrm{ Поллжительно определено в H } Au=f \ref{eq:1} f \in H \]
фикс $ f \in H \forall u \in H_A (u, f)_H :\textrm{ ф-ла } : H_A \rightarrow \mathcal{R} $

\[ | (u, f)_H | \leq ||f||_H {||u||}_H \leq {||f||}_H \frac{1}{\gamma} ||u||_A; \gamma ||f||_H - const \]

\[ \textrm{ Опр } (f, u) \Rightarrow \textrm{ по Т Рисса } \exists u_0\in H_A (f, u)_H = [u, u_0]_A \]

\[ F(u) = [u, u]_A - [u, u_0] - [u_0, u]_A \]

\[ +-[u_0, u_0]_A \label{eq:4}\]

\[ F(u) = || u - u_0 ||^2_A - ||u_0||^2_A  \]

\[ argmin_{u \in H_A} F(u) = u_0 \textrm{ Обощенное решение } Au = f \]

Если H сепарабельно, энергетическое про-во тоже сепарабельно, $ \exists \{ \omega_n \} $ ПОНС

\[ r_0 = \sum_{n=1}^{\infty} [u_0, \omega_n] \omega_n  \ref{eq:5}\]

\[ u = \omega_n [u_0 \omega_n]_A = (f, \omega_n)_H \]

\[ u_0 = \sum_{n=1}^{\infty} (f, \omega_n)_H \omega_n \]

\section{Лекция }

\subsection{ Применение энергетического метода для краевых задач}

\begin{enumerate}
	\item Немного опазадал, пример часть примера пропустил
	...
	\[ {(Lu, u)}_H = \sum_{k=0}^{m} \int_{x_1}^{x_2} p_k (x) (\frac{d^k u }{d x^k })^2 dx >= \int_{x_1}^{x^2} p_{n_1} (x) (\frac{d^m u }{d x^n } dx^3) >= p_0 \int_{x_1}^{x^2} (\frac{d^m u }{dx^m } dx = p_0 {||u_(m)||}^2_H) \]
	...
	\[ (L u_M ) >= \partial^2 {||u||}^2_H, \gamma = \sqrt{p_0} {(\frac{\sqrt{2}}{x_2-x_1})}^m \]
	...
	\[ {||u||}_A \leq \sqrt{p_0 } {||u||^{(m)}}_H \exists \{ u_N (x) \}\]

	\[ {lim}_{n \rightarrow \infty} = 0; u_0 - \textrm{ точное решение  } \]

	\[ {||u_n - u_k ||}_A <= {||u_n - u_0||}_A + {|| u_k - u_0||}_A \rightarrow 0 \]
	\[ u^{(l)}_n (x_1) = u_k^(l) (x_1) = 0, l = \overline{0, m-1} \]
	...

	\item Изгиб балки
	\[ L_{\omega} = \frac{d^2}{dx^2} [E I(x) \frac{d^2 \omega }{dx^2}] + K\omega = q(x)\]
	\[ \omega - \textrm{ Прогиб балки } \]
	\[ E - \textrm{ модуль Юнга } \]
	\[ I(x) - \textrm{ момент инерции } \]
	\[ q(x) - \textrm{ интенсивность нагрузки на балку } \]
	\[ K - \textrm{ коэф податливости основания } \]

	\[ \omega (0) = \omega (l) = 0 \]
	\[ \omega'(0) = \omega'(l) = 0 ; A - \textrm{ Положительно определен } \]
	Аналогично задачи минимизации функционала

	\[ F(\omega) = \int_{0}^{l} (E I(x) {\omega''}^2 + K\omega^2 - 2q(x \omega)) dx = (L\omega , \omega) -2 (\omega, q)\]

	Воспользуемся методом Рица

	\[ {u_n (x)}^{\infty}_{n=1}, \phi_n (x) = {(x-l)}^2 x^{n+1}, \textrm{ Полная система в  } H_A \]

	\[ \omega_n = \sum_{k=1}^{n} a_k \phi_k (x) = {(x-l)}^2 \sum_{k=1}^{n} a_k x^{k+1} \]
	\[ \sum_{n}^{k=1} a_k A_{1k} = b_{ij}; i=\overline{1, n} \]
	\[ b_j = (q, \phi_j)_H = \int_{0}^{l } a(x) (x-l) x dx \]
	\[ A_{ik} = (L \phi_i, \phi_k)_H = \int_{0}^{l} (E I(x) \frac{d^2 \phi_i}{d x^2} \frac{d^2 \phi_k}{d x^2} + k\phi_i \phi_k) dx \]
	...
	\[ \omega(0) = 0; \omega''(l) = 0 \]
	\[ \omega' (0) = 0 \]
	\[ \frac{d }{d x } (E I(x) \frac{d^2 \omega }{d x^2})^{x=l}_{x=0} = 0 \]
	Тут тоже можно доказать полажительную определенность

	\item Краевая задача для систем ОДУ
	\[ - \sum_{k=1}^{s} [ \frac{d }{dx }(p_{jk}(x) \frac{du_k (x)}{dx}) - q_{jk}(x) u_k (x)] = f_j(x) \]
	краевые ...

	\[ - \frac{d }{dx } [P(x) \frac{du }{dx }] + Q(x) u(x) = f(x) \]
	\[ u(x_1) = u (x_2) = 0 \]
	\[ (u, v)_{H= L_2(x_1, x_2)} = \int_{x_1}^{x_2} u (x) \cdot v(x) dx = \int_{x_1}^{x_2} \sum_{k=1}^{s} u_k(x) v_k (x) dx \]

	\subsubsection{Теорема}
	\[ P(x), Q(x) \textrm{ симметр. } x \in [x_1, x_2] \Rightarrow A \textrm{ Симметричный } \]
	Доказательство
	\[ {(Au, v)}_H = - \int_{x_1}^{x^2} v(x) \cdot \frac{d}{dx} [P(x) \frac{du }{dx }] dx + \int_{x_1}^{x_2} v(x) \cdot Q(x u(x dx)) = \]
	\[ = \int_{x_1}^{x_2} P \frac{du }{dx } \cdot \frac{dv }{dx } + v(x \cdot Q(x) u(x))dx \]
	\[ Q u \cdot v = \sum_{j,k=1}^{s } q_{jk u_k \cdot v_j} =  \]
	\[ = \sum_{i,j =1}^{s} q_{k,j} v_j \cdot v_k \]
	Следовательно оператор симметричен

	\subsubsection{Теорема}
	\[ P(x), Q(x) \textrm{ симметрич на } [x_1, x_2] \]
	\[ P(x)  \textrm{ положит. опр. } Q(x)\textrm{  неотр на } (x_1, x_2] \Rightarrow A \textrm{ положительно определен } \]
	доказательство
	\[ P(x) \textrm{ пол. опр } \forall x \Rightarrow \textrm{ пусть } \lambda_1(x) >0  \]
	\[ \exists \lambda > 0 = const; \lambda_1(x) > \hat{\lambda}> 0  x \in [x_1, x_2] \]

	\[ \forall t = (t, ... , s ) \]
	\[ P(x) t \cdot t = \sum_{j, k=1}^{s } P_{jk}(x)t_j t_k \geq  \lambda_1(x) \sum_{k=1}^{s } t_k^2 \geq \]
	\[ \geq \hat{\lambda } \sum_{k=1}^{s } t^2_k \]
	\[ Q(x) t\cdot t = \sum_{k, k=1}^{s }q_{jk} t_j t_k \geq 0 \]

	\[ (u, u)_H = \int_{x_1}^{x_2} (P \frac{du }{dx } \cdot \frac{du }{dx}) dx \geq \hat{\lambda } \int_{k=1}^{s} (\frac{du_k}{idx}^2) dx \]
	\[ (Au, u)_H \geq \frac{2 \hat{\lambda}}{{(x_2 - x_1)}^2} \int_{x_1}^{x_2} (\sum_{k=1}^{s } u_k^2) dx = \frac{2 \hat{\lambda}}{{(x_2 - x_1)}^2} dx = \frac{\hat{\lambda}}{{(x_2-x_1)}^2} {||u||}^2_H \]

	\[ {(Au, u)}_H \geq \gamma^2 {||u||}^2_H \]
	...


\end{enumerate}

\subsection{Основные кр задачи для ур-я Пуассона}

\[ -\Delta u = f(p) \textrm{ в } \Omega \in \mathcal{R}^m \]\label{eq:5_1}

з. Дирихле

\[ u|_{\partial \Omega} = 0 \]\label{eq:5_2}
\[ A u = - \Delta u = \sum_{k=1}^{m } \frac{\partial^2 u }{\partial x_k^2} \]
\[ P_A = \{ u \in c^2 (\overline{\Omega}_1) u |_{2 \Omega} = 0 \} \]
\[ H = L_2 (\Omega) \]
\[ (- \Delta, u )_h = \int_{\Omega}^{} \sum_{i=1}^{n} (\frac{\partial n }{\partial x_i})^2 d \Omega - \int_{ \partial \Omega}^{} u \frac{\partial u }{\partial n } dS = \int_{\Omega}^{} {({grub} u)}^2 d\Omega \geq 0 \]\label{eq:5_3}

\[  \rightleftarrows \]
\[ F(u) = (-\Delta u, u )_H -2(u, f)_H  \]\label{eq:5_4}
\[ F(u) = \int_{\Omega }^{} ((grad u )^2 - 2uf) d\Omega  \]\label{eq:5_5}

\[ [\frac{\partial u }{\partial n} + \gamma (P) u ] |_{\partial \Omega} = 0 \]\label{eq:5_6}

\[ (- \Delta u, u)_H = \int_{\Gamma}^{} (grad u)^2 d\Omega + \int_{\partial u^2}^{} dS \geq 0 \]

\[ (-\Delta u, u)_H =0 \Rightarrow u = condt \int_{ \partial \xi}^{} \gamma c^2 dS = 0 \Rightarrow c = 0 \Rightarrow u= 0\]

\[ F(u) = \int_{\Omega}^{} ((grad u)^2 - 2uf) d\Omega + \int_{\gamma \Omega} \gamma n^2 dS \]\label{eq:5_7}

\[ \frac{\partial u }{\partial n}|_{\partial \Omega } = 0 \]\label{eq:5_8}

з Неймана

\ref{eq:5_1}, \ref{eq:5_8}

\[ (- \Delta, u )_H = - \int_{\partial \Omega}^{} u  \frac{\partial u}{\partial n} dS + \int_{\Omega}^{} (grad u )^2 d\Omega \geq \]

\[ u == 1 (- \Delta u, u)_H = 0 \]

\[ \textrm{ при  } V == 1 \]
\[ \int_{\Omega }^{} \Delta u d\Omega = \int_{\partial \overline{\Omega}}^{} \frac{\partial u }{\partial n } dS = 0 \]

\[ S_{\Omega} f d \Omega = 0\]
Условие разрешимости \ref{eq:5_1} \ref{eq:5_8}

\section{Лекция 6}

**пропустил начало (почти треть) **

Уравнение Фридрехса в общем виде:
\[ \int_{\omega}^{} \sum_{k=1}^{m} (\frac{\partial  n}{\partial x_k})^2 d\Omega \geq x^2 \int_{\Omega}^{} u^2 dx \]
\[ u|_S=0 \]

\[ \int_{\Omega}^{} u^2 d \Omega = \leq c \{ \int_{\Omega}^{}(\frac{\partial u }{\partial x}^2+\frac{\partial n}{\partial y} d\Omega) + \int_{\Omega}^{} u^2 dS \} \]

\[ (\frac{\partial(fv)}{\partial x})^2 + (\frac{\partial (fv)}{\partial y})^2 = f^2 [(\frac{\partial v}{\partial x} + (\frac{\partial v}{\partial y})^2)] - v f\Delta f + \frac{\partial}{\partial x}(v^2 f \frac{\partial f }{\partial x}) \frac{\partial}{\partial y} (v^2 f \frac{\partial f}{\partial y}) \]

Преобразуем правую и левую части

\[ v^2 ((\frac{\partial f}{\partial x})^2+(\frac{\partial f}{\partial y})^2) + f^2 ((\frac{\partial v}{\partial x})^2 + (\frac{\partial v}{\partial y})^2) + 23\frac{\partial v}{\partial x} f \frac{\partial f}{\partial x} + 2v\frac{\partial v }{\partial y} f \frac{\partial f}{\partial y} \]

\[ v^2 (\frac{\partial  f}{\partial x})^2 + v^2 (\frac{\partial  f}{\partial y})^2 + 2v\frac{\partial v}{\partial x}f\frac{\partial  f}{\partial x} + v^2 f \frac{\partial^2 f}{\partial y^2} _ v^2 \frac{\partial^2 f}{\partial y^2} + 2v \frac{\partial v}{\partial y} + \frac{\partial f}{\partial y} - v^2 f \Delta f + f[(\frac{\partial v}{\partial  x})^2+ (\frac{\partial v}{\partial x})^2] \]

Это предполагается очевидным XD
\[ \int_{}^{} ((\frac{\partial (fv)}{\partial x})^2 + (\frac{\partial (fv)}{\partial y})^2) d\Omega \geq + \int_{\Omega}^{}vf\Delta f d \Omega + \int_{\partial \Omega}^{}v^2 f \frac{\partial f}{\partial  n} dS \]
\[ -\int_{\Omega}^{} vf \Delta f d\Omega \leq \int_{\Omega}^{} ((\frac{\partial u}{\partial x})^2 + (\frac{\partial u}{\partial y})^2) d \Omega + \int_{\partial \Omega}^{} v^2 f \frac{\partial f}{\partial n}d S \]

\[ f = \sin (\frac{\pi x}{a})\cdot \sin (\frac{\pi y }{b}) \]
\[ \Delta f = - \pi^2 (\frac{1}{a^2} + \frac{1}{b^2})\cdot f- \int_{\Omega}^{}v^2 f \Delta u^2 = \int_{\Omega}^{} u^2 s\Omega \pi^2 ()  \]

\[ |\int_{\partial u}^{}v^2 f \frac{\partial f}{\partial n} dS| \leq \int_{\partial  \Omega}^{} v^2 f |\frac{\partial f}{\partial n}|dS \leq c_1 \int_{\partial \Omega}^{} u^2 d \Omega \]

\[ \pi^2 (\frac{1}{a^2} + \frac{1}{b^2}) \int_{\Omega}^{} y^2 d\Omega \leq ((\frac{\partial u}{\partial x})^2 +(\frac{\partial u}{\partial x})^2) dx + c_1 \int_{\partial \Omega}^{} v^2 dS \]
\[ c= min \{\frac{c_1}{\pi^2(\frac{1}{a^2}+\frac{1}{b^2})}; \frac{1}{\pi}(\frac{1}{a^2}+\frac{1}{b^2})\} \]

\[ (-\Delta u, u)_H = \int_{\Omega}^{} (grad u)^2 d\Omega - \int_{\partial \Omega}^{}u \frac{\partial u}{\partial n} dS \geq \int_{\Omega}^{} (grad u)^2 d\Omega + \sigma \int_{\partial  \Omega u^2 dS}^{} \geq \sigma \{ (grad u)^2 d\Omega + \int_{\partial \Omega}^{}u^2 dS \} \]

\[ \frac{1}{c} {||u||}_H^2 \leq \int_{\Omega}^{} (grad u)^2 d\Omega + \int_{\partial \Omega}^{}u^2 dS \]

\[ \frac{\sigma_1}{c} {||u||}_H^2 \leq \int_{\Omega}^{} (grad u)^2 d\Omega + \int_{\partial \Omega}^{}u^2 dS \]
\[ \sigma = \sqrt{\frac{\sigma}{c}} \]

\section{Лекция 7}

\[ -(\Delta u, u)_H \geq \frac{\sigma_1}{c} {||u||}_H \]

\[ \Delta u = f \textrm{ в } \Omega \]
\[ \frac{\partial u}{\partial  n}|_{\Omega} = 0 \]

\subsection{Не-во Пуанкаре}

\[ (x_1, y_1), (x_2, x_2) \in \Omega \]

\[ \Omega u^2 d \Omega \in A \int_{\Omega}^{} (grad u)^2 d\Omega + B(\int_{\Omega}^{}u d\Omega)^2\]

\[ \int_{\Omega}^{} u^2 d\Omega \leq (\frac{\partial  u }{\partial x})^2 + (\frac{\partial  u }{\partial y})^2 + B(\int_{\Omega}^{} d\Omega)^2 \]

\[ u^2 (x_2, y_2) + u^2(x_1, y_1) - 2u(x_2, y_2) u(x_1, u_1) = (\int_{x_1}^{x_2} \frac{\partial n}{\partial x} (x, y_1) dx)^2 + (\int_{y_1}^{y_2} \frac{\partial n}{\partial y}(x_2, y) dy)^2 + 2\int_{x_1}^{x_2} \frac{\partial n}{\partial x} (x_1, y_1)dx \int_{y_1}^{y_2} \frac{\partial u}{\partial y} (x, y) \leq 2 \{ |x_2-x_1| \int_{0}^{a} (\frac{\partial n}{\partial x}(x_1, y_1)^2) dx + b \int_{0}^{b} (\frac{\partial u}{\partial y} (x_2, y))^2 dy \} \]

\[ \iiiint u^2 (x_2, y_2) dx_1 dy_1 dx_2 dy_2 = ab \int_{\Omega}^{} u^2 d\Omega \]

\[ ab \int_{\Omega}^{} u^2 d\Omega \]


\[ \iiiint u(x_2, y_2) u(x_1, y_1) dx_1 dy_1 dx_2 dy_2 = (\int_{\Omega}^{} u d \Omega) \]

\[  \int_0^a \int_{0}^{b} \int_{0}^{a} \int_{0}^{b} a \int_{0}^{a}(\frac{\partial u}{\partial x} (x_1, y_1)^2)dx dx_1 dy_1 dx_2 dy_2 = \]
\[ = a^2 b ...\int_{}^{}(\frac{\partial u}{\partial x}(x, y))^2 d\Omega \]

\[ 2 ab \int_{\Omega}^{} u^2 d\Omega -2 (\int_{\Omega}^{} u d\Omega)^2 \leq 2ab \{a^2 \int_{\Omega}^{}(\frac{\partial u}{\partial x} d\Omega + d \int_{\Omega}^{} (\frac{\partial u}{\partial y})^2 d\Omega) \} \]

\[ A = max \{a^2, b^2\}, B = \frac{1}{ab} : ab \]

\[ D_N = D(A_N) = \{ u \in C^2(\overline{\Omega}), \frac{\partial u}{\partial n} |_{\partial  \Omega} = 0; \int_{\Omega}^{} u d\Omega = 0 \} \]

\[ {||u||}^2_H \leq A \int_{\Omega}^{} \sum_{k=1}^{m} (\frac{\partial u}{\partial x_k})^2 d\Omega = \overline{ A} (Lu, u)_H \omega = \frac{2}{\sqrt{ \overline{ A}}} (A_Nu, u) \geq \omega^2 {||u||}^2_H\]

Даже $ A: A_n $ или $ A_D $ $ [u, V]_A = \int_{\Omega}^{} grad u \cdot frad V d\Omega, ||u||_A = \int_{\Omega}^{} (grad u )^2 d \Omega $

\[ u, V \in L_2 (\Omega) ; \psi \in C^{\infty}_0 (\overline{ \Omega}) \]
\[ \textrm{ Если } \forall \psi \in c^{\infty}_0 \int_{\Omega}^{} u \frac{\partial \psi}{\partial x} d\Omega V \psi d \Omega \]

\[ \textrm{ Пусть  } u \in H_{A_D} \exists \{u_N\}\in D_{A_{D}} \]

\[ {||u_n - u||}_H \rightarrow_{n\rightarrow \infty} 0 \]

\[ {||u_n - u||}_A \rightarrow_{n \rightarrow \infty} 0 \]

\[ \int_{{(grad u_n = grad u_0)}^2}^2 d\Omega = \sum_{k=1}^{m} \int_{}(\frac{\partial  u_n}{\partial  x_k} - \frac{u_l}{\partial x_k})^2 d \Omega \rightarrow 0 \]

\[ || \frac{\partial u_n}{\partial x_k} - V||_H \rightarrow 0 \textrm{ покажем, что  } \Omega \rightarrow 0 \]
\[ \textrm{ Пусть } \psi \in c^{\infty}_{0} ( \overline{\Omega}) \]
\[ \int_{\Omega}^{} u_n \frac{\partial \psi}{\partial x_k} d\Omega = - \int_{\Omega}^{} \psi \frac{\partial u_n}{\partial x_j} d\Omega \]

\[ (u_n, \frac{\psi}{\partial  x_k})_H = (\frac{\partial  u_n}{\partial x_k; \psi})_H \rightarrow (u, \frac{\partial  \psi}{\partial  x_k})_H; = -(\psi, V)_H \]

\subsection{Неоднородные краевые условия}

\[ \Delta u = 0 \label{7_1} \Omega \in \mathcal{R^m}\]

\[ u|_{\partial \Omega} = \psi \label{7_2} \]

\[ \textrm{ Пусть  }\exists \psi(P); \psi \in c( \overline{ \Omega}), \label{7_*}\]
\[ \psi (P) = \phi(P) \]
\[ \frac{\partial  \psi}{\partial  x_k} \in C(\Omega), k=1 \overline{1,m} \]

\[ \Phi(u) = \int_{\Omega}^{} (grad u)^2 d\Omega \label{7_3}\]
\[ D_{\Phi} = \{u: ref{7_*} \} \Phi (P) + \eta(P), \]
\[ \eta: \ref{7_*} + \eta|_{\partial  \Omega } = 0 \label{7_4} \]

Пусть ф-ии $ u_0(P) $ достигает $ min \Phi(u) : u_0(P) $ реш. $ \ref{7_1}, \ref{7_2} $

** 1-3 и еще на гарнице ноль $ \label{7_**}$

\[ u_0 + t \eta \in D_{\Phi}, \forall t \in \mathcal{R}, \eta : \ref{7_**} \]

\[ \Phi(u_0 + th)\textrm{ достигает min при  } t=0 \textrm{ как скал функция t } \]

\[ \frac{d}{dt} \{ \Phi(u_0+t \eta) \} |_{t=0} = \{ \frac{d}{dt} \int_{\Omega}^{} \sum_{k=1}^{m} (\frac{\partial (u_0 + t\eta)}{\partial x_k})^2 d\Omega \} |_{t=0} = \{ \frac{d}{dt} \int_{\Omega}^{} \sum_{k=1}^{m} [(\frac{\partial u_0}{\partial  x_k})^2 + 2(\frac{\partial u_0}{\partial  x_k})^2 + 2\frac{\partial u_0}{\partial x_k} \cdot t \frac{\partial u}{\partial x_k} + t^2 (\frac{\partial \eta}{\partial x_k})^2] d\Omega \} | = 0\]

...

\[ \Rightarrow \int_{\Omega}^{} \eta \Delta u_0 d\Omega = 0 \Rightarrow \]
\[ \eta \ref{7_**}\textrm{  плотность в } L_2(\Omega) = H \]
\[ \Rightarrow \Delta u_0 = 0 \]

\[ \Phi (u) = \int_{\Omega}^{}(grad u)^2 d \Omega \]
\[ \psi: \ref{7_*} u=\psi-V \]
\[ \Phi(u) = \Phi (\psi - V) = \int_{\Omega}^{} (grad (u=V))^2 d\Omega = \]
\[ = \int_{\Omega}^{} (grad \psi)^2 d\Omega - 2 \int_{}^{} grad V \cdot grad \psi d \Omega + \int_{\Omega}^{} (grad V) d \Omega \]

\[ F(V) = {||V||}^2_{A_D} - 2 \int_{}^{} grad V grad \psi d\Omega; V \in H_D = H_{A_D}\]

\[ l V = \int_{\Omega}^{} grad \psi grad v d \Omega; |lV| \leq \int_{\Omega}^{} (grad \psi)^2 d\Omega \int_{\Omega}^{} (grad V)^2 d\Omega = c ||V||_{H_{A_D}} \Rightarrow l - \textrm{ ограничение }\]

\[ I \forall \textrm{ ограничение } \Omega\]
\[ \psi \in H' (\Omega): \exists ! \textrm{ Обобщен. реш Дирихле} \]
\[ u \in H (\Omega) \]

\subsection{ур-е с переменным коэф}

\[ Lu = - \sum_{j,k=1}^{m} \frac{\partial }{\partial x} (A_jk(P) \frac{\partial u}{\partial x_k}) + c(P) u; Lu=f \Omega \in \mathcal{R^m} \label{7_2_1} \]
\[ u|_{\partial \Omega} = 0 \label{7_2_2} \]

\[ (N [u] + \partial (P) u ) |_{\partial \Omega} = 0 \label{7_2_3} \]

\[ N(u) |\partial_{\partial \Omega} = u \]

Формула Грина
\[ \int_{\Omega}^{}(V Lu - uLV) d\Omega = - \int_{\partial \Omega}^{}(V N(u) - u N(V)) dS \label{7_2_5}\]

При условиях \ref{7_2_2} и \ref{7_2_4} интеграл сокращается к 0, поэтому останется только \ref{7_2_3}.
\[ N(u) + \sigma u = 0 \]
\[ N(v) + \sigma v = 0 \]
\[ V N(u) + \sigma uV - b N(V) - \partial u V = 0 \]

на $\partial \Omega V N(u) - uN(V) = 0$

\[ \Rightarrow \textrm{ гр. у. \ref{7_2_2}, \ref{7_2_3}, \ref{7_2_4}} \]

Опр L элементт в $ \overline{\Omega},\textrm{  если } A_{jk}(P): $
\[ \exists_{\mu_0} = const > 0 \forall t1, ... t_m \in \mathcal{R}; \forall P \in \overline{\Omega} \]

\[ \sum_{j,k=0}^{m} A_jk(P) t_j t_k \geq \mu_0 \sum_{j,k=1}^{m} t_k^2 \]
Пример оператор Триколи

\[ Ly = y \frac{\partial^2 u}{\partial  x^2} + \frac{\partial^2 u}{\partial y^2} \]
\[ A_H = y, A_{22} = 1 A_{21} = A_{12} = 0 \]

\[ y t_1^2 + 1\cdot t^2_2 \geq B t^2_1 + t^2_2 \geq \hat{B}(t^2_1+t^2_2) \]

\[ \forall \Omega : \overline{\Omega} \in \mathcal{R} x(x, +\infty) L \textrm{ элиптич в} \Omega \]

\[ L \textrm{ эллептический в } \overline{\Omega} \]

\[ C(P) \geq 0 \textrm{ ф-ла Грина } \]

\[ (Lu, u)_H = \int_{}^{}u Lu d\Omega = \int_{\Omega}^{} (\sum_{j,k=1}^{m} A_j k \frac{\partial u}{\partial  x_j}\frac{\partial u}{\partial x_k} + C u^2) d\Omega - \int_{\partial}^{} u N (u) dS  \]

\[ \ref{7_2_2}/\ref{7_2_4} \textrm{ Дир Лейли } \int_{\partial \Omega}^{}(\cdot) dS = 0 \gamma = \sqrt{c_0} \]

\[ (Lu,u)_H \geq a_1^2 d\Omega \geq c_0 \int_{\Omega}^{} u^2 d\Omega =\]
\[ = \Omega^2 ||u||^2_H \]

$ \ref{7_2_3} $ Смен кр з $ N(u) = -\sigma u $ на $ \partial  \Omega \sigma (P) \geq \sigma_0 > 0$
\[ (Lu, u)_H \geq \sigma \int_{ \partial  \Omega}^{} u^2 dS \geq c_0 ||u||^2_H \]

\section{Лекция 8}

\[ Lu = - \sum_{j,k=1}^{m} \frac{\partial }{\partial  x_j} (J_{jk}(P)\frac{\partial n}{\partial x_k} + c(P)u = f(P)) \label{8_1_1}\]

\begin{enumerate}
	\item з Дирихле
	\[ (Lu, u) = \mu \sum_{j=1}^{m} \int_{\Sigma}^{}(\frac{\partial n}{\partial x_j})^2 d\Omega \geq \Sigma^2 {||u||}^2_H; \sigma = \sqrt{\\ae \mu_0}\]
	\item з Робэна $\label{8_3}$
	\[ (Lu, u) \geq (\alpha {(\int_{\Sigma}^{} \sum_{j=1})}^{n}(\frac{\partial u}{\partial x_i})^2) d\Omega + \int_{\partial \Omega}^{}n^2 dS) ; \Rightarrow (Lu, u) \geq \alpha {||u||}^2 \]
	\item з Неймана
	\[ C(P) = 0 \]
	\[ Lu = -\sum_{j,j =1}^{m}\frac{\partial}{\partial  x_j}(A_{jk} \frac{\partial n}{\partial x_k}l) = f(P) \]
	\[ \int_{\Omega}^{}(\cdot) d \Omega + \textrm{ ф-ла Остроградского } \]
	\[ -\int_{S}^{} \sum_{j,k = 1}^{m} A_j \frac{\partial n}{\partial x_k} cos( \overline{n}, x) dS = 0 \]
	\[ \Rightarrow \int_{\Omega}^{}f(P) d \Omega = 0 \]
	\[ \int_{\Omega}^{} u d\Omega = 0 \]
	\[ D_{L_N} = \{ u \in C^2 ( \overline{\Omega}), N(u)|_{\partial \Omega} = 0, \int_{\Omega}^{}u d \Omega = 0 \} \]
	\[ (L_N u, n) = -\int_{\Omega}^{} u \sum_{j,k=1}^{m} \frac{\partial }{\partial x_j}(A_{jk} \frac{\partial n}{\partial x}) d\Omega = \]
	\[ = \int_{\Omega}^{} \sum_{j,k=1}^{m} A_{jk} \frac{\partial n}{\partial x_j} \frac{\partial n}{\partial  x_n} d \Omega \geq \mu_0 \int_{k=1}^{m} (\frac{\partial n}{\partial x_k})^2 d\Omega \]
	\[ \int_{\Omega}^{} u^2 d\Omega \leq A \int_{\Omega}^{} \sum_{k = 1}^{m} (\frac{\partial  n}{\partial x_0}dS + B (\int_{\Omega}^{} u \Omega)^2)\]

\end{enumerate}

\subsection{Энергетический метод для пложительных операторов}

\[ Au = f \label(8_2_1), u, f \in H, A \textrm{ положительный } \]
Все еще работает теорема о функциональной энергии

\[ F(u) = (Au, u)_H - 2(u, f)_H \label{8_2_2}\]
Энергетическое пространство попрожденное опреатором $ H_A $, вообще говоря его элементам нельзя соспоставить элементы из Гильбертова.
\[ H_A - \textrm{  Энергетическое пр-во } \]

\[ (u, f) \textrm{ на  } D_A - \textrm{ плотно в H и в } H_A  \]
\[ (u,f) = lu \textrm{ Функционал } \rightarrow \textrm{ может быть ограничен или не ограницен } \]
Если ограничен в $H_A$ прододжим на $H_A$

в $ H_A $ по теореме Рисса $ \exists u_0 \in H_A (u, f) = [u, u_0]A $

\[ [u - u_0, u-u_0] = {||u||}^2_A + {||u_0||}^2_A - 2{[u, u_0]}_A \]
\[ F(u) = {|| u ||}^2_A - 2[u, u_0]_A = {||u-u_0||}^2_A - {||u_0||^2_A} \label{8_2_4} \]

Минимум достигается на элементе $F(u) = u_0$. Но $ u_0 $ может не лежать в энергетическом про-ве. Обобщенное решение с конечной энергией.

Если H сепарабильно $ \Rightarrow H_A $ сепарабильно $ \Rightarrow \{ \phi_n \} $ в $ H_A $

\[ u_o = \sum_{n=1}^{\infty}{[u_0, \phi]}_A \phi\]
\[ [\phi_n, u_0] = l\phi_n \]
\[\textrm{ Если } \{\phi \} \in D_A \Rightarrow l \phi_n - (f, \phi_n) \Rightarrow u_o = \sum_{n=1}^{\infty} (f, \phi_n) \phi_n  \]

\[ u_k \sum_{n=1}^{k} (f, \phi_n) ||u_k - u_0||_{A} \rightarrow_{k \rightarrow \infty} 0 \]

\[\textrm{ Если } \{\phi \} \in D_A \Rightarrow l \phi_n - (f, \phi_n) \Rightarrow u_o = \sum_{n=1}^{\infty} (f, \phi_n) \phi_n \Rightarrow = \sum_{n=1}^{\infty} (f, \phi_n) \phi_n \]

\subsection{Эллиптические уравнения в бесконечной области}

\[ \Omega = \infty \textrm{ обл }; \partial \Omega \]
\[ -\sum_{j,k=1}^{m} \frac{\partial }{\partial x_j}(A_{kj}^{(P)}\frac{\partial u}{\partial x_k}) = f(P) \]

з Дирихле
\[ U|_{\partial \Omega} = 0; A_{jk} \textrm{ отр } A\]

\[ D_A = \{u \in C^2(\overline{\Omega}), u|_{\partial \Omega} = 0, b(P = 0), |P| >> 1 \} \]

\[ A - \textrm{ положительно определен } \]

\[ HuD \exists u_0 \textrm{ об реш с кон энергией } \]
\[ \exists g(P): f(P = div g(P)) \]
\[ \int_{\Omega}^{} {|g(P)|}^2 d\Omega < \infty \]
\[ h(P)l \textrm{ обобщенная  } \div g(P). \textrm{ Если  } \forall \phi (P) \in c_0^{\infty} (\Omega) \]

\[ {||u_0||}^2_A \leq C \int_{\Omega}^{} {(g(P))}^2 d \Omega \]
Дост усл - я
\[ m \geq 3 \int_{\Omega}^{} |P|^2 f^2 (P) d\Omega < \infty \Rightarrow {||u_0||}_A^2 \leq C^2 \int_{\Omega}^{}{|P|}^2 f (\beta)d\Omega \]
\[ m \geq 2 \int_{\Omega}^{} f^2 (P) d\Omega < \infty \textrm{ и } f(P)= 0 |P| \Rightarrow 1\]

\subsubsection{ Эллиптическое уравнение в бесконечной области}

\[ H_{A_D} = \{ u \in H' (\Omega) \textrm{ и } {u|}_{\partial  \Omega} = 0 \} \]

\[ H_{A_H} = \{ \exists 0 \delta \frac{\partial u}{\partial x} \in L_2(\Omega) \} \]

\[ F(u) = \int_{k, j=1}^{m} A_{jk} \frac{\partial n}{\partial x}, \frac{\partial n}{\partial  x_n} d \Omega - 2lu \]
\[ l_N n = -\int_{\Omega}^{} grub u g d\Omega + \int_{\partial u}^{} u g_{\overline{n}} d S \]
\[ l_D u = -\int_{\Omega}^{} grub  g d\Omega \]

\[ lu = \int_{\Omega}^{} u (P) f(P) d \Omega \]

\[ Au = f \]
\[ B_ju=0; j=1,q \]
\[ H > D_A \] знак принадлежит в обратную сторону
\[ H_A \]

При условии $ u \in D_A $, но не обязятельно $ u\in H_A $ естественные

$ B_j: u \in H_A $ главные гр условия для $A$.

\[ - \Delta u = f \]
\[ \frac{\partial u}{\partial n} + \partial u = 0, \sigma > 0 \]
\[ {(- \Delta u, V)}_H = - \int_{}^{} V \Delta u d\Omega \]

\[ \int_{}^{} grud u grad V d\Omega - \int_{\partial \Omega}^{}V\frac{\partial u}{\partial n} dS \]

\[ F(u,u) = {||u||}^2_A - 2(f, u) = \int_{\Omega}^{} ({grud}^2 u - 2u) d\Omega + \int_{\partial \Omega}^{} \partial  u^2 dS \]

\[ u_0 = argmin F; \]
\[ \frac{d}{dt} (F(u_0 + t \eta))|'_{t=0} = 0 \]
\[ -\int_{\Omega}^{} \eta (\Delta u_0 + f) d\Omega + \int_{\partial \Omega}^{} \eta (\frac{\partial u_0}{\partial \eta} + \partial u_0) dS = 0 \]

\section{Лекция 8}

\subsection{Метод Бубнова-Галеркина}

\[ Lu = f \label{8_1_1}, D_2 \textrm{ плотность в H }\]
Опреатор L не обязательно положительный.
\[ Bu = 0 \label{8_1_1'} \]
\[ \{\phi_n\} \in D_A \textrm{ координатные функции } \]
Удовлетворяет (\ref{8_1_1'})
\[ u_n = \sum_{k=1}^{n}a_n \phi_k(P) \label{2} \]
\[ a_k \textrm{ выбирается из условия, что  } A u_n - f = \perp \phi_1, .., \phi_n \]
\[ \label{8_1_3} \sum_{k=1}^{n} (L\phi_k, \phi) a_k  = (f, \phi_j) j= \overline{1, n}\]

\subsection{Применение метода Б-Г к интегральному уравнению Фредгольма}

\[ u(P) - \int_{\Omega}^{} K(P, Q)u(Q) d\Omega f(P) \label{8_1_4} p \in p \in \Omega \]
\[ \int_{\Omega}^{}\int_{\Omega}^{} k^2 (P, Q) d\Omega_p d\Omega_;:{\Theta} \]
\[ \int_{}^{}f^2(P) < \infty \exists \textrm{ решение  } u(P) \textrm{ в H} \]
\[ H = L_2 (\Omega) \]
\[ \{ \widetilde{\phi_n} \} \textrm{ ПОНС }; (\phi_i, \phi_j) = \delta_{ij} \label{8_1_6} \]
\[ u_n = \sum_{k=2}^{n}a_k \phi_k (P) \]
\[ a_m - \sum_{k=1}^{n} \omega_m a_k =  f_m \label{8_1_7} \]
\[ f_m = (f_1, \phi_m) \]
\[ \Omega_{mk} = \int_{\Omega}^{}\int_{\Omega}^{} k(P, Q) \phi_m (P) \phi (Q) d\Omega_P d\Omega_Q \]
\[ f_n(P) = \sum_{k=1}^{n} f_k \phi_k (P) \]
\[ {lim}_{n\rightarrow \infty} \int_{\Omega}^{} \int_{\Omega}^{} (K_n (P, Q) - K(P, Q))^2 d \Omega_P d \Omega_Q = 0 \]
\[ {lim}_{n\rightarrow\infty} \int_{\Omega}^{} (f_n f^2) d\Omega \]
Вспомогательное уравнение
\[ u_n (P) - \int_{\Omega}^{} K_n (P,Q) u_n (Q)  d\Omega = f_n \]
Из + ИУ при дос большом \[ u(n) \exists!\textrm{ реш } \label{8_1_9}\textrm{  и } {||m_n - u||}_H \rightarrow^{n\rightarrow \infty} 0 \]
\[ u_n(P) = \sum_{k=1}^{n} A_k \phi_K (P) \]
\[ A_k = \sum_{k=1}^{n} \omega_{kl} \int_{\Omega}^{} u_l (Q) u_k(Q) d\Omega + (f, \phi_k) \]
\[ A_k -\sum_{l=1}^{n} \omega A_k = f_k\]

\subsection{Элементы теории приближение}

\[ H_A \supset H_N  \textrm{ - конечномерное }\]
\[ В \exists \textrm{ норм про-во X:  } \exists \textrm{ элемент наилучшего приближения }  \]
\[ \forall u \in X: \rho(u, H_N) = {inf}_{V \in H_N} \rho(u, V) X = C[a, b] \]
\[ 1, x, x^2, ..., x^N, ... \]
\[ | C[a, b] \rightarrow P_{N-1} (x) \]
\[ L_N(x) = \sum_{n=1}^{n} f(x_k) l_k(x) \]
\[ \{ l_k(x) \} -\textrm{ система фундаментальных многочленов } \]
\[ l_k(x) = \frac{(x-x^1) ... (x-x_N)}{(x_k-x_1) ... (x_k - x_N)} = \frac{\omega(x)}{(x-x_k) \omega'(x_K)} \]
\[ l_k(x_m) = \delta _{km}; \omega (x) = П_{k=1}^{m} (x = x_k) \]
\[ || f - L_N(x) ||_C \leq (1+||P||)\rho_N (f, H_N) \]
\[ ||P|| = {max}_{x \in [a, b]} \sum_{k=1}^{n} |l_k(x)| = \Lambda_N - \textrm{ построение Лебега }\]
\[ \Lambda_N - \textrm{ неогр возростает при  } n\rightarrow \infty \textrm{ для всего  } C[a,b] \textrm{и сущ зависит от выбора сетки } x_1, ... , x_N\]
\[ x_k = \frac{a+b}{2} + \frac{2}{b-a} t_k; t_K = -cos\{\frac{\pi}{2N}(2k-1)\}  \]
\[ \Lambda_N \approx \frac{2}{\pi} ln N + 1- q_N, 0<< q_N < \frac{1}{4} \]
\[ Lu = -\frac{d}{dx}(p(x \frac{du}{dx}) + q(x) u) \]
\[ Lu = f\textrm{ + гр у } u(a) = u(b) + 0 \]
\[ u_n(x) = \sum_{k=1}^{n} a_k l_k(x); a_k = u_N(x_k) \]
\[ \sum_{p=1}^{n} a_p (Ll_p, l_K) = (f, l_k) = \int_{a}^{b} f(x) ln(x) dx = f_K \]
\[ l_k(x_m) = \delta_{km}, \omega(x) = П^{m}_{k=1} (x- x_k) \textrm{ СЛАУ с mуравнений }\]
\[ a_{kl} = (Ll_k, l_p) = \int_{a}^{b}p(x) \frac{dl_n(x)}{dx} \frac{dlp(x)}{dx} + \int_{a}^{b}a(x) q(x) l_k(x) f(x) dx \]
\[ x_1 = a; x_N = b \Rightarrow \]
\[ l_1(x_1) = 0; l_N (X_N) = 0 \]

\[ u_n(x) = \sum_{k=2}^{N-1}u(x_K) l_N(x) \]
\[ \sum_{p=1}^{N-1}u_k a_{Kp} = f_{k} \]
\[ p = w \]
Пример
\[ p \equiv 1, a \equiv 0 \]
\[ f(x) = \{1, x \geq 0; -1, x<0 \} \]
\[ N=5; x_1=-1, x_2 = -\frac{1}{2}, x_2=-\frac{1}{2}, x_3 = 0, x_4 = \frac{1}{2}, x_5=1 \]
\[ l_2(x) = \frac{(x+1)x(x-\frac{1}{2})(x-1)}{\frac{1}{2}\cdot(-\frac{1}{2}) -1 -\frac{3}{2}} \]
\[ l_3(x) = (x+1(x+\frac{1}{2})(x-\frac{1}{2})(x-\frac{1}{2})) \]
\[ l_4(x)  = \frac{(x+1(x+\frac{1}{2})x(x-1))}{\frac{3}{2} \cdot 1 \cdot \frac{1}{2} \cdot -\frac{1}{2}} \]
\[ u_N - u_2 l_2(x) + u_3l_3(x) + u_4l_4(x) \]
\[ a_{kp} = \int_{-1}^{1} \frac{dl_k}{dx} \frac{dl_P}{dx} dx \]
период гр у
\[ [a, b] = [0, 2 \pi] \]
\[ u_N = \frac{a}{2} + \sum_{k=1}^{} a_k cos(kx) + b_k sin(kx) \]
\[ x_k = \frac{2 \pi}{x}(k-1) a_0, a_k, b_k \textrm{ Упр. } \]
\[ dim H_N - 2N-1 \]
\[ a_k cos(kx)+ b_k sin(kx) \]
\[ \Lambda_N \frac{1}{\pi} ln N + \delta (2-\frac{2}{\pi}), 0 < \delta < 1 \]

\subsection{Введение в теорию степенных сплайнов}

\[ [a, b] a = x_0 < x_1 < ... < x_{N-1} < x_N = b, h_K = s_k - x_{k-1} k = \overline{0, N-1}\, h_k = x+1 - x_k\]

Определение

Сплайн степени n, дефекта $ \nu $:

\[ S_{n\nu} = P_n (x) = \sum_{k=0}^{n} a_p^(k) (x-x_k)^P = \sum_{p = 0}^{n} {b'}_P{(x_{k+1} - x)}^P \]

\[ (x - x_K)^P_t = \{ (x - x_k)^P, x \geq x_k; 0, x \leq x_k \} \]

\section{Лекция }

\subsection{Степенные сплайны}

\[ \Omega = [a, b]\textrm{ Разбиение } a = x_0 < x_1 < ... < x_n = b \]
\[ h_I = x_i - x_{i-1} \]
\[ h = max_{i=1, N} h_i \]
\begin{enumerate}
	\item кусочнопостоянные сплайны

	задается многочленами степени 0.

	\[ \phi_i (x) = {1, x \in (x_{i-1}, x_i)}; 0, x \cancel{\in} (x_{i-1}, x_i) \]

	\[H_N = \lambda(\phi_1, ... , \phi_N) \]

	\begin{enumerate}
		\item Система линейно независима
		\item $ (\phi_i, \phi_j) = (h_i), i=j $
	\end{enumerate}

	Теорема:

	\[ \forall u \in {W'}_{p} (a, b) \exists V(x) \in H_N: \]
	\[ {|| u - v ||}_{L_2(a,b)} \leq c \cdot h {||u||}_{{W'}_p} (a, b) \]
	\[ {||u||}_{W'_p}(a, b) = {||u||}_{L_p (a, b)} + {|| \frac{du}{dx} ||}_{L_p (a, b)} \]

	Д

	\[ М = \int_{i=1}^{N} u_i \phi_i(x) \]
	\[ u_i = \frac{1}{h_i} \int_{x_i}^{x_{i-1}} uu(\xi)d \xi \]
	\[ {||u-v||}^p_{L_p(a,b)} = \int_{a}^{b} {|u - v|}^p dx = \sum_{i=1}^{N} \int_{x_{i-1}}^{xi} |u(x) - \frac{1}{h_i} - \int_{x_{i-1}}^{x_I} u (\xi) d \xi = \]
	\[ = \sum_{i=1}^{N} \int_{x_{i-1}}^{x_i} | \frac{1}{h_i} |\int_{x_{i-1}}^{x_i} (u(x) - u(\xi)) d\xi| dx = \]
	\[ \sum_{i=1}^{N} \int_{x_{i-1}}^{x_i} |\frac{1}{h_i} \int_{x_{i-1}}^{x_i} d \xi \int_{\xi}^{x} \frac{du}{d\eta} d \eta |^P dx \leq =  \]
	\[ \leq \sum_{i=1}^{N} \int_{x_{i-1}}^{x_i} |\frac{1}{h_i} \int_{x_{i-1}}^{x_i} d \xi \int_{x_{i-1}}^{x_i} |\frac{du}{d\eta} d \eta |^P dx = \]
	\[ = h_i (\int_{x_i-1}^{x_i} (\frac{du}{d\eta} d \eta))^P\]

	Неравенство Гелдера
	\[ | \int_{\Omega}^{} u(x) v(x) dx \leq {(\int_{\Omega}^{} |u(x)|^q dx)}^\frac{1}{q} (\int_{\Omega}^{} (u(x))^P dx)^\frac{1}{p} \]

	\[ |\int_{x_{i-1}}^{x_i} 1 \cdot |\frac{du}{d\eta}| d\eta \leq (\int_{x_{i-1}}^{x_i} 1^q dx)^\frac{1}{q} (\int_{x_{i-1}}^{x_i}| \frac{du}{d\eta} |^P dx )^\frac{1}{p} \]
	\[ (\int_{x_{i-1}}^{x_i} | \frac{du}{d\eta}| d\eta)^P \leq h^{\frac{p}{q}}_i \int_{x_i-1}^{x_i} |\frac{du}{d\eta}|^P d\eta \]
	\[ {||u-v||}^P_{L_P(a, b)} \leq \sum_{i=1}^{N} h_i^{q+\frac{p}{q}} \int_{x_{i-1}}^{x_i} |\frac{du}{d\eta}|^P  d\eta \leq h^{1+\frac{p}{q}} \int_{a}^{b} |\frac{du}{d\eta} d\eta \leq h^P ||u||^P_{{W'}_P} \]

	\[ |x_{(x)} - V^{(x)}| = |\frac{1}{h_i} \int_{x_{i-1}}^{x_i} (u(x)- u(\xi))s \xi| = |\frac{1}{h_i} \int_{x_{i-1}}^{x_i} d\xi \int_{\xi}^{x} \frac{du}{d\eta} d \eta | \leq \]
	\[ \leq h_i {sup}_{(x_{i-1}, x_i)} \geq h {||u||}_{{W'}_{(a, b)}} \Rightarrow \]
	\[ \Rightarrow {||u -v||}_{L_{\infty}(a,b)} \leq h {||u||}_{{W'}_\infty (a, b)} \]

	Для устойчивости нам нужно чтобы матрица грамма $ \widetilde{M} = ((\phi_i, \phi_j)) $ a с.н. были отр $ a_1 < |\Lambda| < a_2 $ $ a_1, a_2 $ не зав от N

	\[ wave{\phi_i} (x) = \frac{1}{\sqrt{h_i}} \{1, x \in (\phi_{i-1}, \phi_i); 0, x \cancel{\in} () \} \]

	\[ \Omega \subset \mathcal{R}^m, \Omega = \cup^{N}_{i=1} \Omega_i \]
	\[ {max}_{i=\overline{1, N}} {sup}_{x_{ij} \leq \Omega_i} |x-y| \leq h \]

	\item Кусочно линейные базисные функции

	\[ 0 < x_0 < x_1 < ... < x_N = b \]
	$ \forall $ узлы сетки $ x_i \Rightarrow \phi_i (x) = \{ \frac{x-x_{i-1}}{h_i}, x\in (x_{i-1} , x_i); \frac{x_{i_1} x_i}{h_{i+1}}, x \in (x_{i}, x_{i+1}); 0, x \cancel{\in} (x_{i-1}, x_{i+1}) \} i = \overline{i, N-1} $
	\[ \phi_0(x = \{ \frac{x_1 -x}{h_1}, x \in (x_0, x_1)); 0, x\cancel{\in} (x_{N-1}, ..., x_N) \} \]


	\[ {(\phi_i, \phi_j)}_{L_2(\Omega)} = \{ \cancel{=} , |i-j| \leq 1; 0, |i-j| > 1 \} \]

	\[ \nu = \sum_{a_i}^{\phi_i(x)} \in H_N \]

	Теорема

	\[ u \in W_2^2 (\Omega) \Rightarrow \exists V \in H_N = W_2^{1, n} (\Omega) \]
	\[ {||u-v||}_{L_2(\Omega)} \leq c_1 h^2 ||u||_{W_2^2} (\Omega) \]
	\[ ||u-v||_{{W'}_2} (\Omega) \leq c_2 \cdot h ||u|| W_2^2 (\Omega) \]
	\[ v(x) = \sum_{i=0}^{N} u(x_i) \psi_i (x) \]
	\[ \forall x \in (x_{i-1, x}) \]
	\[ u(x) - v(x) \]
	\[ \frac{dv}{dx} = \frac{u(x_i) - u(x_{i-1})}{h_i} x\in (x_{i-1}, x_i) \]
	\[ u(x) - v(x) = \int_{x_{i-1}}^{x} \frac{d}{d\xi}(u-v) d\xi =\]
	\[ = \int_{x_{i-1}}^{x} [\frac{du(\xi)}{d\xi} - \frac{\phi(x_i) - u(x_{i-1})}{h_i}] d\xi \]
	\[ = \frac{1}{h_i} \int_{x_{i-1}}^{x_i} d\xi \int_{x_{i-1}}^{x_i} (\frac{du (\xi)}{d\xi} - \frac{du(\eta)}{du}) d\eta =  \]
	\[ = \frac{1}{h_i} \int_{x_{i-1}}^{x_i} d\xi \int_{x_[i-1]}^{x_i} d \eta \int_{\eta}^{\xi} \frac{d^2u}{dt} (t) dt \]
	\[ \Rightarrow |u(x) - v(x)|^2 \leq h_i 4 \int_{x_{i-1}}^{x_i} |\frac{d^2u}{dt^2}|^2, x \in (x_{i-1}, x_i) \]
	\[ \sum_{i=1}^{N} (\cdot) {||u-v||}_{L_2 (\Omega)} \leq c_1 h^2 {||u||}_{W^2_2} (\Omega)\]

\end{enumerate}

\section{Лекция}

Получение нормы в 0v21

Теорема:

$ \textrm{ Если } n(x) \in W^2_{\infty} (\Omega), \textrm{ то } {||u-v||}_{L_{\infty}}(\Omega) \leq c_3 h^2 {||u||}_{W^2_{\infty}}$
\[ || u-v ||_{W'_{\infty}} \leq c_4 h ||u||_{W^2_{\infty}}(\Omega)\]
\[ \textrm{ Если } u \in c^2 (\Omega) \]
\[ {||u-V||}_{C(\omega)} \leq c_5 h^2 {||u||}_{c^2(\Omega)} \]

Докозательство - Упражнение

\[ \phi_i(x) = \frac{1}{\sqrt(h)}
\left\{
\begin{array}{l}
    \frac{x-x_i}{h_i}, x \in (x_{i-1, x_i}) \\
    \frac{x_{i+1-x}}{h_i+1}, x \in (x_i, x_{i+1}) \\
    0, x \in (x_i-1, x_i+1)
\end{array}
\right.
\]
\[ V = \sum_{i=0}^{N} \sqrt{h} u(x_i) \phi_i(x) \]

Пример:

\[
\left\{
\begin{array}{l}
    - \frac{d}{dx}(p(x) \frac{du}{dx}) + q(x) u = f(x), x \in (a, b) \\
    u(a) = 0; \frac{du}{dx}(a) = 0, f \in L_2(a, b)
\end{array}
\right.
\]
\[ H_A: {||u||}_A = \sqrt{\int_{a}^{b} (p (\frac{du}{dx})^2+qu^2) dx} \]
\[ H_N - L_{in} (\phi_0, ... \phi_N); \]

\subsection{Билинейные базисные функции в $\mathcal{R}$}

Все рассматривается для прямоугольной области.

\[ \Omega - \textrm{ прямоугольная в  } \mathcal{R^2}\]
\[ A_0 = x_0 < x_1 < ... < x_N < A_1, \Delta x_i = x_i - x_{i-1}, \Delta x = maxis_{i=1, N} \Delta x_i \]
\[ B_0 = y_0 < y_1 < ... < y_N = B, \Delta y_1 = y_i - y_{i-1}, \Delta y = {max}_{i = 1, N} \Delta y_i \]
\[ \phi_i(x) =
\left\{
\begin{array}{l}
    \frac{x_i - x_{i-1}}{\Delta x_i}, x \in (x_{i-1}, x_i) \\
    \frac{x_i+1 - x}{\Delta x_{i+1}}, x \in (x_i, x_{i+1}) \\
    0, x \cancel{\in} (x_{i-1, x_i})
\end{array}
\right.
\]
\[
y_j =
\left\{
\begin{array}{l}
    u
\end{array}
\right.
\]

\[ Q_ij(x, y) = \phi_i(x) \phi_j(y) \]
\[ y(x, y) = \sum_{i, j=1}^{N}a_{i, j} Q_{ij}(x, y), (x, y) \in \overline{\Omega} \]
\[ L(Q_{ij}) = W_2^{1, h} \owns {W'}_2 \]

Теорема

Если $ uu \in C^2 (\Omega) \Rightarrow \exists u^h \in W^{1, h}_2$
\[ {||u - u^h||}_{L_2(\Omega)} \leq c\cdot h^2 {||u||}_{C^2\Omega} \]
\[ {||u-u^h||}_{{W'}_2(\Omega)} \leq c \cdot h {|| u ||}_{C^2(\Omega)} \]

\[ \xi(x, y) = (x - x_l) \frac{\partial \xi}{\partial x} (x_l, y_k) + (uu - y_k) \frac{\partial \xi}{\partial y} (x_k, y_k) + \int_{x_l}^{x} dx' \int_{x_l}^{x'} \frac{\partial ^2 xi}{\partial {x''}^2} (x'', y_n) dx'' + \]
\[ + \int_{y_k}^{y}dy' \int_{y_k}^{y'}dy' \frac{\partial ^2 \xi}{\partial {y''}^2}(x_l, y'') d y'' + \int_{x_l}^{x} \int_{y_k}^{y} \frac{\partial ^2 \xi}{\partial x' \partial y'} (x', y') dx' dy'  \]

...

\[ (x - x_l) \frac{\partial \xi}{\partial x}(x_l, y_k) =  \]
\[ = \frac{x-x_l}{\Delta x_l H} \int_{x_{l+1}}^{x_l} dx' \int_{x'}^{x_l} \frac{\partial ^2 u}{\partial {x''}^2} (x'', y_k) dx'' \]

\[ \int_{x_l}^{x} dx' \int_{x_l}^{x'} \frac{\partial ^2 \xi}{\partial {x''}} dx' \int_{y_k}^{y} \frac{\partial ^2 \xi}{\partial x' \partial y'} (x', y') dx' dy' = \]
\[ = \frac{1}{\Delta x_{l+1}, \Delta y_{k+1}} \int_{x_l}^{x_{l+1}} dx'' \int_{y_k}^{y_{k+1}} dy'' \cdot \int_{x_l}^{x} dx' \int_{y_k}^{y} d y' (\frac{\partial u}{\partial x' \partial  y'}( \frac{\partial u}{\partial x' \partial y'}(x', y') - \frac{\partial ^2 u}{\partial x''\partial y''})) dy' \]

...
здесь была получена первая оценка

ПОлучить вторую оценку это второе упражнение.

Более сильная оценка. Упр со *.
\[ x {||u - y^h||}_{C(\Omega)} \leq C (\Delta x^2 + \Delta y^2) \sum_{i=2}^{} {||D^{(1)} u}_{\Omega}\]

\[ F(u) = \int_{\Omega}^{} ((\frac{\partial u}{\partial x})^2 + (\frac{\partial u}{\partial y})^2 + u^2 - 2uf) dx dy \]

\[ H_A - {\overset{0}{W}'}_2( \Omega) \]

Теорема

\[ \forall u \in W'_2 (\Omega) \cup C^2(\Omega) \]
\[ \exists u^h \in W_2^{1, h} : \textrm{ оценить } (T^1) \]

\subsection{Построение проекционно сеточной схемы для ОДУ 2-го порядка}

\[
\left\{
\begin{array}{l}
    -\frac{d}{dx}(p(x) \frac{du}{dx})\frac{du}{dx} + q(x)u(x) = f(x), f \in L_2(a, b) \\
    u(a) = u(b) = 0 \\
    Au = f,  0< p_0 \leq p(x) \leq p, 0 \leqq(x) \leq q
\end{array}
\right.
\]

\[ H = L_2 (a, b) \Rightarrow A \textrm{ положительно определена } \Rightarrow \exists A^{-1} \Rightarrow ! \textrm{ реш } u(3) \]

\[ \ \]
\[ {||u||}_{W_2^2} (\Omega) \leq c {||f||}_H \]
\[ H_A = \overset{0}{W}_2'(\Omega); c_0 {||u||}_{W'_2} \leq {||n||}_A \leq c_1 {||u||}_{{W'}_2} \]
\[ F(u) = [u, u] -2 (u, f) \rightarrow min \textrm{ на } \overset{0}{{W'}_2} = H_A; u_i = \frac{1}{\sqrt{h}} \textrm{ далее тоже самое что и раньше } \]
\[ u^h(x) = \sum_{i=1}^{N-1} a_i \phi_i (x) \]
\[ \overset{0}{W}_2^{i, h} = \{ v = \sum_{i=1}^{N-1} a_i \phi_I (x) \} \]
\[ \textrm{ Минимизируем }F(v) \textrm{ на } \overset{0}{W}_2^{1, h}; \]
\[ a_i \textrm{ из }\frac{\partial F}{\partial a}(u_n) = 0, i=1, N-1 \]
\[ \hat{A a} = f; \hat{A} = (A_{ij}) = ([[\phi_I, \phi_j]]) \]
\[ a = (a_{i}, ... , a_{N-1})^T \]
\[ {f = (f_1, ... , f_{N-1})}^T \]
\[ f_i = \int_{\Omega_i}^{} f \phi_i dx \]
\[ \phi_i = \int_{\Omega}^{} (p \frac{d\phi_i}{dx}\frac{d \phi_i}{dx} + q \phi_i \phi_j) \]

\[ \exists ! \textrm{ реш } {(a_1, ..., a_{N-1})}^T \]
кот однозначно определяется $ u^h \leftarrow argmin F(Vq) $

Упр Найти $ A_{ij} в случае p и q кусочнопостоянны на сетке$
\[ p_{i-\frac{1}{2}} = p(x)\]
\[x\in (x_{i-1}, x_i) i = \overline{1, N} \]
\[ q_{i-\frac{1}{2}} = - ... \textrm{ аналогично } \]

\section{Лекция}

Продолжение прошлой лекции

Постановка с прошлой лекции:
\[ -\frac{d}{dx} [ p(x) \frac{du}{dx}] + q(x) u = f \]
\[ u(a) = u(b) = 0 \]

\[ ||u - u_{H}|| \leq ||u - V_n||,\textrm{ где } V_n = \sum_{i=1}^{N-1} h_i \phi_i; \forall V_n \in H_{A^{(N)}} = \overset{\circ}{W}_2^1, h \]

\[ c_0 {||u||}_{{W'}_2} \leq {||u||}_A \leq c_1 {||u||}_{{W'}_c} \]

\[ {||u-u_h||}_{{W'}_2} \leq ch {||f||}_H \]

\[ {||u-u_h||}_{H=L_2} \leq c_2 h {||f||}_H \]

\[ {[u, V]}_A = (f, V), v \in {W'}^0_2 \]

\[ {[u_n, V_n]}_Aa = (f, V_n);  \forall V_h \in \overset{\circ}{W}_2^{1, h} = H_A^{(N)} \]

\[ [u - u_n, V_n] = 0 \forall V_n \in \overset{\circ}{W}_2^{1, n} \]

Рассмотрим вспомогательную задачу

Здесь немного путаю n и h, нужно поправить

\[ A \Phi = F, F = u - u_h \textrm{ по св-вам } A \exists! \textrm{ реш } \Phi: \]

\[ ||\Phi||_{W^2_2} \leq c ||F|| = C ||u - u_h|| \]

и $ \Phi $ удовлетворяет:

\[ [\Phi, V] = (F, V) \forall V \in \overset{\circ}{W}^1_2 \]
Рассмотрим $ V=u - u_h $

\[ (F, V) = (u - u_h, u - u_h) = (A \Phi, u - u_h) = \]
\[ = [\Phi, u - u_n] - [u - u_n \Phi_h] = \]
\[ = [\Phi - \Phi_h, u - u_n] \leq {||\Phi - \Phi_n||}_A \cdot {||u - u_n||}_A \leq \]
\[ \leq ch ||f|| {||\Phi - \Phi_n||}_A \leq \]
\[ \leq ch ||f|| \cdot {||\Phi - \Phi_h||}_{{W'}_2} \]

Из результатов аппроксимации можно выбрать $ \Phi_h $:

\[ {||\Phi - \Phi_h||}_{{W'}_2} \leq ch {||\Phi||}_{W^2_2} \]

\[ \rightarrow {||u-u_h||}^2 \leq ch {||f||} \cdot {||\Phi- \Phi_h||}_{{W'}_2} \leq ch^2 ||f|| \cdot {||\Phi||}_{W_2^2}  \]

\[ ||u - u_h|| \leq ch^2 ||f|| \cdot ||u - u_h|| \]

Расмотрим задачу с другими граничными условиями.

упр 1

\[
\left(
    \begin{array}{c}
        -\frac{d}{dx} (p(x) \frac{du}{dx}) + q(x) u = f(x) \\
        u(a) = 0, \frac{du}{dx}(b) = 0
    \end{array}
\right)
\]
упр 1 Проверитть оценки в $ {W'}_2 $
упр 2 Найти $ \hat{A} $ при $ h_i = h, i = overline{1, N} $,
$ \Phi = 0, P = const $

\subsection{Применение ВП к з. Дирихле для ур-я Лапласа}

\[ \Omega = \{0 < x < a, 0 < y < b\} \{- \delta u = f(x, y); u|_{\partial \Omega} = 0 \} \]

\[ \forall A u = - \Delta u , D(A) = \{u \in W_2^2(\Omega) , u|_{\partial \omega} = 0 \}, f \in L_2(\Omega) \]

A симметричный положительно определенный

\[ \Rightarrow \forall f \in L_2 \exists ! n \in \overset{\circ}{W'}_2 \textrm{ пересечение } W_2^2 \]

\[ H_A = \overset{\circ}{W'}_2 (\Omega), {(u, V)}_A \int_{\Omega}^{} (\frac{\partial u}{\partial x}\frac{\partial v}{\partial x}+\frac{\partial u}{\partial y}\frac{\partial V}{\partial y})d\Omega\]
\[ {[u, V]}_A = (f, V), \forall V \in H_A; h_x = \frac{a}{N_x} \]
\[
\phi_ij = \frac{1}{\sqrt{h_x h_y}}
\left(
    \begin{array}{c}
        x_j \leq x \leq x_{i+1}; (1 - (\frac{u}{h_y}-j)) \\
        \textrm{ даллее упр }
    \end{array}
\right)
\]

\[ u_h = \sum_{i=1}^{N_x -1} \sum_{j=1}^{N_y - 1} a_{ij} \phi_{ij} (x, y); H_A^N = \overset{\circ}{W'}_2^{1, h} \subset H_A = \overset{\circ}{W'}_2 \]

$ a_ij $ - решение СЛАУ

\[ {[u_h, phi_{kl}]}_A = (f, \phi_{kl}) \]

\[ \hat{A} a = f \]

\[ A_{ijk} = {[\phi_{ij}, \phi_{kl}]}_A; \phi_i; \textrm{ кусочно лин. } \]

Упр Найти $ A_{ijkl} $

\[ a (a_{11}, a_{21}, ... , a_{N_{x-1}}, ... , a_{1, N_y-1}, a_{2 N_y-1, ... , a_{N_x-1 N_{y-1}}}) \]

Оценки:
\[ {||u - u_h||}_{{W'}_2} \leq ch ||f|| \]

\[ ||u - u_h|| \leq ch^2 ||f|| \]

\[ A = - \frac{d}{dx} (p(x) \frac{du}{dx}) + q(x)u \]

упр показать предедущее выражение

\[ \frac{2a_{ij} - a_{i-1, j} - a_{j+1, i}}{h^2_x} + \frac{2a_{ij} - a_{i, j-1} - a_{i, j+1}}{h^2_y} = f_ij = {(f, \phi_{ij})}_H \]

\[ \partial  \Omega_h \subset \partial  \Omega \]

h - max сторона треугольника $ \triangle $
$\Theta_0 - min \triangle $

\[ {||u-u_h||}_{{W'}_2(\Omega)} \leq c \frac{h}{sin \Omega_0} ||f|| \]

\[ u_h = \sum_{i=1}^{N} a_i \phi_i (x, y) \]

\subsection{Подходы к решению неоднородных краевой задачи}

\[
\left(
    \begin{array}{c}
        - \Delta u = f \textrm{ в } \Omega \\
        u|_{2\omega} = g
    \end{array}
\right)
\]

\[ f \in L_2 (\Omega) , g \in W^{\frac{3}{2}}_2 (\partial \Omega) \]

\[ c_3 ({||f||}_L_2 + {||g||}_{W_2^{\frac{3}{2}}}) \leq {||u||}_{W^2_2(|omega)} \leq c_u() \]

\begin{enumerate}
    \item Сведение к однородным гр. у

    \[ пусть  \exists \Phi \in  D(A): \Phi=g  \textrm{ на \partial \Omega } \]
    \[ V = u - \Phi \]
    \[ - \Delta V = f = f + \Delta \Phi \in L_2 (\Omega); V |_{\partial \Omega} = 0 \]

    \[ u_h = V_h + \Phi \]

    \[ \Phi \in {W'}_2 (\Omega) \]

    \item СНос тр. у

    \[ u_h \sum_{i=1}^{N} a_i \phi_i (x, y) \]

    \[ i = \overline{1, N} = [u_h, \phi_i]_A = (f, \phi_i) \]

    \[ i = \overline{N + 1, \widetilde{N}} a_i \phi_i (x_i, y_i) = g(x_i, y_i) \]

\end{enumerate}

\subsection{Метод штрафа}

рассмотрим модифицированную 3 краевую задачу
\[
\left(
    \begin{array}{c}
        - \Delta u_{\epsilon} = f \textrm{ в  } \Omega \\
        u_{\epsilon} + \epsilon \frac{\partial u}{\partial n} = g \textrm{ на  } \partial \Omega
    \end{array}
\right), \epsilon > 0 \textrm{ мало }
\]

для нее $ [u, v]_A = \int_{\Omega}^{} \frac{\partial u}{\partial x} \frac{\partial V}{\partial x} + \frac{\partial u}{\partial y} \frac{\partial V}{\partial y} d\Omega + \int_{\partial \Omega}^{} \frac{1}{\epsilon} u V dS $

\[ u_{\epsilon, h} = \sum_{i=1}^{N} a_i \phi_i(x) \]

$ a_i $ находится из

\[ [u_h, \phi_i]_A = (f, \phi_i) + \int_{\partial \Omega}^{}g \phi dS \]

\[ ||u_\epsilon - u_{\epsilon, n} ||_{W'}_2 (\Omega) \leq \]
\[ \leq \frac{ch}{sin \Theta_0} {(q + \frac{h}{\epsilon})}^2 ({||f||}_{L_2} + \frac{1 }{\epsilon} ||\phi||_{W_2^2} (\partial \Omega )) \]


\end{document}
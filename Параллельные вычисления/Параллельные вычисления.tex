\documentclass{article}

\usepackage[T1,T2A]{fontenc}
\usepackage[utf8]{inputenc}
\usepackage[english,russian]{babel}

\usepackage{pdfpages}
\usepackage{multirow}

\usepackage{caption}

\usepackage{amsmath}
\usepackage[hidelinks]{hyperref}

\usepackage{graphicx}%Вставка картинок
\graphicspath{{noiseimages/}}

\usepackage{float}%"Плавающие" картинки
\usepackage{wrapfig}%Обтекание фигур (таблиц, картинок и прочего)

\setlength{\emergencystretch}{10pt}

\begin{document}



\section{Вступительная, короткая лекция}

ФИО преподователя -  Борис Николаевич Четверушкин.

Уильм Оккамский - ученый в честь которого назвали язык (Occam).

Транcпютер - — это микропроцессор со встроенными средствами межпроцессорной коммуникации, предназначенный для построения многопроцессорных систем.

Компания INMOS. Компьютеп t-800.

\section{Вторая лекция}

\begin{enumerate}
	\item Возможность параллельного разбиения
	\item Равномерная загрузка узлов
	\item Минимизация обмена между узлами
	\item Автоматизация
	\item Логическая простота
	\item Корректность алгоритмов и математических моделей.
\end{enumerate}

\begin{enumerate}
	\item Искать другую модель или другой метод
	\item Передать точки более загруженным процессорам
	\item Пример про моделирование ковида и явного метода Рунге
\end{enumerate} 

Пока говорим, только об однородном алгоритме

На следующей лекции h физическая модель, которая позволяет считать лучше.

\section{Лекция 3}

$ 19^9 $ число частиц в кубометре воздуха.

\[ f(t, \overline{x}, \overline{\xi }) \]

\[ \rho = \int_{}^{} mf(t, \overline{x}, \overline{\xi }) d \overline{\xi  } \]
\[ \rho \overline{u} = \int_{}^{} mf(t, \overline{x}, \overline{\xi }) \xi    d \overline{\xi  } \]

\[ \overline{c } = \overline{\xi   } - \overline{u } \] 

\[ P_{ij} = \int_{}^{} mc_i c_j f(t, \overline{x}, \overline{\xi }) d \xi   \]

\[ q_i = \int_{}^{} \frac{m }{2 } e^2 c_j f(t, \overline{x}, \overline{\xi }) d \xi   \]

\[ \frac{3}{2} k T = \frac{1}{\rho   } \int_{}^{} \frac{mc^2}{2} f d \overline{\xi   } \]

\[ P = \rho \frac{K }{m } T = \rho R T \]

\[ P = \frac{1}{3} (P_{11} + P_{22} + P_{33}) \]

законы сохранения:
\begin{enumerate}
	\item \[ m + m_1 = {m}' + {m_1}' \]
	\item \[ m \overline{\xi} + m_1 \overline{\xi} = {m \overline{\xi}}' + {m_1 \overline{\xi}}' \]
	\item \[ \frac{m \xi^2}{2} + \frac{m_1 \xi^2}{2} = \frac{{m \xi^2}'}{2} + \frac{{m_1 \xi^2}'}{2} \]
\end{enumerate}

\[ f(t, \overline{x}, \overline{\xi}) \]

\[ t_1 = t + \Delta t \]
\[ \overline{x_1} = \overline{ x} + \overline{\xi} \Delta t \]
\[ \overline{\xi_1} = \overline{\xi} + \overline{\gamma} \Delta t \]

\[ \frac{\partial f }{\partial t } = \xi_i \frac{\partial f }{\partial x_i} + \gamma \frac{\partial f }{\partial \xi_i } = 0 \]

\[ \sum_{}^{} \overline{(\xi)} d \overline{x } d \overline{\xi} = dx d\xi \int_{}^{} f(t, \overline{x }, \overline{\xi}) f(t, \overline{x_1 }, \overline{\xi_1}) |g| \gamma d \gamma d \Theta d \overline{\xi_1} \]

\[ \sum_{+}^{} dx d\xi' = dx d\xi' \int_{}^{}f' {f'}_1 |g'| B' dB d \Theta d {\xi'}_1\]

Теорема Леувилля о сохранении фазового объема.

\[ \frac{\partial f }{\partial t } = \xi_i \frac{\partial f }{\partial x_i} + \gamma \frac{\partial f }{\partial \xi_i } = \iint (f' {f'}_1 - f f_1) |g| d db d \Theta d \overline{\xi_1} \]

\[\int_{}^{} \phi(\xi ) \frac{\partial f }{\partial t } = \xi_i \frac{\partial f }{\partial x_i} + \gamma \frac{\partial f }{\partial \xi_i } = \iint (f' {f'}_1 - f f_1) |g| d db d \Theta \phi(\xi) d \overline{\xi_1} \]

\[ I_{\phi}(t, \overline{x}) = \int_{}^{} g(t, \overline{x}, \overline{\xi}) \phi(\xi) d \overline{\xi}\]

\[ I_{\phi} = \frac{1}{2} (I_{\phi} + I_{{\phi}_1})\]

\[ I_{\phi}(\xi) d \xi =  \xi_i \frac{\partial f }{\partial x_i} + \gamma \frac{\partial f }{\partial \xi_i } = \iint (f' {f'}_1 - f f_1) |g| d db d \Theta \phi(\xi) d \overline{\xi_1} \]

\[ f'\int_{}^{} \phi(\xi) \frac{\partial f }{\partial t }d \xi = \frac{\partial \int_{}^{} f \phi (\xi) }{\partial t} \]

\[ \int_{}^{} \phi (\xi) \xi_i \frac{\partial f }{\partial x_i }d \xi = \frac{\int_{}^{} \xi_i \phi(\xi ) d \overline{\xi}}{\partial x} \]

\[ \int_{}^{} \phi (\xi) \gamma_i \frac{\partial f }{\partial \xi_i} d \xi = \int_{-\infty}^{+\infty} d \xi_k \int_{-\infty}^{\infty} d \overline{\xi} = \int_{-\infty}^{\infty}  -\int_{}^{} f \gamma_i \frac{\partial \phi}{\partial \xi } d \overline{\xi} \]

\section{Лекция 4}
Начало лекции пропустил. 

\[ f_o(t, \overline{x }, \overline{\xi}) = \frac{\rho (t, \overline{x})}{{(2 \pi R T(t, x))}^{\frac{3}{2}}} \overline{l}^{\frac{{(3i- u(x, t))}^2}{2RT}} = \frac{\rho }{ {(2 R T)}^{\frac{3}{2}} l^{\frac{-l^2_i}{2RT}}} \]

\[ \int{}_{\infty}^{\infty} x^{2n} l^{-\beta x^2} dxl = \frac{(2n-1) \dots 3 \cdot 1}{2^n} \sqrt{\frac{\pi }{\beta^{2n+1}}} \]

\[ P_ij = \int e_i e_j f_0 d\xi  = \frac{\rho}{{(2\pi R T)}^\frac{3}{2}} ... \]

\[ P_ij = \int e^2_i f d\xi  \frac{\rho}{(2\pi RT)^{\frac{3}{2}}} [ \int e^{\frac{-c i^2}{2 RT }} d e \int e^{\frac{-e_k}{2RT}} \int e^2_i e^{-\frac{e^2_i}{2RT}} dc_i = \rho RT] \]

БГК 

\[ \frac{dR}{dt} = \nu (f - f_0) \]  - число столкновений частиц велико

\[ f = f_0 - \frac{1}{\nu} \frac{df_0}{dt } + \frac{1}{\nu } \frac{d^2 f }{dt } + \dots =  \]

\[ q_i \frac{1}{\nu } \frac{5}{2} R^2 \rho T \frac{\partial t }{\partial x^i} \]

Анонас следующей лекции:

\[ f = f_0 = \frac{\rho_i}{{2RT_i}} k^{-\frac{{(3_k - u_{ki})}^2}{2RT_i}} \] 

**рисунок**

\[ f^{j+1} (\xi) = f^j_{i_0} + \xi \Delta t \frac{f_{i+1, 0} - f^j_{i-1, 0}}{2 \Delta x} - \Delta t \frac{|\xi|}{2} \frac{f^{i_+1, 0} - 2^i_{i_0} + f^j_{i-1}}{\Delta x} \]

\section{Лекция 5}

\[ f^{j+1}_{i} = f^i_{i_0 }(\xi) + \xi \Delta t \frac{f^i_{i+1, 0} - f^j_{i-1, 0}}{2 \Delta x} - \Delta x \frac{|\xi|}{2} \frac{f^j_{i+1, 0}-2f^j_{i-1, 0} + f^i_{i-1, 0}}{\Delta x}\]

\[ \int f^{i+1}_j d\xi = \rho^{i+1}\]
\[ \int f^{i}_i d\xi = \rho^{i}\]

\[ \beta = \frac{1}{\sqrt{ 2 R T}} \]
\[ a_{x0} = \frac{a_{i+1} - a_{i-1}}{2 \Delta x} \]
\[ a_{overline{x} x} = \frac{a_{i+1} - 2a_i + a_{i-1}}{\Delta x^2} \]

Разностная схема:

\[ \frac{\rho _i^{j+1} - \rho_i^j}{\Delta t} + (\rho u)_{x with 0} = \frac{\Delta x}{2} [\rho u erf(s) + \frac{\rho}{ \beta \sqrt{\pi} } exp(-s^2)]_{overline x x} \]

\[ \frac{\rho }{} \]

еще 2 уравнения не успел 

\[ \frac{\partial f }{\partial t} + \xi \frac{\partial f }{\partial x} = J({f'}_1, f') \]

\[ \frac{f^{i+1} - f^i_j }{\Delta t} + \xi \frac{ f^i_j - f^i_{j-1}}{\Delta x} = J, \xi > 0\]
\[ \frac{f^{i+1} - f^i_j }{\Delta t} + \xi \frac{ f^i_j - f^i_{j+1}}{\Delta x} = J, \xi > 0\]

\[ \frac{f^{i+1} - f^i_j }{\Delta t} + \xi \frac{ f^i_j - f^i_{j+1}}{\Delta x} \frac{\Delta x }{2} |\xi| \frac{f^j_{i+1} - 2f^i_j + f^j_{i-1}}{\Delta x^2} + J(ff'), \xi > 0\]

Ур Больцмана - Ур газ динамики - разностная схема
Ур Больцмана - разностное киетич ур - ???

\[ \frac{f^{i+1}_j - f^{i}_j}{\Delta t} + \frac{\xi + |\xi| }{2}\frac{15 f^i_j - 2f^i_{j-1} + 0.5 f^j_{i-2}}{\Delta x} = ... \]


\[ K_n = \frac{l}{L} << 1\]

**рисунок**

\[ f(t^{i+1}, \overline{x}, \overline{\xi}) - f_0^i(t, x - \xi J, \xi) \]

\[ \frac{f^{i+1}_i - f^i_{i 0}}{\Delta t } + div (\xi f_0)^i = \frac{\partial}{\partial X_i} \frac{J }{2} \xi_i \xi_k \frac{\partial f^j_0}{\partial x_k} \]

Система уравнений (которая оказывается совпадает с уравнением Навье - Стокса)

...

\[ \frac{\partial \rho u }{\partial t} + \frac{\tau}{2} \frac{\partial^2 \rho u  }{\partial t^2} + \frac{\partial (\rho u^2 + P)}{\partial x} = \frac{\partial }{\partial x} \frac{\tau}{2} \frac{\partial }{\partial x} (\rho) ...\]

...



\[ \frac{\partial \rho}{\partial \tau} + J \frac{\partial^2 \rho}{\partial \tau^2} + \frac{\partial}{\partial x} (\rho u = \frac{\tau}{2} \frac{\partial }{\partial x } ((\rho u^2 + P)))  \]

\[ \frac{\partial Q }{\partial t } + \frac{\tau}{2} \frac{\partial^2 Q}{\partial x^2} \]

Построим схему более устойчивая.

\[ \frac{\partial T }{\partial t} = \epsilon \frac{\partial^2 T}{\partial t^2} = x \frac{\partial^2 T }{\partial x^2} \]

\[ \frac{T^{j+1}_{i} - T_i^{j-1}}{2 \Delta \tau} + \epsilon \frac{T_i^{j+1} - 2 T_i^j + T_i^{j-1}}{\Delta t^2} + x \frac{T^j_{i+1} - 2T_i^j+T_{i-1}{j}}{\Delta x^2} \]

\end{document}

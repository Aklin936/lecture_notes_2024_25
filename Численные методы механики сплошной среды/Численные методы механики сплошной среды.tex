\documentclass[12pt, a4paper]{article}

\usepackage[T1,T2A]{fontenc}
\usepackage[utf8]{inputenc}
\usepackage[english,russian]{babel}

\usepackage{pdfpages}
\usepackage{multirow}

\usepackage{caption}

\usepackage[fleqn]{amsmath}
\usepackage{amssymb}
\usepackage{cancel}

\usepackage[hidelinks]{hyperref}

\usepackage{graphicx}%Вставка картинок

\usepackage{float}%"Плавающие" картинки
\usepackage{wrapfig}%Обтекание фигур (таблиц, картинок и прочего)


\setlength{\emergencystretch}{10pt}

\usepackage{indentfirst} %Отступ после секции

\usepackage[left=2cm,right=1cm,
top=2cm,bottom=2cm,bindingoffset=0cm]{geometry}

\usepackage{setspace}

\usepackage{tocloft}
\setlength\cftsecnumwidth{0em}

\usepackage{titlesec}% http://ctan.org/pkg/titlesec
\titleformat{\section}%
[hang]% <shape>
{\normalfont\bfseries\Large}% <format>
{}% <label>
{0pt}% <sep>
{}% <before code>
\renewcommand{\thesection}{}% Remove section references...
\renewcommand{\thesubsection}{\arabic{section}.\arabic{subsection}}%... from subsections

\begin{document}
\thispagestyle{empty}

\begin{center}
	\ \vspace{-1cm}

	{Московский государственный университет имени М. В. Ломоносова}\\
	Факультет вычислительной математики и кибернетики\\
	Кафедра вычислительных методов

	\vspace{8cm}
	\begin{spacing}{2.5}
		{\huge \bfseries ЧИСЛЕННЫЕ МЕТОДЫ МЕХАННИКИ СПЛОШНОЙ СРЕДЫ}
	\end{spacing}


\end{center}

\vfill

\begin{center}
	Москва, 2024
\end{center}

\enlargethispage{2\baselineskip}

\newpage

\tableofcontents

\newpage

\section{Лекция 1}

\subsection{Уравнение конвекции с диффузией}

\[ \frac{\partial Q}{\partial t} +\frac{\partial }{\partial x} (u Q) = \frac{\partial }{\partial x} (\mathcal{D} \frac{\partial Q}{\partial x}); t>0; x \in [-L, L] \]

Уравнение конвекции с диффузией.

\[ Pe = \frac{UL}{\mathcal{D}} \] - число Пекле

\[ \frac{L}{U} = t_{\textrm{конв}} \]
\[ \frac{L^2}{\mathcal{D}} = t_{\textrm{дифузионная}} \]
\[ \frac{t_{\textrm{диф}}}{t_{\textrm{конв}}} = \frac{L U}{\mathcal{D}} \]

\[ Re << 1 \] почти дифф процесс
\[ Re >> 1 \] почти конв процесс

Сначала будем рассматривать стационарное уравнение.

\[ \frac{d}{dx}(uQ) - \frac{d}{dx} (\mathcal{D} \frac{d Q}{d x}) = 0 \]

На отрезке вводится разностная сетка \[ \Omega_n (x_{i-1}, x_{i}, x_{i+1})\].

Потоковые точки - точки с полуцелые точки

\[ x{i+\frac{1}{2}} = \frac{x_i + x_{i+1}}{2} \] - потоковые узлы

\[ h_{i+\frac{1}{2}} = x_{i+1} - x_{i} \]

\[ h_{i} = \frac{h_{i+\frac{1}{2}} - h_{i-\frac{1}{2}}}{2} \]

Вводим сеточные функции.

\[ Q_i = (x_{i+\frac{1}{2}}; x_{i-\frac{1}{2}}) \]
константа на отрезке
\[ [x_{i+\frac{1}{2}} \textrm{до} x_{i-\frac{1}{2}}] \]

D - константа в полуцелых точках (от $ x_i $ до  $ x_{i+1} $ константа)

Считаем скорость и тепло в потоковых точках, где диффузия постоянна

\[ Q_x = \frac{Q_{i+1}i = Q_i}{h_{i+\frac{1}{2}}} \]
\[ Q_{\overline{x}} = \frac{Q_{i} - Q_{i-1}}{h_{i-\frac{1}{2}}} \]
\[ Q_{\widehat{x}} = \frac{Q_{i+\frac{1}{2}} - Q_{i-\frac{1}{2}}}{h_{i}} \]

Между потоковыми точками интегрируем получаем формулу.

\[ \int_{x_{i}+\frac{1}{2}}^{x_{i}+\frac{1}{2}} \frac{d}{d x} [uQ - \mathcal{D}\frac{dQ}{dx}]dx = \]
\[ = [uQ - \mathcal{D}\frac{dQ}{dx}]|_{x_{i} - \frac{1}{2}}^{x_i + \frac{1}{2} } = \]
\[ W^{tot}_{i + \frac{1}{2}} \]

\[ W^{tot}_{i-\frac{1}{2}} = U_{i+\frac{1}{2}} Q_{i+\frac{1}{2}} = W^D_{i+\frac{1}{2}} - W^D_{i-\frac{1}{2}} \]

\[ W^D_{i+\frac{1}{2}} = - D_{i+\frac{1}{2}} \frac{dQ}{dx}|_{i+\frac{1}{2}} \]

\[ U_i{i+1/2} Q_{i+1/2} - D_{i+1/2}\frac{Q_{i+1} - Q_i}{h_{i+\frac{1}{2}}} -\]
\[ - [U_i{i-1/2} Q_{i-1/2} - D_{i-1/2} Q_{{\overline{x}}_{i-1/2}}]= 0 \]

\[ \frac{U_{i+1/2} Q_{i+1/2}  - U_{i-1/2} Q_{i-1/2}}{\overline{h_i}} -\]
\[-\frac{D_{i+1/2} Q_x - D_{i-1/2} Q_{\overline{x}}}{\overline{h_i}} = 0 \]

Интерполируем Q в полуцелые точки.

\[ Q_{i+1/2} = \Theta_{i+1/2} Q_i + (1-Q_{i+1/2}) \Theta_{i+1} \]

\begin{enumerate}
	\item \[ \Theta = 1/2 \]

	\[ W = .... \]

	Получится схема с центральными разностями

	\[ u Q_x - D Q_{x with lint x} = 0 \]

	\item \[ \Theta_{i+1/2} = 1/2(1+\frac{|u_{i+1/2}|}{U_{i+1/2}}) \]

	\[ W_{i+1/2} = u^+_{i+1/2} Q_i - \overline{U}_{i+1/2} Q_{i+1} \]
	\[ U'_{i+1/2} = \frac{1}{2} (U_{i+1/2} + |U_{i+1/2}|) \]
	\[ U with lin_{i+1/2} = \frac{1}{2} (U_{i+1/2} - |U_{i+1/2}|) \]

	\[ (u^+_{i+1/2} Q_i + \overline{u}^+_{i+1/2} Q_i+1) - D Q_{\overline{x} \widehat{x}} = 0 \]

\subsection{Модельная задача}

	\[ \frac{d(UQ)}{dx} - D \frac{d^2Q}{dx^2} = 0 \]

\end{enumerate}

При увеличении числа Пeкле график прижимается к координатам снизу.

\subsection{Решение разностной задачи (центральные разности).}

\[ U \frac{Q_{i+1} - Q_{i-1}}{2h} - \frac{Q_{i+1} - 2Q_{i} - Q_{i-1}}{h^2} = 0\]

\[ Q_i = a + b q^i \]

\[ U \frac{Q_{i+1} - Q_i}{2h} - \frac{Q_{i+1} - 2Q_{i} - Q_{i-1}}{h^2} = 0\]

Подставляем

\[ \frac{U}{2h} b(q^{i+1} - q^{i-1}) -  D b \frac{q^{i+1}-2q^i+q^{i-1}}{h^2} = 0 \]

\[ \frac{u}{2} (q^2-1) - \frac{D}{h}(q^2-2q+1) = 0 \]

\section{Лекция 2}

Точное решение

\[ Q(x) = \frac{1-e^{\frac{Ux}{\mathcal{D}}}}{1-e^{\frac{U}{\mathcal{D}}}} = \]

\[ Q(x) = \frac{1-e^{Pe \cdot i}}{1-e^{Pe_n N}}=  \]
\[ Q(x) = \frac{1-\overline{q}^i}{1-\overline{q}^N} \]

\[ \overline{q} = e^{Pe_N} = 1+ \frac{Pe_n + Pe_n^2}{2}\]

Сеточная функция (Центральные разности)

\[ q = \frac{1+ \frac{Pe}{2}}{1- \frac{Pe}{2}} =  \]
\[ (1+ \frac{Pe_N}{2}(1+ \frac{1}{2} {Pe}_n + \frac{1}{4}{Pe}^2_n+...)) =  \]
\[ 1+ {Pe}_n + \frac{{Pe}^2_n}{2} \]

Но число пекре должно быть меньше 2

Направленные разности

\[  q + {Pe}_n \]

Для схемы с центральными разностями

\[ U \frac{Q_{i+1} - Q_i}{2h} - \frac{Q_{i+1} - 2Q_{i} - Q_{i-1}}{h^2} = 0\]
\[ Q_{i+1} (\frac{u}{2h} - \frac{\mathcal{d}}{h^2}) + Q_{i} (\frac{2\mathcal{d}}{h^2}) + Q_{i-1} (-\frac{u}{2h} - \frac{\mathcal{d}}{h^2}) = 0 \]
\[ - Q_{i+1} (1 - \frac{1}{2} \mathcal{D}) + 2Q_{i} - Q_{i-1} (1 - \frac{1}{2} {Pe}_n) = 0 \]

Для схеммы с направленными разностями

\[ U \frac{Q_{i} - Q_{i-1}}{h} - \mathcal{D}\frac{Q_{i+1} - 2Q_{i} + Q_{i-1}}{h^2} = 0\]
\[ Q_{i-1} (-\frac{u}{h}-\frac{\mathcal{D}}{h^2}) + Q_i(\frac{u}{h}-\frac{2\mathcal{D}}{h^2}) + Q_i(-\frac{\mathcal{D}}{h^2}) = 0\]
\[ -Q_{i-1} (1+{Pe}_n) ... \]

Попробуем предсказать поведение вышеописанных схем

\[ Q_{i-1} = Q_i - h \frac{\partial q}{\partial x}|_i + \frac{h^2}{2!} \frac{\partial^2 q }{\partial x} - \frac{h^3}{3!} \frac{\partial^3 Q }{\partial x^3}+... \]
\[ Q_{i+1} = Q_i + h \frac{\partial q}{\partial x}|_i + \frac{h^2}{2!} \frac{\partial^2 q }{\partial x} + \frac{h^3}{3!} \frac{\partial^3 Q }{\partial x^3}+... \]

Первое слагаемое:
\[ \frac{U }{h }{(\frac{\partial Q }{\partial x})}_i - \frac{h^2}{2} \frac{\partial^2 Q }{\partial x^2} + \frac{h^3}{3!} \frac{\partial^3 Q }{\partial x^3} ... \]

\[ U \frac{\partial Q}{\partial x}|_i-\frac{1}{2}Uh \frac{\partial^2Q }{\partial x^2} - \mathcal{D} \frac{\partial^2}{\partial x^2}|_i/... = 0 \]
\[ U \frac{\partial Q}{\partial x}|_i - \mathcal{D}(1+\frac{1}{2} {Pe}_n)\frac{\partial^2 Q }{\partial x^2}|_i \]

ДЛя того чтобы коэф диффузии был похож на то что было в исходном уравнении $ \frac{{Pe}_n}{2} \textrm{ << } 1 $

\subsection{Схема А.А.Самарского}

\[ U^+Q_{\overline{x}} + U^- Q_x - \frac{\mathcal{D }}{1 + \frac{{Pe}_n }{2}} Q_{\overline{x}x} = 0 \]

\[ \overline{D} = \frac{\mathcal{D }}{1 + \frac{{Pe}_n}{2}} = \mathcal{D(1- \frac{|{Pe}_n| /2}{1+\frac{|{Pe}_n| }{2}})} \]

\[ u^+ Q_{\overline{x}} - u^-Q_x = u Q_{x0} - \frac{|U|_h}{2}Q_{\overline{x} x}\]

\[ U Q_{x0} - \mathcal{D} [\frac{1}{1+\frac{|Pe|}{2}}+\frac{{Pe}_n }{2}]Q_{\overline{x}x} = 0 \]
- Коэфф диффузии
\[ \mathcal{D} [\frac{1}{1+\frac{|Pe|}{2}}+\frac{{Pe}_n }{2}] = \mathcal{D}^* \]

\[ {Pe}^*_n = \frac{Uh }{\mathcal{D }} \textrm{ - эффективное число Пикле } \]

\[ f = {Pe}^*({Pe}) \]
\[ \mathcal{D}^* = \mathcal{D} \frac{1+\frac{|{Pe}_n|}{2}+\frac{|{Pe}_n|^2}{4}}{1+\frac{|{Pe}_n|}{2}} \]
\[ {Pe}^* = \frac{Uh }{\mathcal{D}} \frac{1+\frac{|{Pe}_n }{2}}{1+\frac{|{Pe}_n|}{2}+\frac{|{Pe}_n|^2}{4}} \]

\[ {Pe}_n \rightarrow 0; {Pe}^*_n \rightarrow {Pe}_n  \]

\[ {Pe}_n \rightarrow \inf; {Pe}^*_n \rightarrow 2 \]

\[ |{Pe}*n| = \frac{{Pe}^2_n (\frac{1}{2} + \frac{1}{|{Pe}_n|})}{{Pe}^2_n (\frac{1}{4}+\frac{1}{2|{Pe}_n|}+\frac{1}{|{Pe}_n|^2})} \]

\subsection{Экспоненциальная схема}

\[ u \frac{dQ }{dx} - \mathcal{D} \frac{d^2Q }{dx^2} = 0 \]
\[ Q(x) = Q_i + {(Q_{i+1} - Q_i)}\frac{{exp({Pe}_{n, i+ \frac{1}{2}}\frac{x-x_i }{h_{i+\frac{1}{2}}})}}{exp({Pe}_{n, i+\frac{1}{2}})-1} \]

\[ W^{tot} = u_{i+\frac{1}{2}} Q_{i+\frac{1}{2}} - \mathcal{D} \frac{dQ}{dx}|_{i+\frac{1}{2}} = \]
\[ = u_{i+\frac{1}{2}} Q_i + u_{i+\frac{1}{2}}(Q_{i+\frac{1}{2}} Q_i) \frac{exp(\frac{{Pe}_{i+\frac{1}{2}}}{2}-1)}{exp({Pe_{n, i+\frac{1}{2}}}-1)} - \]
\[ - u_{i+\frac{1}{2}}(Q_{i+1}-Q_i) \frac{exp(\frac{{Pe}_{i+\frac{1}{2}}}{2})}{exp({Pe_{n, i+\frac{1}{2}}})-1} \]

\[ \frac{W_{i+\frac{1}{2}} - W_{i-\frac{1}{2}}}{h_i} = u Q_{\overline{x}} + u h \frac{e^P \frac{{Pe}_N}{2}-1}{e^{{Pe}_N}-1} Q_{\overline{x}x} - u h \frac{e^P \frac{{Pe}_n}{2}}{e^{{Pe}_n}-1} Q_{\overline{x}x} = 0 \]

\[ u Q_{\overline{x}} = u Q_{x0} - \mathcal{D} \frac{Pe}{2} Q_{\overline{x}x} \]

\[ u Q_{x0} - \mathcal{D}^2 Q_{\overline{x}x} = 0 \]
\[ \mathcal{D}^* = \mathcal{D} (\frac{{Pe}_N}{2} \frac{e^{Pe} + 1}{e^{Pe}-1}) = \mathcal{D} \frac{{Pe}_n }{2} coth\frac{{Pe}}{2} \]

\[ {Pe}_n = \frac{Uh }{\mathcal{D }} = 2 th \frac{{Pn}_n}{2} \]

\subsection{Схема Сполдинга}

\[ {Pe}^*_{h, i+\frac{1}{2}} = Pe_{h, i+\frac{1}{2}}, {Pe}_h \leq 2 \]
\[ = 2, {Pe}_h > 2 \]

\[ \Theta_{i+\frac{1}{2}} = \Theta_{i+\frac{1}{2}} Q_i + (1 - \Theta_{i+\frac{1}{2}}Q_{i+1})\]

\[ \Theta_{i+\frac{1}{2}} = \frac{1}{{Pe_n, i+\frac{1}{2}}}[{Pe_{n, \frac{1}{2}}}-1+max(- {Pe}_{n, i+\frac{1}{2}}, 1-\frac{{Pe}_{n, i+\frac{1}{2}}}{2}, 0)] \]


\begin{enumerate}
	\item
	\[ |Pe_{n, i+\frac{1}{2}}| \leq 2 \]
	\[ \Theta_{i+\frac{1}{2}} = \frac{1}{2} \Rightarrow \] буквально схема с центральными разностями.
	\item
	\[ |Pe_{n, i+\frac{1}{2}}| > 2 \]

	\begin{enumerate}
	\item
	\[ {Pe}_n > 2 \]
	\[ \Theta_{i+\frac{1}{2}} = \frac{1}{{Pe}_{n, i+\frac{1}{2}}} [{Pe}_{n+\frac{1}{2}} -1 + 0] = 1 - \frac{1}{Pe} \]

	\item
	\[ Pe_{n} < -2 \]
	\[ \Theta_{i+\frac{1}{2}} = \frac{1}{{Pe}_{n, i+\frac{1}{2}}} [{Pe}_{n+\frac{1}{2}} -1 - {Pe}_n] = - \frac{1}{{Pe}_{n}} \]

	\end{enumerate}
	Или оба варианта в одной формуле

	\[ \frac{1}{2}(1+ \frac{|{Pe}_n|}{{Pe}_n}) - \frac{1}{{Pe}_n} \]

\end{enumerate}

Подставляем $ \Theta $
\[ Q_{i+\frac{1}{2}} = [ \frac{1}{2} (1 + \frac{|{Pe}_n|}{{Pe}_n}) = \frac{1}{{Pe}_n}] Q_i + [\frac{1}{2} (1 - \frac{|{Pe}_n|}{{Pe}_n} + \frac{1}{{Pe}_n })]Q_{i+\frac{1}{2}} \]

\[ W_{i+\frac{1}{2}} = u Q_{i+\frac{1}{2}} = U^+_{i+\frac{1}{2}} Q_i + U^-_{i+\frac{1}{2}} Q_i+1 + D Q_x \]

\[ \frac{d }{dx }u Q - D \frac{d^2Q }{dx^2} = 0 \]
\[ W_{i+\frac{1}{2}} - W_{i-\frac{1}{2}} = U^+ Q_{\overline{x}} + \overline{u} Q_x + D Q_{\overline{x}x} \]

Таких схем множество, но суть такова, что до определенного момента рассматривается направленные разности, а потом переходят на константу

\subsection{Схема Патанкара}

Надо разобрать к экзамену! Есть в презентации.

\[ \Theta_{i+\frac{1}{2}} = \frac{1}{{Pe}_{n, i \frac{1}{2}}}[{Pe}_{n, i+\frac{1}{2}} - 1 + max(0, {Pe}_n) + max(0, (1 - 0.1 |{Pe}_n|)^5)] \]

Теперь нужно определить в полуцелых точках.

\subsection{Разностная схема на расширенных шаблонах}

\[ Q_{i+\frac{1}{2}} = \frac{1}{2} (Q_{i+1} + Q_{i}) - \eta ( Q_{i+1} - 2 Q_i + Q_{i-1}) \]

\begin{enumerate}
	\item \[ u \frac{d }{dx} = u Q_{x^0} -\frac{U h }{2}( Q_x \overline{x} - Q_{x x})    \frac{1}{2} h^2 Q_{\overline{x} \overline{x} } \]

	\item
	\[ \eta = \frac{1}{2} \textrm{ - схема с направленными разностями } \]

	\item
	\[ \eta = \frac{1}{4} -\textrm{  схема Фрома } \]
	\item
	\[ \eta = \frac{1}{6} -\textrm{  схема с искуственной дисперсией }\]
	\item
	\[ \eta = \frac{1}{8} -\textrm{  схема QUICK  }\]

\end{enumerate}

\section{Лекция 3}

\subsection{Нестационарные задачи}

\[ \frac{\partial Q}{\partial t} + \frac{\partial }{\partial x }(u Q) = \mathcal{D} \frac{\partial^2 Q }{\partial x^2 } \]

\[ u = const, {Pe} = \frac{U L }{\mathcal{D}} >> 1 \]

Сетка вводится аналогично стационарной задаче.

\[ t_k = \tau_k \]

\[ \int_{t_k }^{t_{k+1}}\int_{x_{i-\frac{1}{2}}}^{x_{i+\frac{1}{2}}} \frac{\partial Q }{\partial t} dx dt + \int_{t_k }^{t_{k+1}}\int_{x_{i-\frac{1}{2}}}^{x_{i+\frac{1}{2}}} \frac{\partial}{\partial x} (u Q) dx dt = \int_{t_k }^{t_{k+1}}\int_{x_{i-\frac{1}{2}}}^{x_{i+\frac{1}{2}}} \frac{\partial}{\partial x} (\mathcal{D} \frac{\partial Q }{\partial x }) dx dt \]

По слагаемым
\begin{enumerate}
	\item Выносим, так как не зависит от t, по x константа на отрезке интегрирования
	\[ h_i \cdot (Q^{k=1}_i - Q^k_i) \]
	\item \[ \int_{t_k }^{t_{k+1}}(W_{i+\frac{1}{2}} - W_{i-\frac{1}{2}} dt ) \]
	Воспользуемся квадратурной формулой
	\[ = \tau (W^{\sigma}_{i+\frac{1}{2}} + W^{\sigma}_{i-\frac{1}{2}}) \]
	\[ f^{(\sigma)} = \sigma f^{t_k + 1} + (1-\sigma) f^{t_k} = \sigma \hat{f} + (1-\sigma) f \]
	Возникает ровно та же проблема определения W в полуцелых точках
	\item
	\[ \int_{t_k }^{t_{k+1}} [ \mathcal{D} \frac{\partial Q }{\partial x} |_{i+\frac{1}{2}} - \mathcal{D} \frac{\partial Q }{\partial x} |_{i-\frac{1}{2}}] dt = \]
	\[ \tau [(\mathcal{D} \frac{\partial Q }{\partial x})_{i+\frac{1}{2}}^{(\sigma)} - (\mathcal{D} \frac{\partial Q }{\partial x})_{i-\frac{1}{2}}^{(\sigma)}] \]
\end{enumerate}

\subsection{Решение модельной задачи}

Отрезок от -L до L. Начальные данные: 1 при отрицательных значений, 0 при положительных.

\[ \frac{\partial Q }{\partial t } + u \frac{\partial Q }{\partial x} = \mathcal{D} \frac{\partial^2 Q }{\partial x^2} \]

\[ S = t, y = x - ut \]
\[ Q(x,t) = Q(y(x,t), S(t)) \]
\[ \frac{\partial Q }{\partial t } = \frac{\partial Q }{\partial S} \frac{ Q }{\partial t } + \frac{\partial Q }{\partial y}\frac{\partial y}{\partial t} = \frac{\partial Q }{d S}- \frac{\partial Q }{\partial y}\]
\[ \frac{\partial Q}{\partial y} - u \frac{\partial Q}{i\partial y}+ u \frac{\partial Q }{\partial y } \mathcal{D } \frac{\partial^Q }{\partial y^2} \]
\[ \frac{\partial Q }{\partial S } = \mathcal{D } \frac{\partial^2 Q}{\partial y^2} \Rightarrow\] интеграл Пуассона
\[ Q = \frac{1}{\sqrt{\pi}} [\frac{\sqrt{\pi}}{2} - \int_{0}^{\frac{y}{2 sqrt{\mathcal{D S}}}} e^{-z^2} dz] =  \]
\[ \frac{1}{\sqrt{\pi}} [\frac{\sqrt{\pi}}{2} - \int_{0}^{\frac{x - ut }{2 \sqrt{\mathcal{D }t }}} e^{-z^2} dz] \]
Подставить это дома и убедиться, что оно является решением уравнения.

\subsubsection{Направленные разности.}

**рисунок фронта**

Самое большое размытие при использовании чисто неявной схемы ($ \sigma = 1 $)

Меньше всего размывается явная схема ($ \sigma = 0 $).

То есть неявная схема ухудшает проблемы направленных разностей

\subsubsection{Центральные разности}

**рисунок фронта

Чисто неявная схема снова размывает сильнее.

($ \sigma = 0.5 $)

\subsection{Диффузия и дисперсия}

Дисперсия - скорость убывания волны зависит от амплитуды.

\[ \frac{\partial Q }{\partial t} + u \frac{\partial Q }{\partial x } = \mathcal{D } \frac{\partial^2 Q }{\partial x^2 } + \beta \frac{\partial^3 Q }{\partial x^3 } + \gamma \frac{\partial^4 Q }{\partial x^4} \]

\[Q(t, x) = e^{-P(m)t} e^{i m (x - q(m) t)} \]
m - волновое число,
P(m) - скорость убывания амплитуды,
q(m) - скорость распространения волны,
$ i^2 = -1 $.

\[ Q = e^{-P t } e^{im(x-qt)} \]

\[ \frac{\partial Q }{\partial t} + u \frac{\partial Q }{\partial x } = \mathcal{D } \frac{\partial^2 Q }{\partial x^2 } + \beta \frac{\partial^3 Q }{\partial x^3 } + \gamma \frac{\partial^4 Q }{\partial x^4} \]

Подставим Q и посмотрим каими должны быть P и m, для того чтобы оно удовлетворяло уравнениям.

\[ -P - i m q + U i m = - \mathcal{D} m^2 - i m^3 p + \gamma m^4 \]

\[ P = \mathcal{D }m^2 - \gamma m^4 \] - скорость убывания
\[ q = u + M^2 \beta \] - скорость волны

На амплитуду влияют четные производные ( но они чередуются уменьшает-увеличивает-уменьшает-...)

скорость движения от третьей производной (нечетныепроизводные) проверить с каким знаком.

\subsection{Дифференциальное представление разностных схем}

\[ L(Q) = 0 \textrm{ Дифференциальное уравнение} \]
\[ L^{\tau}_h (Q, \tau, h) = 0 \textrm{ - Разностная схема } \]
\[ L_H^{\tau} (Q) = L(Q) + \epsilon(Q, \tau, h) = 0 \textrm{ - Погрешность аппроксимации } \]

Не будем убирать погрешность и будем его рассматривать как бесконечный ряд.

\[ L_h^{\tau} (Q) = L(Q) + \sum_{i=0}^{\infty} \alpha_i \frac{\partial Q }{\partial x^i } = 0 \]

Процедура по созданию подобного представления (в общих чертах):

\subsection{Дифференциально приближение для схемы с направленными разностями.}

\[ Q_t + u Q^{(\sigma)}_{\overline{x}} - \mathcal{D} Q^{(\sigma)}_{\overline{x}} x = 0 \]

\[ Q^{(\sigma)} = Q^{0.5} + (\sigma - 0.5) \tau Q_t = \frac{\hat{Q} + Q}{2} + (\sigma - 0.5) (\hat{Q} Q)\]

\[ Q_t + u Q^{(0.5)}_{\overline{x}} - \mathcal{D} Q^{(0.5)}_{\overline{x} x} + u(\sigma - 0.5) \tau Q_{t \overline{x }} - \mathcal{D }(\sigma-0.5) Q_{t \overline{x} x} \]

Разложим в точке $ (t_k + \frac{\tau }{2}, x_i Q (t_k + \frac{\tau }{2}) Q^(0.5)) $

\[ Q_t = \frac{\partial \overline{Q }}{\partial t } + \frac{t^2}{2} \frac{\partial^3 \overline{Q  }}{\partial t^3 } + \dots \]

\[ Q^{(0.5)} = \overline{Q} +\frac{\tau^2}{8} \frac{\partial^2 Q}{\partial x ^2} + \dots \]

и тд.

Как избавиться от высоких производных и смешанных производных по t и x. Метод дифференциального представления.

\section{Лекция 4}

\subsection{Устойчивость разностных схем. Метод Неймана (Метод гармоник.)}

\[ Q(t, x) = e^{-p(m)t} e^{Im(x-qt)} \textrm{ - Бегущая волна } = a(t) e^{I m x}\]

\[ a(t) = e^{-t [ p(m) + I q m]} \textrm{ Комплексная амплитуда }\]

\[ p(m) = \mathcal{D} m^2, q=u \textrm{ Для нашего конкретного уравнения } \]

\[ Q(t + \tau, x) = e^{-t + \tau}[\mathcal{D}m^2 + (I m q)] e^{Imx} = G Q (t,x) \]

G - Множитель перехода

\[ G = e^{- \tau (\mathcal{D} m^2 = I m u)}  \]

\subsubsection{Явная схема с направленными разностями}

\[ Q^{k+1}_i = Q^k_i - c(Q^k_i - Q^k_{i-1}) + S (Q^k_{i+1} - 2 Q_i^k + Q^k_{i-1}) \]

\[ c = \frac{u \tau}{h} \textrm{ - число Куррента }, S = \frac{\mathcal{D } \tau }{h^2} \textrm{ - Тепловое число Куррента }\]

\[ Q^k_i = Q (t_k, x_i) = e^{-pt_k} e^{Im [x_i - q t_k]} = \]
\[ = e^{\tau k [ p - I m q]} e^{I m h i} =  \]
\[ = a_k e^{I \Theta i}\]

\[ \tau k = t_k, ih = x_i, \Theta = m h \]

Надо выяснить при каких P и Q при подстановке бегущей волны данное уравнение является решением нашего уравнения.

\[ a_{k+1} e^{I \Theta i} = a_k e^{I \Theta i} - c a_k (e^{I \Theta (i-1)}) + S a_k (e^{I \Theta (i-1)} - 2e^{I \Theta i }) + e^{I \Theta(i+1)} \]

\[ e^{e m i}; a_{k+1} = a_k G \]

\[ G = 1 - c(1-e^{-I \Theta}) + S (e^{I \Theta} - 2 + e^{-I \Theta}) \label{eq:f1} \]

\[ c(1-e^{-I \Theta}) = 1 - cos \Theta + i sin \Theta \]

\[ (e^{I \Theta} - 2 + e^{-I \Theta}) = 2 cos \Theta - 2 \]

Напоминание:
\[ e^{i \phi} = cos \phi + i sin \phi \]
\[ e^{-i \phi} = cos \phi - i sin \phi \]
-----------------------------------------

\[ \ref{eq:f1} = 1 - (c+2S)(1-cos \Theta) + I c sin \Theta \textrm{ Множитель перехода со слоя на слой } \]

Схема устойчива тогда, когда $ |G|>1 $
\[ \Theta = m h \]
\[ \lambda_{min} = 2h = \frac{2 \pi}{m} \textrm{ - минимальная длинна волны } \]

\[ \Rightarrow mh = \pi \textrm{ Короткие волны }\]

\[ mh \rightarrow 0 \textrm{ - Длинные волны } \]

\[ 1 - (c + 2S ) (1- cos \Theta ) - I c sin \Theta \]

\[ {|G|}^2 = 1 - 4 (c+2S) sim^2 \frac{\Theta}{2} + {(c+2S)}^2 \cdot 4 {sin}^4 \frac{\Theta   }{2} + 4 c^2 {sin}^2 \frac{\Theta   }{2} {cos}^2 \frac{\Theta   }{2} < 1\]

\[ -(c+2S) + {(c+2S)}^2 {sin}^2 \frac{\Theta   }{2} + c^2 {cos^2} \frac{\Theta}{2} < 0 \]

\[ -(c+2S) ({sin}^2 \frac{\Theta}{2} + {cos}^2 \frac{\Theta}{2}) + {(c+2S)}^2 {sin}^2 \frac{\Theta   }{2} + c^2 {cos^2} \frac{\Theta}{2} < 0 \]

\[ (c+ 2S) [ (c+ 2S) - 1] {sin}^2 \frac{\Theta}{2} + [c^2- (c+2S)] {cos}^2 \frac{\Theta    }{2} < 0; \forall \Theta \in [0, \pi] \]

\[ c+2S - 1 < 0; c^2 - (c + 2S) < 0 \textrm{ - Условия устойчивости }\]

Метод Неймана дает необходимое условие устойчивости, но вообще говоря не достаточное.

\subsection{Изменение амплитуды волны.}

\[ {|G|}^2 = 1+ 4( c + 2S) {sin}^2 \frac{\Theta    }{2} = 4 (c+ 2S )^2 {sin}^2 \frac{\Theta    }{2} {cos}^2 \frac{\Theta   }{2} \]

Для волны $ \Theta = mh \rightarrow 0 $

\[ {|G|}^2 = 1 - 4 (c + 2S) \frac{{(mh)}^2}{4} + 4 (c +2S)^2 \frac{{(mh)}^4}{2}^4 + 4c^2 \frac{{(mh)}^2}{4} = 1 + {(mh)}^2 [ c^2 - (c+ 2S)] =  \]
\[ = 1 - (mh)^2 [ c+2S - c^2] \]
\[ |G| = 1 - {(mh)}^2 (S + \frac{1}{2} c(1-c)) \label{eq:1*} \]

\[ |G| = 1 - (m^2 h^2) (\frac{\mathcal{D} \tau}{h^2} + \frac{1}{2} c (1-c)) =  \]
\[ = 1 - m^2 \tau \mathcal{D} - \frac{1}{2} m^2 h^2 c (1-c) \]

\[ 1 - m^2 \tau \mathcal{D}\textrm{ То как убывает волна в исх уравнении } \]
\[ \frac{1}{2} m^2 h^2 c (1-c) \textrm{ Добавка разностной схемы } \]

\[ |G| = e^{-\tau \mathcal{D} m^2} = 1 - \tau \mathcal{D} m^2 \]

\subsubsection{Коротко про устойчивость схемы с центральными разностями}

\[ Q_t + u Q_{x^0  \mathcal{D }Q_{\overline{x }x}} \]
\[ Q^{k+1}_i = Q^k_i - \frac{1}{2} c (Q^k_{i_k} - Q^k_{i-1}) + S(Q_{i-1}^k - 2 Q^k_i + Q_{i+1}^k) \]

\[ G = 1 - \frac{c }{2} (e^{I \Theta} - e^{-I \Theta}) + S(e^{-I\Theta} - 2 + e^{I\Theta}) = \]
\[ = 1- 4 S {sin}^2 \frac{\Theta}{2} -2 c I sin \frac{\Theta}{2} cos\frac{\Theta}{2} \]

\[ {|G|}^2 = 1 - 8 S {sin}^2 \frac{\Theta}{2} + 16 S^2 {sin}^4 \frac{\Theta}{2} + 4c^2 {sin}^2 \frac{\Theta}{2} {cos}^2 \frac{\Theta}{2} < 1\]

\[ -2S+4S^2 {sin}^2 \frac{\Theta}{2} + c^2 {cos}^2 \frac{\Theta}{2} < 0 \]

\[ {sin}^2 \frac{\Theta}{2} [4S^2 - 2S] + {cos}^2 \frac{\Theta}{2}[c^2 - 2S] < 0 \]
\[ 2S^2 - S < 0; c^2-2S < 0 \textrm{ - Условие устойчивости } \]


Рассмотрим для длинных волн
\[ {|G|}^2 = 1 - 8 S {sin}^2 \frac{\Theta}{2} + 16 S^2 {sin}^4 \frac{\Theta}{2} + 4c^2 {sin}^2 \frac{\Theta}{2} {cos}^2 \frac{\Theta}{2} < 1\]

\[ \approx = 1 - 8 S \frac{{(mh)^2}}{4}+ 4 c^2\frac{{(mh)}^2}{4} =  \]
\[ = 1 - {(mh)}^2(2S-c^2) \]

\[ |G| \approx 1 - {(mh)}^2 S + {(mh)}^2 \frac{c^2}{2} =  \]
\[ 1 - \frac{m^2 h^2 \mathcal{D} \tau}{h^2} + m^2 h^2 \frac{c^2}{2} \]

Длинные волны убывают медленнее чем в дифференциальном случае. (В прошом примере было наоборот)

дз решить задачи из таблицы (хотя бы некоторые), позже возможно следует ей прислать.

\section{Лекция 5}

\subsection{Уравнение движения вязкой несжимаемой жидкости}

\begin{equation}\label{eq:5_1}
\rho [\frac{\partial V }{\partial t } + (v \cdot \triangledown)v] = - \triangledown p + \chi \triangle V = g \rho
\end{equation}
\begin{equation}\label{eq:5_2}
div V = (\triangledown \cdot v) = 0
\end{equation}
\begin{equation}\label{eq:5_3}
\rho = \rho(p, t) = \rho_0 (1 + \beta T)
\end{equation}

V - вектор скорости
p - Давление
$ \chi $ - коэффициент вязкости

\[ \frac{D \rho}{D x} = \{ \textrm{ Полная производная  } \} = \frac{\partial \rho}{\partial t} + V_x \frac{\partial \rho}{\partial x} + V_y \frac{\partial \rho}{\partial y} + V_z \frac{\partial \rho}{\partial z} \]

\[  \frac{D \rho}{D x} + \rho div V = 0 \]
\[ = \frac{\partial \rho}{\partial t} + V_x \frac{\partial \rho}{\partial x} + V_y \frac{\partial \rho}{\partial y} + V_z \frac{\partial \rho}{\partial z} + \rho div V = 0\]

Жидкость несжимаема тоже самое, что полная производная равна 0, это одновременно означает что дивергенция равна 0.
\[ \frac{D\rho}{Dt} = 0 \Rightarrow div V = 0 \]

Перепишем уравнение \ref{eq:5_1} в виде:
\[ \frac{\partial V}{\partial t} + (V \cdot \triangledown) V = - \triangledown \frac{p}{\rho_0} + \nu \triangle V + F \]

\begin{equation}\label{eq:5_4}
\eta = \frac{\chi}{\rho_0} - \textrm{ Кинетическая вязкость }
\end{equation}

Из \ref{eq:5_4} Исключаем давление.

\[ \Delta V = - rotrot V; \]
\[ (V \cdot \triangledown) V = rot V \times V + \triangledown(\frac{V^22}{2}) \Rightarrow \]
\begin{equation}\label{eq:5_5}
\frac{\partial V}{\partial t} + rot V \times V = -\triangledown (\frac{p}{\rho_0} + \frac{V^2}{2}) - \nu rot rot V + F
\end{equation}

Применим ротор к \ref{eq:5_5}
\begin{equation}\label{eq:5_6}
\frac{\partial \Omega}{\partial t} + (V \cdot \triangledown) V = \nu u \triangle + rot F
\end{equation}

пусть $ V = (V_x(t,x, y)) $

\section{Лекция 5}

\[ \frac{\partial \omega}{ \partial t } + \frac{\partial}{\partial x} (u \omega) + \frac{\partial}{\partial y }(V \omega) = \nu  \Delta \omega\]

\[ (u \omega) = W^x \]
\[ (V \omega) = W^y \]

\[ u = \frac{\partial \psi}{\partial y}, V = -\frac{\partial \psi}{\partial x} \]
\[ div V = 0 \]

\[ \triangle \psi = -\omega \]

\[ \psi, \omega \textrm{ в узлах сетки }, u(i, j+\frac{1}{2}), V(i+\frac{1}{2}, j) \]
Тогда дивергенция по ячейке автоматически равна нулю. Ячейки называются нулю.
\[ div V|_{S_{i+\frac{1}{2}, j+\frac{1}{2}}} \]

Ячейка с центром в U $ S(i, j+\frac{1}{2}) $
Ячейка с центром в V $ S(i+\frac{1}{2}, j) $

Аппроксимация производной nо $ \omega $

\[ \frac{\partial \omega}{\partial t} \]

\[ \int_{S_{i,j}}^{} \frac{\partial\omega}{\partial t} dx dy + \int_{S_ij}^{}(W^x + \frac{\partial}{\partial y } W^y) dx dy = \mu \int_{S_{ij}}^{} \triangle \omega dx dy \]

$ \overline{S} $ означает площадь
\[ \frac{\partial}{\partial t} [\omega_{ij}] \overline{S_{ij}} = \frac{\partial}{\partial t} \omega_{ij} \overline{h_i^x} \overline{h}^y_i \]

\[ W = (W^x, W^y) \]

\[ I_2 = \int_{S_{ij}}^{} (\frac{\partial}{\partial x} W^x +\frac{\partial}{\partial y } W^y) dx dy = \int_{\partial S_{ij}}^{} (w, n )dl = \]
\[ = W^x_{i+\frac{1}{2}, j} \overline{h}^y_i + W^y_{ij+\frac{1}{2} \overline{\overline{h}^x_i}} - W^x_{i-\frac{1}{2}, j} \overline{h}^y_i - W^y_{ij-\frac{1}{2} \overline{\overline{h}^x_i}}\]
Делим на $ \frac{1}{\overline{h^x_i} \overline{h}^y_i} $

\[ \frac{W^x_{i+\frac{1}{2}} - W^y_{i-\frac{1}{2}}}{\overline{h_i}} \]

\[ u_ij = \frac{h^y_{i, j+\frac{1}{2}} u_{ij+\frac{1}{2}} + h^y_{ij-\frac{1}{2}} u_i, j-\frac{1}{2}}{2h^y_j} \]
\[ V_ij = \frac{h^x_{i+\frac{1}{2}, j} V_{i+\frac{1}{2},j} + h^x_{i-\frac{1}{2}, j} V_{i, j-\frac{1}{2}}}{2h^y_j} \]

\[ u_{ij} = \psi_{y^0}= \frac{\psi_{i, j+1 - \psi_{i, j-1}}}{2 \overline{h}^y_j}, V_{ij} = \psi_{x^0}= \frac{\psi_{i+1, j - \psi_{i-1, j}}}{2 \overline{h}^x_j} \]

\[ W^x_{i+\frac{1}{2}, j} = {(u \omega)}_{i\frac{1}{2}, j} \]

\subsection{Переинтерполяции}
\subsubsection{Переинтерполяция 1}

\[ W^x_{i+\frac{1}{2}} = \frac{1}{2} (u_{i+1, j} \omega _{i+1j} + u_{ij} + \omega_{ij}) \]
\[ W^y_{ij+\frac{1}{2}} = \frac{1}{2} (V_{ij+1} \omega_{ij+1} + V_{ij} \omega_{ij}) \]

\subsubsection{Переинтерполяция 2}

\[ W^x_{i+\frac{1}{2}j} = \frac{u_{i+1, j}+u_{ij}}{2} + \frac{\omega_{i+1,j}\omega_{ij}}{2} \]

\[ W^y_{i+\frac{1}{2}j} = \frac{V_{i,j+1}+V_{ij}}{2} + \frac{\omega_{i,j+1}\omega_{ij}}{2} \]

\subsubsection{Первый способ применение}
Начнем с первого способа.

\[ \mathcal{K}_h(\psi, \omega) \overline{h}^x_i \overline{h}^y_j = (W^x_{i+\frac{1}{2}} - W^x_{i-\frac{1}{2}, j}) \overline{h}^y_i + (W^y_{ij+\frac{1}{2}} - W^y_{i,j-\frac{1}{2}}) \overline{h}^x_i \]

\[ u_{i+1,j} = \psi_{y^0}(+1_x); u_{ij} = \psi_{y^0} \]
\[ W^x_{i+\frac{1}{2}, j} =  \frac{1}{2}[\psi_{y^0} (+1_x) \omega_{i+1j} + \psi_{y^0} \omega_{ij}] \]

\[ W^x_{i-\frac{1}{2}, j} = \frac{1}{2}[\psi_{y^0} \omega_{i+1j} + \psi_{y^0}(-1_x) \omega_{i-1, j}] \]

\[ W^x_{i+\frac{1}{2}, j} - W^x_{i-\frac{1}{2}, j} \]

\[ \mathcal{K}_h(\psi, \omega) \overline{h}^x_i \overline{h}^y_j = (W^x_{i+\frac{1}{2}} - W^x_{i-\frac{1}{2}, j}) \overline{h}^y_i + (W^y_{ij+\frac{1}{2}} - W^y_{i,j-\frac{1}{2}}) \overline{h}^x_i =  \]
\[ \frac{1}{2} [\psi_{y_0}(+1+x) \omega_{i+1j} - \psi_{y^0} \omega_{i-1,j}] \overline{h^y_i} - \frac{1}{2} [\psi_{x^0}(+1_y) \omega_{ij+1} - \psi_{x^0} - \psi_{x^0}(-1_y) \omega_{ij-1}] \overline{h}^x_i \]

\[ \mathcal{K}_h (\psi, \omega) = (\psi_{y^0} \omega)_{x^0} - (\psi_x^0 \omega)_{y^0} \]

Теперь осталось проинтегрировать оператор лапласа.

\[ I = \nu \int (\frac{\partial^2 \omega }{\partial x^2} + \frac{\partial^2 \omega}{\partial y^2}) dx dy = \int_{\partial S_{ij}}^{} ((grad \omega, n) dl) \]

Еще есть условие, его тоже надо аппроксимировать

\[ \omega = - \Delta \psi \]

\[ \int_{S_{ij}}^{} \omega dx dy = - \int_{S_{ij}}^{} (\frac{\partial^2 \psi}{\partial x^2}) (\frac{\partial^2 \psi}{\partial y^2}) dx dy  = - \int_{\partial S_{ij}}^{} (grad \psi, n) dl \]

\[ = -[{(\frac{\partial \psi}{\partial x})}_{i+\frac{1}{2}}^y + (\frac{\partial \psi}{\partial y}_{ij+\frac{1}{2} \overline{h}^x_i}) -(\frac{\partial \psi}{\partial x})_{i-\frac{1}{2}j} \overline{h}^y_i - (\frac{\partial \psi}{\partial y})_{i, j-\frac{1}{2}} \overline{h}^x_i]\]

\[ \frac{\partial \psi}{\partial x} |_{i+\frac{1}{2}, j} = \frac{\psi_{i+1,j} - \psi_{ij}}{h^x_{i+\frac{1}{2}}} \frac{\psi_{i, j+1} - \psi_{ij}}{h^y_{j+\frac{1}{2}}} \]

Но что происходит на границе?

\[ \omega_{ij} S_{i0} = [(\frac{\partial \psi}{\partial x})_{i+\frac{1}{2}, 0} \overline{h}^y_0 + \frac{\partial \psi}{\partial y}_{i\frac{1}{2}} \overline{h}^2_i - \frac{\partial \psi}{\partial x}_{i-\frac{1}{2},0} \overline{h}^h_0 - (\frac{\partial \psi}{\partial y})_{i,0} \overline{h}_i^x]\]

\[ S_{i0} = \overline{h}^x_i \overline{h}^y_0\]
\[ \overline{h}^y_0 = \frac{h^y_\frac{1}{2}}{2} \]

\[-[{(\frac{\partial \psi}{\partial x})}_{i+\frac{1}{2}}^y \overline{h}^y_0 + {(\frac{\partial \psi}{\partial y})}_{ij+\frac{1}{2}} \overline{h}^x_i -(\frac{\partial \psi}{\partial x})_{i-\frac{1}{2}j} \overline{h}^y_i - (\frac{\partial \psi}{\partial y})_{i, j-\frac{1}{2}} \overline{h}^x_i]\]

$ \psi $ на границе равно нулю $ \Rightarrow $

\[ \omega_{ij} \overline{h}^x_i \overline{h}^y_0 = -(\frac{\partial \psi}{\partial y})_{i,j+\frac{1}{2}} \overline{h}^x_i \]
\[ \omega_{ij} = -\frac{2}{h^y_{\frac{1}{2}}} \frac{(\psi_{i1} - \psi_{i0})}{h^y_{\frac{1}{2}}} \]
\[ \omega_{Ij} = -\frac{2}{{(h^y_\frac{1}{2})}^2} \psi_{i1} \textrm{ - итоговое граничное условие (условие Тома (Thom)) } \]

\subsection{Баланс кинетической энергии}

Переобозначим скалярные произведения

\[ f(x_i, y_i) \rightarrow {(f,g)}^0 = \sum_{(i,j) \in \overline{I} \times \overline{J}}^{} f_{ij g_{ij}S_{ij}} = \sum_{i=0}^{N_x}\sum_{j=0}^{N_y} f_{ij} g_{ij} \overline{h}^x_i \overline{h}^y_j \]

\[ (f,g)_h^{(1)} = \sum_{i=0}^{N_x} \sum_{j=0}^{N_{y-1}} f_{ij+\frac{1}{2}} g_{ij+\frac{1}{2}} \overline{h}^x_i \overline{h}^y_j+\frac{1}{2} \]

\[ {(f,g)}^{(2)}_h = \sum_{i=0}^{N_x-1} \sum_{j=0}^{N_{y-1}} f_{i, j+\frac{1}{2}} g_{i+\frac{1}{2}} h^x_{i+\frac{1}{2}} \overline{h}^y_j \]


\section{Лекция 8}

\[ \nu(\Delta \omega, \psi)(\omega, \omega) \]

\[ (\Delta, \psi)^0_w \sum_{(i,j)}^{} \not{h}_i^x \not{h}_j^y(\omega_{\overline{x}x}+\omega_{\overline{y}y}) \psi_{ij}= \]

\[ = \sum_{j=1}^{N_y=1} \not{h}^y_u\sum_{i=1}^{N_x-1} (\omega_x - \omega_{\overline{x}} \psi_{ij}+...) = \]

\[ = \sum_{j=1}^{N_y} \not{h}_j^y [\sum_{i=1}^{N_x-1} \omega_x \psi_{ij}-\sum_{i=0}^{N_x-1}\omega_x \psi_{i+1j}] = \]

\[ = \sum_{j=1}^{N_y-1} \not{h}^y_j [\sum_{i=1}^{N_x-2} \omega_x \psi_x h^x_{i+\frac{1}{2}}+ \omega_x (N_x-1)\psi_{N_x-1,j}-\omega(0, j) \psi_{1j}] = \]

\[ \omega_x \psi_x h_{i+\frac{1}{2}}^x (\omega_x (N_x-1) \psi_{N_x-1} - \omega_x(N_x-1) \psi_{N_x}) = -\omega_x (N_x-1) (\psi_{N_x})- ... \]

\[ = - \sum_{j=1}^{N_y-1} \not{h}^y_j \sum_{i=1}^{N_x-1} \omega_x \psi_x h^x_i+\frac{1}{2} = \{\omega = - \Delta \psi = - (\psi_{\overline{x}x}+ \psi_{\overline{y}y})\} = \]

\[ = - \sum_{j=1}^{N_{y-1}} \not{h}^y_i \sum_{i=1}^{N_x-1}(\omega_{i+1}j-\omega_ij)\psi_x = \sum_{j=1}^{Ny-1} \not{h}^y_i [\sum_{i=1}^{N_x}\omega_{ij} \psi \overline{x} - \sum_{i+0}^{N_x-1} \omega_{ij} \psi_{x}] =  \]

\[ = - \sum_{j}^{} \not{h}^y_j \sum_{1}^{N-x-1} \omega_{ij} (\psi_{\overline{x}}- \psi_x) \]
\[ \psi_{x \overline{x}} = (\psi_x - \psi_{\overline{x}}) \frac{1}{h} \]

\[ = \sum_{0}^{} \not{h}^y_j\sum_{i=1}^{N_x-1} \omega_{ij} (\psi_x - \psi_{\overline{x}}) + \omega_{N_xj} \psi_{\overline{x}}(N_{xj}) - \omega_{0j} \psi(0,j) = \]

Из граничных условаий следует (буквальо есть омеги на границе вправа).

\[ = -\sum \omega^2_ij \not{j}^x_i \not{h}_j^y \]

\subsection{Уравнение теплопроводности}

\[ \frac{\partial T }{\partial  t } \frac{\partial }{\partial  x} (uT) + \frac{\partial  }{\partial  y} (\nu T) =\textrm{  \ae } \Delta T \]

\[ \int_{\Omega}^{} \frac{1}{2} \frac{\partial }{\partial t} (T^2) dx dy + \int_{\Omega}^{} \frac{\partial }{\partial x}(u T)\cdot T)dxdy = \textrm{ \ae } \int_{\Omega}^{} \Delta T dx dy =  \]

\[ \mathcal{K(T)} = -\int_{\Omega}^{}(uT\frac{\partial T}{\partial x} + \nu T \frac{\partial T}{\partial y}) dx dy = -\frac{1}{2} \int_{\Omega}^{} (u \frac{\partial }{\partial x}(T^2))dx dy =\frac{1}{2} \int_{\Omega}^{} (T^2 \frac{\partial u}{\partial x }+ T^2\frac{\partial \nu}{\partial y})dxdy = \frac{1}{2} \int_{\Omega}^{} T^2 div V dx dy = 0 \]

\[ \overset{\omega}{W}^x_{i+\frac{1}{2}j} = \frac{1}{2} (u_{i+1j}\omega_{i+1} + u_ij \omega_{ij}) \] - та самая хорошая интерполяция.

\[ overset{T}{W}^x_{i+\frac{1}{2}j} = \frac{u_{i+1j}+u_{ij}}{2} \frac{T_{i+1j}+T_{ij}}{2} \]

\[ \frac{1}{4} \sum_{ij}^{} \not{h}^y_{j} [T_{ij}(\psi_{overset{0}{y}(+1_x)}+\psi_{\overset{0}{y}})(T_{i+1j}+T_{ij})-T_{ij}(\psi_{\overset{0}{y}+}+\psi_{\overset{0}{y}(-1_x)}) (T_{ij}+T_{i-1j})] \]

\[ \frac{1}{4} \sum_{}^{} \not{h}^x_i [(\psi_{\overset{0}{x}}+\psi_{\overset{0}{x}}) (T_{ij+1}+T_{ij}) - T_{ij}(\psi_{\overset{0}{x}} \psi_{\overset{0}{x}}(-1y))(T_{ij} + T_{ij-1})] \]


\[ \sum_{}^{} = \frac{1}{4}\sum_{}^{ij} T_{i-1j} T_{i-1j} [\psi_{\overset{0}{y}}(T_{ij} + T_{i-1j})+ T_{ij} \psi_{\overset{0}{y}}(T_{i+1}+T_{ij})] - \frac{1}{4} \sum_{ij}^{}[T_ij \psi_{\overset{0}{y}}(T_{ij}+T_{i-1j}) + T_{ij}\psi_{\overset{0}{y}(ij)}(T(_{i+1j}+T{ij}))] = \]

Сгруппируем слогаемые из скобок попарно (первое с первым, второе со вторым). В формуле выше опущены шаги сетки.

\[ = \psi_{\overset{0}{y}}(T_{ij}+T_{i-1j})(T_{i-1j}-T_{ij}) + \psi_{\overset{0}{y}} (T_{i+1j} + T(ij))(T_{ij-T_{i+1j}}) = \]

Разность квадратов в обеих скобках.

\[ = \frac{1}{2}\sum_{}^{} \psi_{\overset{0}{y}} (T^2_{i-1j} - T^2(i+j)) = -\frac{1}{2}\psi_{\overset{0}{y}}(T^2)_(\overset{0}{x}) = \]

\[ = \sum_{}^{(1)}+\sum_{}^{(2)} = \frac{1}{2} \sum_{}^{} \not{h}_i^x \not{h}_i^y (- \psi_{\overset{0}{y}} T^2_{\overset{0}{x}}+\psi_{\overset{0}{x}} T^2_{\overset{0}{y}}) = \frac{1}{2} \sum_{ij}^{} \not{h}^x_i \not{h}^y_i (\psi_{overset{0}{x}\overset{0}{y}} \psi_{overset{0}{y}\overset{0}{x}}) \]

\subsection{Уравнение Навье-Стокса в естественных переменных}

\[ \frac{\partial V}{\partial t} + (V \cdot \triangledown)V = - \triangledown \rho \nu \Delta V \]
\[ did V = 0 \]

\[ \triangledown = (\frac{\partial}{\partial x_1}+\frac{\partial }{\partial x_2} + \frac{\partial }{\partial x_3}) \]
\[ \Delta = \frac{\partial ^2}{\partial ^2 x_1} + \frac{\partial ^2}{\partial ^2 x_2} + \frac{\partial ^2}{\partial ^2 x_3} \]

\[\frac{ \partial V_i}{\partial t}+ v_j \frac{\partial  v_i}{\partial  x_j} = - \frac{\partial  \rho}{\partial x_j} + \nu \Delta v_i \]

\[ v_j \frac{\partial  v_i}{ \partial  x_j} = V_1 \frac{\partial V_i}{\partial x_1} + V_2 \frac{\partial V_i}{\partial x_2} + V_3 \frac{\partial V_i}{\partial x_3} =  \]
\[ \frac{\partial }{\partial x_1} (V_1 V_i) + \frac{\partial }{\partial x_2} (V_2 V_i) + \frac{\partial }{\partial x_3} (V_3 V_i) = \]
\[ V_1 \frac{\partial V_i}{\partial x_1} + V_i \frac{\partial V_1}{\partial  x_1} + V_2 \frac{\partial V_i}{\partial x_2} + V_i \frac{\partial V_2}{\partial  x_2} + V_3 \frac{\partial V_i}{\partial x_3} + V_i \frac{\partial V_3}{\partial  x_3} \]

\[ V (\frac{\partial V}{\partial  t} V) + (\mathcal{K}(V) \circ V \cdot V) = \]
\[ = -(\triangledown \rho V) + \eta (\Delta V, V) \Rightarrow \frac{d}{d t} \int_{\Omega}^{} E_{kin} d\Omega + \sum_{i=1}^{3} (\mathcal{K}(V)\circ V_i, V_i) d\Omega = \eta \sum_{}^{}(\Delta V_i, V_i) \]

\begin{enumerate}
	\item \[ \frac{\partial \rho}{\partial  x_i} V_i dx_1 dx_2 dy_2  = -\sum_{i=1}^{3}p \frac{\partial V_i}{\partial  x_i} dx_1 dx_2 dx_3 = \rho (\frac{\partial V_1}{\partial  x_1}+\frac{\partial V_2}{\partial x_2}+\frac{\partial V_3}{\partial x_3}) dx_1 dx_2 dx_3 = 0\]
\end{enumerate}

\subsection{Вклад конвективных членов}

\[ \int_{\Omega}^{}\sum_{i=1}^{3}[ \frac{\partial }{\partial x_1} (V_i V_2) + \frac{\partial }{\partial x_2(V_i V_2) + \frac{\partial }{\partial x_3(V_i V_3)}}] V_i dx_1 dx_2 dx_3 = \]

\[ -\frac{1}{2} \int_{\Omega}^{}( \sum_{i=1}^{3} V_1 \frac{\partial }{\partial x_1}(V_i^2)+V_2 \frac{\partial }{\partial x_2}(V_i^2)+V_3 \frac{\partial }{\partial x_3}(V_i^2)) dx_1 dx_2 dx_3 =  \]

\[ = \frac{1}{2} \int_{\Omega}^{} \sum_{i=1}^{3} V_i^2 \div V dx_1 dx_2 dx_3 = 0 \]

Конвективный член не вносит изменений (ошибок)

\[ \sum_{i=1}^{3} \int_{\Omega}^{} \Delta V_i \cdot V_i dx_1 dx_2 dx_3 = - \sum_{i=1}^{3} \int_{\Omega}^{} [ (\frac{\partial V_i}{\partial x_1})^2+ (\frac{\partial V_i}{\partial x_2})^2 + (\frac{\partial V_i}{\partial x_3})^2] \]

Введем сетку

\[ V_1 = U \]
\[ V_2 = V \]
\[ x_1 = x \]
\[ x_2 = y \]

Аппроксимация дивергенции скорости V

\[ \div V \]
\[ \int_{S_{i+\frac{1}{2} j+\frac{1}{2}}}^{} \div V dx dy = \]
\[ = \int_{\partial  S_{i+\frac{1}{2} j+\frac{1}{2}}}^{} (V, n)dl = \]
\[ u_{i+1 j + \frac{1}{2}} h^y_{J+\frac{1}{2}} + V_{i+\frac{1}{2} J+1} h^x_{i+\frac{1}{2}} - u_{ij+\frac{1}{2}} h^y_{j+\frac{1}{2}} - V_{i+\frac{1}{2}j} h^x_{i+\frac{1}{2}} \]

\[ \div V = \frac{U_{i+1j+\frac{1}{2} - u_{ij+\frac{1}{2}}}}{h^x_{i+\frac{1}{2}}} + \frac{V_{i+\frac{1}{2}-V_i+\frac{1}{2}j}}{h^y_j\frac{1}{2}} = 0 \]

\section{Лекция 10}

В лекции 9 написали аппроксимации уранения импулься и интегрировали по ячейке.
Переинтерполировали так чтобы дивергенция внутри ячейки была равна 0.

Вывод разностного уравнения для второй компоненты в разностном случае.

Повторили построение аппроксимации.

Уравнение

\[ \frac{\partial V}{\partial t} + ... - grad p + \nu \Delta V \]
\[ \div V = 0 \]

$u, V, P$

На давление нет уравнения. Обычно используют предисктор-корректор, не очень хороший споб, но лучше нет.

\begin{enumerate}
	\item Предиктор скорости
	\[ \frac{\widetilde{V} - V^k}{t} + K_h (V^k) \widetilde{V} = \nu \Delta \widetilde{V} \]
	\[ \widetilde{V}|_r = 0 \]
	\item корректор
	\[ V^{k+1} = \widetilde{V} = \tau grad p^{k+1} \]
	\[ \frac{V^{k+1}-V^k}{} \]
	\[ div V^{k+1} = div \widetilde{V} - \tau div grad P^{k+1} = 0 \]
	\[ div grad p = \frac{div widetilde{V}}{\tau} \]
\end{enumerate}

\subsection{Исследование устойчивости}

\[ \frac{\hat{u}_{ij+\frac{1}{2}} - u_{ij+\frac{1}{2}}}{\tau} + \frac{u^2_{i+\frac{1}{2}j+\frac{1}{2}} - U^2_{i-\frac{1}{2}j+\frac{1}{2}}}{h^x} + \frac{(uV)_{ij+1} - (uV)_{ij}}{h^y} = - \frac{P_{i+\frac{1}{2}j+\frac{1}{2}}}{h^x} + \nu \Delta u_{ij+\frac{1}{2}} \]

\[ \frac{u_{i+1j+\frac{1}{2}} - u_{ij+\frac{1}{2}}}{h_x} + \frac{V_{i+\frac{1}{2}j+1}-V_{i+\frac{1}{2}j}}{h_y} = 0 \]

Процедура линеаризации

\[ u = a + \overline{u} \]
\[ V = b + \overline{V} \]
\[ p = p_0 + \overline{p} \]
\[ \overline{u} << 1 \]
\[ \overline{V} << 1 \]
\[ \overline{p} << 1 \]

\[ K= (a + \overline{u})^2_{\overline{x}} + {[(a + \overline{u})(f+\overline{V})]}_y = \]
\[ = (a^2 + 2a \overline{u} + \overline{u}^2)_{\widetilde{x}} (ab + b \overline{u} + a \overline{V} + \overline{u} \overline{V})_y = \]
\[ = 2 a \overline{u}_{\widetilde{x}} + b \overline{u}_y + a \overline{V}_y = \]
\[ = a \overline{u}_{\widetilde{x}} + b \overline{u}_y + a(\overline{u}_{\widetilde{x}}\overline{V}_y) \]
\[ K = a \overline{u}_{\widetilde{x}} + b \overline{u}_y \]

Результат линеаризации:

\[ \frac{\hat{\overline{u}}_{ij+\frac{1}{2}}-\overline{u}_{ij+\frac{1}{2}}}{\tau} + a \hat{\overline{u}}_{\widetilde{x}} + b \hat{\overline{u}}_{y} = - \hat{p}_{\hat{x}} + \nu \Delta \hat{\overline{u}} \]
\[ \frac{\overline{V_{i+\frac{1}{2}j}- \overline{V}_{i+\frac{1}{2}}}}{\tau} + a \hat{\overline{V}}_y = - \hat{p}_y + \nu \Delta \hat{\overline{V}} \]
\[ \hat{\overline{u}}_x + \hat{\overline{V}}_y = \nu \]

\section{Лекция}

\subsection{Устойчиость схемы для Н-С}

Система уравнений:
\[ \frac{\hat{U} - U}{\tau} + a u_{\overset{x}{0}} + b U \overset{y}{0} = -P \hat{x} + \nu (U_{\overline{x}x} + U_{\overline{y}y}) \]
\[ \frac{\hat{U}-V}{\tau}... \]
\[ div V = 0 \]
Конец системы

\[ \phi = l_1 h^x, \Theta = l_1 h^y \]
\[ U^k_{mn} = U^k(x_m, y_k) = U^k e^{i(l_1x + l_2 y)} = q^k U_a e^{i(m\phi + n \Theta)} \]
\[ U_a \frac{q^{k+1}-q^k}{\tau} e^{i(m\phi + (n+\frac{1}{2} \Theta))} + a U_a q^{k+1} e^{i(n+\frac{1}{2})\Theta} \frac{e^{i(m+1)\phi} - e^{i(m-1)\phi}}{wh^x} + b U_a q^{k+1} e^{im\pi}\frac{e^{i(n+\frac{3}{2})\Theta}-e^{i(n-\frac{1}{2})\Theta}}{2h^y} = \]
\[ - q^k U_p \frac{e^{i(m+\frac{1}{2}\phi+(n+\frac{1}{2})\Theta) - e^{i(m-\frac{1}{2})\phi + (n+\frac{1}{2}\Theta)}}}{h^x} + \nu[\frac{1}{{(h^x)}^2} q^{k+1} U_a(e^{i[(m+1)\phi+(n+\frac{1}{2})\Theta]})-2e^{i[m\phi + (n+\frac{1}{2})\Theta]} + e^{i((m-1)\phi + (m+\frac{1}{2})\Theta)} = \]
\[ = U_a \frac{q^{k+1} q^{k}}{\tau} + U_a q^{k+1} a \frac{e^{i\phi}-e^{-i\phi}}{2h^x} + U_a b q^{k+1} \frac{e^{i\Theta} - e^{-i\Theta}}{2h^y} \]
\[ = -q^k P_a \frac{e^{i\frac{\phi}{2}} - e^{-i\frac{\phi}{2}}}{h^x} + \nu q^{k+1} \cdot [\frac{e^{i\phi - 2 + e^{-i\phi}}}{(h^x)^2} + \frac{e^{i\Theta} - 2 + e^{-i\Theta}}{(h^y)^2}] = \]
\[ = U_a \frac{1-\frac{1}{q}}{\tau} + a U_a \frac{i sin \phi}{h_x} + b U_a \frac{i sin \Theta}{h^y} =  \]
\[ = \frac{2 i sin \frac{\phi}{2}}{h^y} = -P_a \frac{2 i sin \frac{\phi}{2}}{h^x} + \nu U_a [ \frac{1}{(h^x)^2}(wisin\frac{\phi}{2})^2 + \frac{1}{(h^y)^2} (2 i sin \frac{Q}{2})^2] =  \]
\[ = [\frac{1-\frac{1}{q}}{\tau} + \frac{ia}{h^x} sin \phi + \frac{ib}{h^y} sin \Theta + \nu 4(\frac{{sin}^2\frac{\phi}{2}}{(h^x)^2} + \frac{{sin}^2 \frac{\Theta}{2}}{(h^y)^2})] + \frac{P_n}{h^x} 2 i {sin} \frac{\phi}{2} = 0 \]
Результат:
\[ \mathcal{Z}u_a + O V_a + \frac{2i P_a}{h_x} {sin}\frac{\phi}{2} = 0 \]
То же самое нужно проделать для второй компаненты, полностью аналогично:
\[ (OU_a + \mathcal{Z} V_a + \frac{29}{h_y} {sin}\frac{Q}{2} P_a) = 0  \]
Дивергенция:
\[ U_a a^{k+1} \frac{e^{i(m+1)\phi + (n+\frac{1}{2}\Theta)} - e^{i[m\phi + (n +\frac{1}{2})\Theta]}}{h^x} + V_a a^{k+1} \frac{e^{i(m+\frac{1}{2})\phi + (n+1)\Theta} - e^{i[h_x(m+\frac{1}{2}\phi + n\phi)]}}{h^y} = 0 \]
\[ \frac{U_a}{h_x}(e^{i\frac{\phi}{2} - e^{-i\frac{\phi}{2}}}) + \frac{V_b}{h_y}(e^{i\frac{\Theta}{2}} - e^{-i\frac{Q}{2}}) = 0 \]

Третье уравнение:
\[ \frac{1}{h_x} {sin}\frac{\phi}{2}U_a + \frac{1}{h_y} {sin} \frac{\Theta}{2} V_a + O P_a = 0 \]
Итого система:
\begin{enumerate}
	\item \[ \mathcal{z}u_a + O V_a + \frac{2i P_a}{h_x} {sin}\frac{\phi}{2} = 0 \]
	\item \[ (OU_a + \mathcal{Z} V_a + \frac{29}{h_y} {sin}\frac{Q}{2} P_a) = 0  \]
	\item \[ \frac{1}{h_x} {sin}\frac{\phi}{2}U_a + \frac{1}{h_y} {sin} \frac{\Theta}{2} V_a + O P_a = 0 \]
\end{enumerate}
посчиатем определитель коэффициентов этой системы:
\begin{enumerate}
	\item \[ \mathcal{z}u_a + O V_a + \frac{2i P_a}{h_x} {sin}\frac{\phi}{2} = 0 \]
	\item \[ (OU_a + \mathcal{Z} V_a + \frac{29}{h_y} {sin}\frac{Q}{2} P_a) = 0  \]
	\item \[ \frac{1}{h_x} {sin}\frac{\phi}{2}U_a + \frac{1}{h_y} {sin} \frac{\Theta}{2} V_a + O P_a = 0 \]
\end{enumerate}
\[ = - \mathcal{Z} \frac{1}{(h^x)^2} 2i {sin}^2 \frac{\phi}{2} - \mathcal{Z} \frac{1}{h^y}^2 2i {sin}^2 \frac{\Theta}{2} = 0 \Rightarrow \mathcal{Z} = 0\]
Подставим z
\[ \mathcal{Z} = \frac{1 + \frac{1}{q}}{\tau} + \frac{ia}{h^x} {sin} \phi + \frac{ib}{h^y} \sin \Theta + 4\nu (\frac{{sin}^2 \frac{\phi}{2}}{(h^x)^2} + \frac{{sin}^2\frac{Q}{2}}{(h^y)^2}) = 0 \]
\[ |\frac{1}{q}| < 1; |q| < 1\]
Следовательно схема безусловна устойчива при этом условии

Для явной схемы
\[ \mathcal{Z} = \frac{q-1}{\tau} + \frac{ia}{h_z} {sin} \phi + \frac{ib}{h_y} {sin} \Theta + 4 \nu (\frac{{sin}^2 \frac{\phi}{2}}{ (h^x)^2} + \frac{{sin}^2 \frac{\Theta}{2}}{(h^y)^2}) \]
Так как мы ищем достаточное условие можем взять
\[ h = h^x = h^y \]
\[ a = b: \phi = \Theta \]

\[ \frac{q-1}{\tau} + \frac{ia}{h}sin \phi + \frac{ib}{h} sin \phi + 4 \nu (\frac{{sin}^2\frac{\phi}{2}}{h^2} + \frac{{sin}^2 \frac{\phi}{2}}{h^2}) = 0 \]
\[ \frac{q-1}{\tau} + \frac{i(a+b)}{h} sin \phi + 8 \nu (\frac{{sin}^2 \frac{\phi}{2}}{h^2}) = 0 \]
\[ q = - \tau \frac{i(a+b)}{h} sin \phi + 8 \nu (\frac{{sin}^2 \frac{\phi}{2}}{h^2}) + 1 \]
\[ q = \frac{-\tau i(a+b) sin \phi}{h} + \frac{8 \nu (sin^2 \frac{\phi}{2})}{h^2} + 1 \]
\[ q = \frac{-2\tau ia sin \phi }{h}+ \frac{8 \nu (sin^2 \frac{\phi}{2})}{h^2} + 1 \]

\[ q = (1 - 8\frac{\nu \tau}{h^2} {sin^2} \frac{\phi}{2}) - \frac{2i\tau a}{h} {sin} \phi \]
Устойчив при модуле q < 1

\[ {|q|}^2 = (1 - \frac{8 \tau \nu}{h^2})^2 + \frac{4 \tau^2 a^2}{h^2} {sin}^2 \phi < 1\]
\[ 1 - \frac{16\tau\nu}{h^2} {sin}^2 \frac{\phi}{2} + \frac{64 \tau^2 \nu^2}{h^4} {sin}^4 \frac{\phi}{2} + \frac{4\tau^2 a^2}{h^2} {sin}^2 \phi < 1 \]
\[ \frac{4 \nu} {sin}^2 \frac{\phi}{2} + \frac{16 \tau \nu^2}{h^2} {sin}^4 \frac{\phi}{2} + 4 \tau a^2 {sin}^2 \frac{\phi}{2} \cos^2 \frac{\phi}{2} \]
\[ \nu + \frac{4\tau \nu}{h^2} {sin}^2 \frac{\phi}{2} \tau a^2 {sin}^2 \frac{\phi}{2} {cos}^2 \frac{\phi}{2} < 0 \]
\[ -\nu + \tau [\frac{4 \nu {sin}^2 \frac{\phi}{2}}{h^2} + a^2 {cos}^2 \frac{\phi}{2}] < \nu \]
\[ \tau < \frac{\nu}{4 \nu \frac{{sin}^2 \frac{\phi}{2}}{^2}} + a cos^2 \frac{\phi}{2} \]
\[ \tau \sim h^2 \]

\subsection{Задача Стефана}

Задача о фазовом переходе (например плавление или замерзание). Задача с подвижной внутренней границей.

В твердой фазе:
\[ c_p \rho_p \frac{\partial T}{\partial  t} = \frac{\partial }{\partial x}(\kappa^s \frac{\partial T}{\partial x}) \]

\[ c_l \rho_l \frac{\partial T}{\partial  t} = \frac{\partial }{\partial x}(\kappa^l \frac{\partial T}{\partial x}) \]

\[ T_{ph} = 0 \]

Необходимо еще одно условие чтобы получить границу раздела сред, где одно уравнение переходит в другое.

\[ \epsilon^s = C^S_p \rho^S (T - T_0) \]
\[ \epsilon^l = c_p^S \rho^s(T_{ph}-T_0) + \lambda \rho + C_p^l \rho^l (T-T_{ph}) \]

\[ E =0 \int_{0}^{\xi(t)} \epsilon^s dx + \int_{\xi(t)}^{L} \epsilon^l dx\textrm{ - внутренняя энергия всей системы } \]
\[ \frac{dE}{dt} = \int_{0}^{\xi(t)} \frac{\partial \epsilon^S}{\partial t} dx + \xi(t) \epsilon^S |_{x= \xi(t)} + \int_{\xi(t)}^{L} \frac{\partial  \epsilon^l}{\partial t} dx - \xi \epsilon^l |_{x = \xi(t)} = \]
\[ =\int_{0}^{\xi(t)} C_p^S \rho^S \frac{\partial }{\partial t}(T - T_0) dx + \xi(t) [\epsilon^S|_{x=\xi(t)} - \epsilon^l |_{x-\xi(t)}] \int_{\xi(t)}^{L} [C_P^S \rho^S (T_{Ph - T_0}) + \lambda \rho + C_P^l(T-T_0)] dt = \]
\[ = \int_{0}^{\xi(t)} \frac{\partial }{\partial x} (\kappa^S \frac{\partial T}{\partial x}) dx + \int_{\xi(t)}^{L} \frac{\partial }{\partial x}(\kappa^l \frac{\partial T}{\partial x})sx + \xi(t) [\epsilon^S|_{x=\xi(t)}] = \]
\[ = \kappa^S \frac{\partial T}{\partial  x}|_{x=\xi(t)} - \kappa^S \frac{\partial T}{\partial x}|_{x=0} + \kappa^l\frac{\partial T}{\partial x}|_{x=L} - \kappa^2\frac{\partial T}{\partial x}|_{x=\xi(t)} + \xi(t) [ C^S_p \rho^S (T_{PH} - T_0) - C^S_p \rho^S(T_{PH - T_0}) - \lambda \rho - C^l_p - C_p^l p^l(T_{PH} - T_{PH})] \]
Условие Стефана:
\[ \kappa^S \frac{\partial T}{\partial x}|_{{}\xi(t)}^s - \kappa^l\frac{\partial T}{\partial x}|_{{\xi(t)}^l} = \lambda \rho \overset{\xi}{\cdot} \]

\[ \kappa = \ae \]

\section{Лекция 10}

\begin{enumerate}
	\item Подход подвижной сетке закрепляем узлы как на резинке, и они двигаются вместе с перемещением границы раздела.

	\item Замена переменных так чтобы левая граница была 0, граница раздела 1, правая граница 2.
\end{enumerate}

\subsection{Замена переменных (неподвижная сетка)}

Замена переменных (переход в новую систему переменных)
\[ \widetilde{t} = t \]
\[ x = \phi (t, y) \{y \xi(t) - "S" \textrm{ (твердая фаза) }; \xi(t) + (L - \xi(t)) (y-1) "l" \textrm{ (жидкая фаза) } \} \]

\[ y = \frac{x}{\xi(t)} - "s"; y = 1 + \frac{x - \xi(t)}{l^l} - "l" \]

Можно проверить что система невырождена, посчитав якобиан:
\[ l^s \{ \textrm{ длинна твердой фазы } \} \]
\[ l^l \{ \textrm{ длинна жидкой фазы } \} \]

Отображение становится вырождено только если длинна одной из зон станет равной нулю.

\[ c_p \rho \frac{\partial T}{\partial \widetilde{t}} = \frac{\partial }{\partial x}( \ae \frac{\partial T}{\partial x}); T(t, y) = T (t(\widetilde{t}), y(\widetilde{t}, x)) \]

\[ \frac{\partial T}{\partial x} = \frac{\partial T}{\partial y} \frac{\partial y}{\partial x}, \frac{\partial }{\partial x} = \frac{\partial }{\partial y} \cdot \frac{\partial y}{\partial x} \]
\[ \frac{\partial }{\partial \widetilde{t}} = \frac{\partial }{\partial t} \frac{\partial t}{\partial  \widetilde{t}} + \frac{\partial }{\partial y}\frac{\partial y}{\partial  \widetilde{t} = \frac{\partial }{\partial t}} + \frac{\partial }{\partial y}\frac{\partial y}{\partial \widetilde{t}} \]

дифференциалы прямого и обратного преобразования:

x:

\[ d \widetilde{t} = dt \]
\[ dx = \frac{\partial \phi}{\partial t} dt + \frac{\partial \phi}{\partial y}dy \]
Инварианты с матрицами
\[
\begin{pmatrix}
	d \widetilde{t} \\
	d x
\end{pmatrix}
 =
\begin{pmatrix}
	1 & 0 \\
	\frac{\partial \phi}{\partial t} & \frac{\partial \phi}{\partial t}
\end{pmatrix}
\begin{pmatrix}
	dt \\
	dy
\end{pmatrix}
\]

Умножить его на обратную матрицу.

\[
\begin{pmatrix}
	d t \\
	d x
\end{pmatrix}
 =
\begin{pmatrix}
	1 & 0 \\
	\frac{\partial \phi}{\partial t} & \frac{\partial \phi}{\partial t}
\end{pmatrix}^{-1}
\begin{pmatrix}
	d \widetilde{t} \\
	dy
\end{pmatrix}
\]
\[
\begin{pmatrix}
	d t \\
	d y
\end{pmatrix}
 =
\begin{pmatrix}
	1 & 0 \\
	\frac{-\frac{\partial \phi}{\partial t}}{\frac{\partial \phi}{\partial y}} & \frac{1}{\frac{\partial \phi}{\partial y}}
\end{pmatrix}
\begin{pmatrix}
	d \widetilde{t} \\
	dx
\end{pmatrix}
\]

y:
\[ dt = d\widetilde{t} \]
\[ dy = \frac{\partial y}{\partial  \widetilde{t}} d \widetilde{t} + \frac{\partial y}{\partial x} dx \]
Инварианты с матрицами

\[
\begin{pmatrix}
	d t \\
	d y
\end{pmatrix}
 =
\begin{pmatrix}
	1 & 0 \\
	\frac{\partial y}{\partial \widetilde{t}} & \frac{\partial y}{\partial t}
\end{pmatrix}
\begin{pmatrix}
	d \widetilde{t} \\
	dy
\end{pmatrix}
\]

тогда посчитаем
\[ \frac{\partial y}{\partial x} = \frac{1}{\frac{\partial \phi}{\partial y}} \begin{pmatrix}
	\frac{1}{l^s} \\
	\frac{1}{l^l}
\end{pmatrix}
\frac{1}{l}
\]

\[ \frac{\partial y}{\partial \widetilde{t}} = - \frac{\frac{\partial \phi}{\partial t}}{\frac{\partial \phi}{\partial y}} = - \frac{\phi_t}{l} \]

Итого
\[ \frac{\partial }{\partial \widetilde{t}} = \frac{\partial }{\partial t} \frac{\partial t}{\partial  \widetilde{t}} + \frac{\partial }{\partial  y} \frac{\partial y}{\partial \widetilde{t}} = \frac{\partial }{\partial t} + \frac{\partial }{\partial y} \frac{\partial y}{\partial \widetilde{t}} \]
\[ c(\frac{\partial t}{\partial t} - \frac{\phi_t}{l} \frac{\partial T}{\partial y}) = \frac{1}{l} \frac{\partial }{\partial y} (\ae \frac{1}{l}\frac{\partial T}{\partial y}) \]

Итого получаем совсем другое уровнение, но по постоянным областям.

Теперь перейдем к границному условию:
\[ \frac{\ae}{l^s} \frac{\partial T}{\partial y}|_{1-0} - \frac{\ae}{l^l}\frac{\partial  T}{\partial y}|_{1+0} \]
\[\ae \frac{\partial T}{\partial x} |_{x=\xi(t-0)} - \ae \frac{\partial T}{\partial x}_{\xi(t) + 0} = \lambda \frac{d\xi}{dt} \]

Теперь апроксимируем это граничное условие

\[ c (l \frac{\partial T}{\partial t}) - (\phi_t \frac{\partial T}{\partial y}) = \frac{\partial }{\partial y}(\ae \frac{1}{l}\frac{\partial T}{\partial y}) \]

\[ c(\frac{\partial }{\partial t}(lT)) - T \frac{\partial l}{\partial t} - \frac{\partial }{\partial y} (\phi_t T) + T \phi_{ty} \]
\[ \phi_{ty} = \frac{\partial l}{\partial t} \]

\[ c(\frac{\partial }{\partial t}(lT)) - \frac{\partial }{\partial y} (\phi_t T) = \frac{\partial }{\partial y} (\frac{\ae}{l} \frac{\partial T}{\partial y}) \]

Пусть узел связанный с границей раздела фаз называется $ y_{i*} $

Вводим потоковые точки связанные с серединами отрезков, а температуру относим к узлам.

$\ae$  и  $c$ Относим к потоковым точкам, так как они зависят от фазы отрезка, также как и l (Длинна отрезка l зависит от отрезка).

Нужно написать алгоритм однородный отностельно границы раздела сред.

\[ \int_{t_j}^{j+1} \int_{y_{i-\frac{1}{2}}}^{y_{i+\frac{1}{2}}} c [\frac{\partial }{\partial t}(lT) - \frac{\partial }{\partial y}(\phi_t T)] dt dy = \int_{t_j}^{t_{j+1}} \int_{y_{i-\frac{1}{2}}}^{y_{i+\frac{1}{2}}} \frac{\partial }{\partial y} (\frac{\ae}{l}\frac{\partial T}{\partial y}) dt dy \]

Первый интеграл:

\[ I_1 = \int_{t}^{} \int_{y_{i-\frac{1}{2}}}^{y_{i+\frac{1}{2}}} c \frac{\partial}{\partial t} (lT) dy = \int_{t_j}^{t_{j+1}} dt \frac{\partial }{\partial t} \int_{y{i-\frac{1}{2}}}^{y_{i+\frac{1}{2}}} c (lT) d y = \int_{t_j}^{t_{j+1}} dt \frac{\partial }{\partial t}[\int_{y_{i-\frac{1}{2}}}^{y_i} c l T dy + \int_{y^{i+\frac{1}{2}}}^{y_{i}} clT dy ] =  \]

\[ = \int_{t_j}^{t_{j+1}} dt \frac{\partial }{\partial t}[c_{i-\frac{1}{2}} (l_{i-\frac{1}{2}}T_i)h_{i-\frac{1}{2}}\cdot \frac{1}{2} + c_{i+\frac{1}{2}}(l_{i+\frac{1}{2}} T_i) h_{i+\frac{1}{2}}\frac{1}{2}] = \]
\[ = \frac{h_{i-\frac{1}{2}}}{2} c_{i-\frac{1}{2}} [(l_{i+\frac{1}{2}} T_i) - (l_{i-\frac{1}{2}}T_i)] + \frac{h_{i+\frac{1}{2}}}{2} c_{i+\frac{1}{2}}[\widehat{(l_{i+\frac{1}{2}}T_i)} - (l_{i+\frac{1}{2}T_i})] \]

Второй интеграл:

\[ I_2 = \int_{t_j}^{t_{j+1}} dt [\int_{y_{i-\frac{1}{2}}}^{y_i} c_{i-\frac{1}{2}}\frac{\partial }{\partial y}(\phi_t T) dy \int_{y_i}^{y_{i+\frac{1}{2}}} c_{i+\frac{1}{2}} \frac{\partial }{\partial y} (\phi_t T) dy] =\]

\[ = \int_{t_i}^{t_{i+1}} [c_{i-\frac{1}{2}}[(\phi_t T)(y_i) - (\phi_t T)(y_{i-\frac{1}{2}})] + [c_{i+\frac{1}{2}} ((\phi_t T)(y_{i+\frac{1}{2}})-(\phi_tT)(y_i))]] = \]

\[ \phi_t|_{i+0} = \phi_t|_{i-0} \]
\[ \phi_t|_{i=i^*} = \frac{d\xi}{dt}; \phi = y \xi \]
\[ \phi_t = y \cdot \overset{\cdot}{\xi} \]
\[ {l''} \phi_t = (2-y) \overset{\cdot}{\xi} \]

\[= \int_{t_j}^{t_{j+1}}(c_{i-\frac{1}{2}} - c_{i+\frac{1}{2}}) \frac{d\xi}{dt}T + c_{i+\frac{1}{2}}(\phi_t T y_{i+\frac{1}{2}} - c_{i-\frac{1}{2}} (\phi_t T)_{y (i-\frac{1}{2})}) dt \]

\[ (\phi_t T)_{i+\frac{1}{2}} = \phi_t(i+\frac{1}{2}) \frac{T_i + T_{i+1}}{2} \]

Остался последний интеграл

\[ \frac{\partial }{\partial y} (\frac{\ae}{l}\frac{\partial T}{\partial y}) \]
\[ \int_{t_j}^{t_j+1} dt [\int_{y_{i-\frac{1}{2}}}^{y_i} \frac{\partial }{\partial y} (\frac{\ae}{l}\frac{\partial T}{\partial y}) dy + \int_{y_i}^{y_{i+\frac{1}{2}}} \frac{\partial }{\partial y}(\frac{\ae}{l} \frac{\partial T}{\partial y}) dy] =  \]

\[ = \int_{t_j}^{t_j+1} dt[(\frac{\ae}{l}\frac{\partial T}{\partial y})_{y_{i-0}} - (\frac{\ae}{l} \frac{\partial T}{\partial y})_{y_{i-\frac{1}{2}}} + (\frac{\ae}{l}\frac{\partial T}{\partial y})_{y_{i+\frac{1}{2}}} - (\frac{\ae}{l} \frac{\partial T}{\partial y})_{y_i + 0}] =  \]
\[ = \int_{t_j}^{t_{j+1}} [ \frac{d\xi}{dt} \lambda \delta_{ii*} + (\frac{\ae}{l} \frac{\partial T}{\partial y})_{i+\frac{1}{2}} - (\frac{\ae}{l} \frac{\partial T}{\partial y})_{i-\frac{1}{2}}] dt \]

\section{Лекция}

\subsection{Разностная схема на подвижной сетке}

Вводим сетку так, чтобы один узел был всегда на границе раздела фаз ($ \xi(t) $).
Если двигается узел $ \xi(t + \tau) $, то все узлы справа и слева сдвигаются, так чтобы расстояние между узлами были равны.

\[ h_{i+\frac{1}{2}}(t) x_{in}(t) - x_i(t) \]

\[ h_{i+\frac{1}{2}} = x+i+1(t + \tau) - x_i (t + \tau) \]

\[ h_i(t) = \cancel{h_i} = \frac{h_{i+\frac{1}{2}} + h_{i-\frac{1}{2}}}{2} \]

\[ \cancel{f_x} = \frac{f(x_{i+1}(t)) - f(x_i(t))}{\cancel{h_i+\frac{1}{2}}(t)} \]

Строим разностную схему

\[ c \frac{\partial T}{\partial t} = \frac{\partial }{\partial x}(\ae \frac{\partial T}{\partial x}) \]

*Картинка с узлами и отрезком интегрирования*

\[ \int_{t_j}^{t_{j+1}} \int_{x_{i-\frac{1}{2}}(t)}^{i+\frac{1}{2}(t)} c \frac{\partial T}{\partial t} dt dx =  \]

вспомним как дифференцируется (формула интегрирования по переменным границам или формула интегрирования по параметру):

\[ \frac{d}{dt}\int_{x_{i+\frac{1}{2}}}^{x_{x+\frac{1}{2}}} (c T) dx k =  \]
\[ = \int_{x_{i-\frac{1}{2}}(T)}^{i+\frac{1}{2}(t)} \frac{\partial }{\partial t} (cT) dx + (cT)_{i+\frac{1}{2}} \frac{dx_{i+\frac{1}{2}}}{dt} - (CT)_{i-\frac{1}{2}} \frac{dx_{i-\frac{1}{2}}}{dt} \]

Применим эту формулу к интегралу:

\[ I_1 = \int_{t_j}^{t_j+1} dt [\int_{x_{i+\frac{1}{2}}}^{x_i(t)} \frac{\partial }{\partial t} (CT)dx + \int_{x_i}^{x_{i+\frac{1}{2}}}\frac{\partial }{\partial t} (CT)dx ] = \]
% \[= \int_{t_j}^{t_j+1} dt [\frac{d}{dt} \int_{x_{i-\frac{1}{2}}}^{x_{i+\frac{1}{2}}}(CT)dx - (CT)_{i+\frac{1}{2}} dx] \]

\[ = \int_{t_j}^{t_{j+1}} dt [\frac{d}{dt} \int_{x_{i+\frac{1}{2}}}^{x_i}(CT) dx - (CT) dx - (CT) \frac{dv_i}{dt} + (CT)_{x_{i-\frac{1}{2}}} \frac{ dx_{i-\frac{1}{2}}}{dt} + \frac{d}{dt} \int_{v_i(t)}^{x_i+\frac{1}{2}(t)} (CT) dx - (CT)_{x_i+\frac{1}{2}} \frac{dx_{i+\frac{1}{2}}}{dt} + (CT)_{x_i+0} \frac{dx_i}{dt} ] =\]

\[ = \frac{c_{i-\frac{1}{2}}}{2} [ \hat{ h_{i-\frac{1}{2}}^w T_i} - h^x_{i+\frac{1}{2}} T_j] + \frac{c_{i+\frac{1}{2}}}{2} [\hat{h^x_{i+\frac{1}{2}}T_i} - h^x_{i+\frac{1}{2}} T_i] - \tau [{(CT^(\sigma) x_t)}_{i+\frac{1}{2}} - {(CT^{\sigma} x_t)}_{i-\frac{1}{2}}] - \tau [(c_{i-\frac{1}{2}} - c_{i+\frac{1}{2}}) T^{(\sigma)}_i(x_i)_t] \]

\[ T^{\sigma} = \hat{T} + (1-\sigma)T \]

Шапка значит что $ h^x_{i-\frac{1}{2}}(t_{j+1}) +  $ (берем по следующему временному слою)

\[ \frac{dx_i}{dt} = \frac{\hat{x_i} - x_i}{\tau} \]

\[ I_2 = \int_{t_j}^{t_{j+1}} \int_{x_{i-\frac{1}{2}}(t)}^{x_i+\frac{1}{2}(t)} \frac{\partial }{\partial x} (\ae \frac{\partial T}{\partial x}) dt dx =  \]

\[ \int_{t_j}^{t_{j+1}} dt [\int_{x_{i-\frac{1}{2}}(t)}^{x_i(t)} \frac{\partial }{\partial x}(\ae \frac{\partial T}{\partial x}) dx + \int_{x_i}^{x_{i+\frac{1}{2}}} \frac{\partial }{\partial x} (\ae\frac{\partial T}{\partial x} dx)] = \]

\[ \int_{t_{j}}^{t_{j+1}} dt [ \ae \frac{\partial T}{\partial x} |_{x_i(t)} - \ae \frac{\partial T}{\partial x}|_{x_{i-\frac{1}{2}}(t)} \ae \frac{\partial T}{\partial x}|_{x_i}] =  \]

\[ = \tau \frac{dx_i}{dt} \lambda \delta_{ij} + \tau [\ae_{i+\frac{2}{2}} \frac{\partial T}{Px}|_{i+\frac{1}{2}} - \ae_{{i-\frac{1}{2}}} \frac{\partial T}{\partial x}|_{i-\frac{1}{2}}] \]

\subsection{задача Стефана в двумерном пространстве}

Рассмотрим на фиксированной сетке, на подвижной рассматривать не будем так как нет времени, а они эквивалентны вплоть до замены переменных.

Нарисуем цилиндр со стенками конечной толщины.  внутри есть жидкая фаза "l" и твердая фаза "S".

Плотность снова одинаковая

\[ \frac{\partial T}{\partial t} =\frac{1 }{\widetilde{r}} (\widetilde{r} \ae \frac{\partial T}{\partial \widetilde{r}}) + \frac{\partial }{\partial z}(\widetilde{r} \ae \frac{\partial T}{\partial \widetilde{t}}) \]

\[ {||\ae \triangledown T \cdot N ||} = S t V_{{Ph}(l_z, N)} \]

N - нормаль границы раздела

Условия на границе:

\[ \ae \frac{\partial T}{\partial n}|_S = \ae \frac{\partial T}{\partial n}|_l \]

\[\frac{\partial T}{\partial }  \]

Границы раздела - $ \widetilde{z_1} $ - дно, $ \widetilde{z_2} $ - граница раздела сред, $ \widetilde{z_3} $ - крышка, $ \widetilde{z_0}, \widetilde{z_4} $ - нижняя поверхность, верхняя поверхность

Переводим эти кривые в прямые линии и получаем разделение по отдельным слоям.

Сделаем замену переменных:

\[ t = \widetilde{t} \]

\[ r = \widetilde{r} \]

\[ z = \frac{\widetilde{z} - \widetilde{z_{\gamma}}}{\widetilde{l^{\gamma}}} \]

Например при $ \gamma = 0 $:

\[ z = \frac{\widetilde{z} - z_0}{l^0} \]

\[ \gamma = 0, 1, 2, 3 \]

Делаем замену переменных

$$
\left\{ \begin{array}{c}

\frac{\partial T}{\partial \widetilde{t}} = \frac{\partial T}{\partial t} + \frac{\partial T}{\partial z} \frac{\partial z}{\partial \widetilde{t}} \\

\frac{\partial T}{\partial \widetilde{r}} + \frac{\partial T}{\partial z}\frac{\partial z}{\partial \widetilde{r}} \\

\frac{\partial T}{\partial \widetilde{z}} = \frac{\partial T}{\partial z} \frac{\partial z}{\partial \widetilde{z}} \\

\end{array} \right.
$$

\[ d \widetilde{t} = dt \]

\[ d \widetilde{r} = dr \]

\[ d \widetilde{z} = \frac{\partial \phi}{\partial t} dt + \frac{\partial \phi}{\partial r} dr + \frac{\partial \phi}{\partial z} d z\]

\[ d \widetilde{t} = dt \]

\[ d \widetilde{r} = dr \]

\[ d \widetilde{r} = \frac{\partial z}{\partial \widetilde{t}} d \widetilde{t} d \widetilde{t} + \frac{\partial z}{\partial \widetilde{r}}d \widetilde{z} d \widetilde{z} \]

$$
\begin{bmatrix}
  d \widetilde{t} \\
  d \widetilde{r} \\
  d \widetilde{z}
\end{bmatrix}
=
\begin{bmatrix}
  1 & 0 & 0 \\
  0 & 1 & 0 \\
  \frac{\partial \phi}{\partial t} & \frac{\partial \phi}{\partial r} & \frac{\partial \phi}{\partial z}
\end{bmatrix}
\cdot
\begin{bmatrix}
    dt \\
    dr \\
    dz
\end{bmatrix}
$$

$$
\begin{bmatrix}
  d t \\
  d r \\
  d z
\end{bmatrix}
=
\begin{bmatrix}
  1 & 0 & 0 \\
  0 & 1 & 0 \\
  \frac{\partial \phi}{\partial \widetilde{t}} & \frac{\partial \phi}{\partial \widetilde{r}} & \frac{\partial \phi}{\partial \widetilde{z}}
\end{bmatrix}
\cdot
\begin{bmatrix}
    d\widetilde{t} \\
    d\widetilde{r} \\
    d\widetilde{z}
\end{bmatrix}
$$
$$
\begin{bmatrix}
  1 & 0 & 0 \\
  0 & 1 & 0 \\
  \frac{\partial \phi}{\partial t} & \frac{\partial \phi}{\partial r} & \frac{\partial \phi}{\partial z}
\end{bmatrix}^(-1)
=
\begin{bmatrix}
  1 & 0 & 0 \\
  0 & 1 & 0 \\
  \frac{\partial \phi}{\partial \widetilde{t}} & \frac{\partial \phi}{\partial \widetilde{r}} & \frac{\partial \phi}{\partial \widetilde{z}}
\end{bmatrix}
$$

$$
\begin{bmatrix}
  1 & 0 & 0 \\
  0 & 1 & 0 \\
  \frac{\partial \phi}{\partial t} & \frac{\partial \phi}{\partial r} & \frac{\partial \phi}{\partial z}
\end{bmatrix}^(-1)
=
\begin{bmatrix}
  1 & 0 & 0 \\
  0 & 1 & 0 \\
  \frac{-\frac{\partial \phi}{\partial t}}{\frac{\partial \phi}{\partial z}} & \frac{-\frac{\partial \phi}{\partial r}}{\frac{\partial \phi}{\partial z}} & -\frac{1}{\frac{\partial \phi}{\partial z}}
\end{bmatrix}^(-1)
$$

\begin{tabular}{|c|c|c|c|}
  \hline
  $S_\gamma$ & $ \phi^\gamma_z = l $ & $\phi_t $ & $ r $ \\ \hline
  $S_0 $ & $ z_1 - z_0 $ & $ 0 $ & $ (z-1)z_{2, r}$ \\ \hline
  $S_1 $ & $ z_1(\widetilde{r} \widetilde{t} - z_1) $  & $ (z-1) z_{2, t} $ & \\ \hline
  $S_2 $ & $ z_3 - z_2(\widetilde{r, \widetilde{t}}) $ & $ (3 - z)z_{2, t} $ & \\ \hline
  $S_3$ & $ z_4 - z_3 $ & $ z_j $ &\\ \hline
\end{tabular}

Взять таблицу из лекций, но там есть опечатка!

\[ \frac{1}{r} [ \frac{\partial }{\partial \widetilde{r}} (\ae \widetilde{r} \frac{\partial T}{\partial  \widetilde{r}}) + \frac{\partial }{\partial \widetilde{z}}(\widetilde{r \ae \frac{\partial T}{\partial \widetilde{z}}})] \cdot r l = \]

\[ l [\frac{\partial }{\partial \widetilde{r}} Q_1 + \frac{\partial }{\partial \widetilde{z}} Q_2] \textrm{ , где } Q_1 = \ae \widetilde{r} \frac{\partial T}{\partial  \widetilde{r}}, Q_2 = \ae \widetilde{r} \frac{\partial T}{\partial \widetilde{z}} \]

\[ l \frac{\partial Q}{\partial r} = l \frac{\partial Q_1}{\partial r} - \phi_r \frac{\partial Q_1}{\partial z} = \frac{\partial }{\partial r} (l Q_1) - Q_1 \frac{\partial }{\partial r}(l Q_1) - Q_1 \frac{\partial l}{\partial r} - \frac{\partial }{\partial z} - \frac{\partial }{\partial z} (\phi_r Q_1) + Q_1 \frac{\partial \phi_r}{\partial z} = \]

\[ = \frac{\partial }{\partial r} (l Q_1) - \frac{\partial }{\partial z}(\phi_r Q_1) =  \]

\[ = \frac{\partial }{\partial r}(lr\ae (\frac{\partial T}{\partial r} - \frac{\phi_r}{l} \frac{\partial T}{\partial z})) = \]

\[ = \frac{\partial }{\partial z} (\phi_r r \ae (\frac{\partial T}{\partial r} - \frac{\partial _r}{l} \frac{\partial T}{\partial z})) = \]

\[ = \frac{\partial }{\partial r} (r \ae (l \frac{\partial T}{\partial r} - \phi_r\frac{\partial T}{\partial z})) - \frac{\partial }{\partial z} (r \ae (\phi_r \frac{\partial T}{\partial r} - \frac{\phi^2_r}{l}\frac{\partial T}{\partial z})) \]

\[ l \frac{\partial Q}{\partial \widetilde{z}} = \frac{\partial }{\partial z}(\ae r \frac{1}{l} \frac{\partial T}{\partial z}) \]

Введем следующие обозначения

\[ L^{rr} = l; Z^{rz} = Z^{zr} = - \phi_r; Z_zz = \frac{1 + \phi^2}{l} \]

Тогда:

\[ \mathcal{L} = \frac{\partial }{\partial r} [\ae r (L^{rr}\frac{\partial T}{\partial r} + L^{rz} \frac{\partial T}{\partial z})] + \frac{\partial }{\partial z}[\ae r (L^{zr} \frac{\partial T}{\partial r} + L^{zz}) \frac{\partial T}{\partial z}] \]

Этот оператор хороший - эллиптический, самосопряженный, самоопределенный доказать в качестве упражнения.

Подумать как записать краевые условия для выражений $ \frac{\partial T}{\partial n} = 0 $ и $ \frac{\partial T}{\partial n} = 0$

\end{document}